\documentclass[a4paper]{article}

\usepackage[spanish]{babel} % Le indicamos a LaTeX que vamos a escribir en espa�ol.
\usepackage[latin1]{inputenc} % Permite utilizar tildes y e�es normalmente
\usepackage{pdfpages}
%\usepackage{framed}
\input{Algo1Macros}% Macros especificas para especificar problemas en AyEDI

\newcommand{\comen}[2]{%
\begin{framed}
\noindent \textsf{#1:} #2
\end{framed}
}
% Aca solo vamos a poner el esqueleto del documento, pero no vamos a especificar nada.

\begin{document} % Todo lo que escribamos a partir de aca va a aparecer en el documento.

\includepdf{caratula}

\section{Tipos} %Defino los renombres de tipos b�sicos.

\sinonimo{Empleado}{String}
\sinonimo{Energia}{\ent}
\sinonimo{Cantidad}{\ent}
\enum{Bebida}{Pesti Cola, Falsa Naranja, Se ve nada, Agua con Gags, Agua sin Gags}
\enum{Hamburguesa}{McGyver, CukiQueFresco (Cuarto de Kilo con Queso Fresco), McPato, Big Macabra}


\section{Combo} %Defino el tipo combo, con su especificaci�n.


\begin{tipo}{Combo}
	\observador{bebida}{c: Combo}{Bebida}
	\observador{sandwich}{c: Combo}{Hamburguesa}
	\observador{dificultad}{c: Combo}{Energia}
	\medskip % Dejo un espacio entre los observadores y el invariante
	\invariante[dificultadHasta100]{energiaEnRango(dificultad(c))}
\end{tipo}
 % Aca va la definici�n del tipo.


\begin{tipo}{Combo}
	\observador{bebida}{c: Combo}{Bebida}
	\observador{sandwich}{c: Combo}{Hamburguesa}
	\observador{dificultad}{c: Combo}{Energia}
	\medskip % Dejo un espacio entre los observadores y el invariante
	\invariante[dificultadHasta100]{energiaEnRango(dificultad(c))}
\end{tipo}
 % La especificaci�n del tipo combo.

\newpage %Salto de p�gina


\section{Pedido} 


\begin{problema}{nuevoP}{n: \ent, e: Empleado, cs: [Combo]}{Pedido}
\requiere [numeroPositivo]{n > 0}
\requiere [hayaPedidoAlgo]{\vert cs \vert > 0}
\asegura {numero(res) == n}
\asegura {atendio(res) == e}
\asegura {combos(res) == cs}
\end{problema}

\begin{problema}{numeroP}{p: Pedido}{\ent}
\asegura {res == numero(p)}
\end{problema}

\begin{problema}{atendioP}{p: Pedido}{Empleado}
\asegura {res == atendio(p)}
\end{problema}

\begin{problema}{combosP}{p: Pedido}{[Combo]}
\asegura {combos(p) == res)}
\end{problema}

\begin{problema}{agregarComboP}{p: Pedido, c: Combo}{}
\modifica {p}
\asegura [mantieneNumero]{numero(p) == numero(pre(p))}
\asegura [mantieneEmpleado]{atendio(p) == atendio(pre(p))}
\asegura [agregoUnCombo]{combos(p) == combos(pre(p)) ++ [c]}   
\end{problema}

\begin{problema}{anularComboP}{p: Pedido, i:\ent}{}
\requiere {0 \leq i < |combos(p)|}
\modifica {p}
\asegura [mantieneNumero]{numero(p) == numero(pre(p))}
\asegura [mantieneEmpleado]{atendio(p) == atendio(pre(p))}
\asegura [anuloElComboDelParametro]{combos(p) == (combos(pre(p))_{[0..i)} ++ combos(pre(p))_{(i..|combos(pre(p))|)} )}  
\end{problema}


\begin{problema}{cambiarBebidaComboP}{p: Pedido, b: Bebida, i:\ent} {}
\requiere {0 \leq i < \vert combos(p) \vert}  
\modifica {p}
\asegura [mantieneNumero]{numero(p) == numero(pre(p))}
\asegura [mantiendoEmpleado]{atendio(p) == atendio(pre(p))}
\asegura [mantieneCantidadDeCombos]{\vert combos(p) \vert == \vert combos(pre(p)) \vert }
\asegura [mantieneCombosIguales]{(\forall \ j \leftarrow [0..\vert combos(p) \vert), j \neq i) \ combos(p)_j == combos(pre(p))_j  } 
\asegura [mantieneSandwich]{sandwich (combos(pre(p)_i) == sandwich (combos(p)_i)}
\asegura [mantieneDificultad]{dificultad (combos(pre(p)_i) == dificultad (combos(p)_i)} 
\asegura [cambioLaBebida]{bebida (combos(p)_i) == b}

\end{problema}

\newpage

\begin{problema}{elMezcladitoP}{p: Pedido}{}
\requiere [hayaPermutacionesSuficientes]{\vert combos(p) \vert \leq \vert bebidasUsadas(p) \vert \times \vert sandwichesUsados(p) \vert }
\modifica {p}
\asegura [mantieneNumero]{numero(p) == numero(pre(p))}
\asegura [mantiendoEmpleado]{atendio(p) == atendio(pre(p))}
\asegura [mantieneCantidadDeCombos]{\vert combos(p) \vert == \vert combos(pre(p)) \vert }
\asegura [losCombosSonDistintos]{distintosSyB(combos(p))}
\asegura [mantieneBebidas]{mismos(bebidasUsadas(p), bebidasUsadas(pre(p))}
\asegura [mantieneSandwiches]{mismos(sandwichesUsados(p), sandwichesUsados(pre(p))}
\asegura [usoLaMenorCantidadDeMovimientos]{\\ 
cantidadMovimientos(combos(p), combos(pre(p)) ==  cantRepeticionesDeCombos(combos(pre(p))
}
\end{problema}


\begin{problema}{nuevoP}{n: \ent, e: Empleado, cs: [Combo]}{Pedido}
\requiere [numeroPositivo]{n > 0}
\requiere [hayaPedidoAlgo]{\vert cs \vert > 0}
\asegura {numero(res) == n}
\asegura {atendio(res) == e}
\asegura {combos(res) == cs}
\end{problema}

\begin{problema}{numeroP}{p: Pedido}{\ent}
\asegura {res == numero(p)}
\end{problema}

\begin{problema}{atendioP}{p: Pedido}{Empleado}
\asegura {res == atendio(p)}
\end{problema}

\begin{problema}{combosP}{p: Pedido}{[Combo]}
\asegura {combos(p) == res)}
\end{problema}

\begin{problema}{agregarComboP}{p: Pedido, c: Combo}{}
\modifica {p}
\asegura [mantieneNumero]{numero(p) == numero(pre(p))}
\asegura [mantieneEmpleado]{atendio(p) == atendio(pre(p))}
\asegura [agregoUnCombo]{combos(p) == combos(pre(p)) ++ [c]}   
\end{problema}

\begin{problema}{anularComboP}{p: Pedido, i:\ent}{}
\requiere {0 \leq i < |combos(p)|}
\modifica {p}
\asegura [mantieneNumero]{numero(p) == numero(pre(p))}
\asegura [mantieneEmpleado]{atendio(p) == atendio(pre(p))}
\asegura [anuloElComboDelParametro]{combos(p) == (combos(pre(p))_{[0..i)} ++ combos(pre(p))_{(i..|combos(pre(p))|)} )}  
\end{problema}


\begin{problema}{cambiarBebidaComboP}{p: Pedido, b: Bebida, i:\ent} {}
\requiere {0 \leq i < \vert combos(p) \vert}  
\modifica {p}
\asegura [mantieneNumero]{numero(p) == numero(pre(p))}
\asegura [mantiendoEmpleado]{atendio(p) == atendio(pre(p))}
\asegura [mantieneCantidadDeCombos]{\vert combos(p) \vert == \vert combos(pre(p)) \vert }
\asegura [mantieneCombosIguales]{(\forall \ j \leftarrow [0..\vert combos(p) \vert), j \neq i) \ combos(p)_j == combos(pre(p))_j  } 
\asegura [mantieneSandwich]{sandwich (combos(pre(p)_i) == sandwich (combos(p)_i)}
\asegura [mantieneDificultad]{dificultad (combos(pre(p)_i) == dificultad (combos(p)_i)} 
\asegura [cambioLaBebida]{bebida (combos(p)_i) == b}

\end{problema}

\newpage

\begin{problema}{elMezcladitoP}{p: Pedido}{}
\requiere [hayaPermutacionesSuficientes]{\vert combos(p) \vert \leq \vert bebidasUsadas(p) \vert \times \vert sandwichesUsados(p) \vert }
\modifica {p}
\asegura [mantieneNumero]{numero(p) == numero(pre(p))}
\asegura [mantiendoEmpleado]{atendio(p) == atendio(pre(p))}
\asegura [mantieneCantidadDeCombos]{\vert combos(p) \vert == \vert combos(pre(p)) \vert }
\asegura [losCombosSonDistintos]{distintosSyB(combos(p))}
\asegura [mantieneBebidas]{mismos(bebidasUsadas(p), bebidasUsadas(pre(p))}
\asegura [mantieneSandwiches]{mismos(sandwichesUsados(p), sandwichesUsados(pre(p))}
\asegura [usoLaMenorCantidadDeMovimientos]{\\ 
cantidadMovimientos(combos(p), combos(pre(p)) ==  cantRepeticionesDeCombos(combos(pre(p))
}
\end{problema}

\newpage

\section{Local} 

\begin{tipo}{Local}
	\observador{stockBebidas}{l: Local, b: Bebida}{Cantidad}
	\requiere{b \in bebidasDelLocal(l)}

	\observador{stockSandwiches}{l: Local, h: Hamburguesa}{Cantidad}
	\requiere{h \in sandwichesDelLocal(l)}

	\observador{bebidasDelLocal}{l:Local}{[Bebida]} 
	\observador{sandwichesDelLocal}{l:Local}{[Hamburguesa]} 
	\observador{empleados}{l: Local}{[Empleado]}
	\observador{desempleados}{l: Local}{[Empleado]}
	\observador{energiaEmpleado}{l: Local, e: Empleado}{Energia}
	\requiere{e \in empleados(l)}

	\observador{ventas}{l: Local}{[Pedido]}
	
	\medskip

	\invariante[hayBebidasySonDistintas]{|bebidasDelLocal(l)|>0 \land distintos(bebidasDelLocal(l))}
	\invariante[haySandwichesySonDistintos]{|sandwichesDelLocal(l)|>0 \land distintos(sandwichesDelLocal(l))}
	\invariante[stockBebidasPositivo]{(\forall b \selec bebidasDelLocal(l)) stockBebidas(l,b) \geq 0 }
	\invariante[stockSandwichesPositivo]{(\forall h \selec sandwichesDelLocal(l)) stockSandwiches(l,h) \geq 0 }
	
	\invariante[empleadosDistintos]{distintos(empleados(l)++desempleados(l))}
	\invariante[energiaHasta100]{(\forall e \selec empleados(l)) energiaEnRango(energiaEmpleado(l,e))}
	\invariante[empleadosQAtendieronDelLocal]{(\forall v \selec ventas(l)) atendio(v) \in empleados(l)++desempleados(l)}
	\invariante[ventasCorrelativas]{correlativos([numero(p) | p \leftarrow ventas(l)])}
	\invariante[combosDeLocal]{(\forall x \leftarrow [ c | p \leftarrow ventas(l), c \leftarrow combos(p)]) \\ (bebida(x) \in bebidasDelLocal(l)) \wedge (sandwich(x) \in sandwichesDelLocal(l)) }

\end{tipo}


\begin{tipo}{Local}
	\observador{stockBebidas}{l: Local, b: Bebida}{Cantidad}
	\requiere{b \in bebidasDelLocal(l)}

	\observador{stockSandwiches}{l: Local, h: Hamburguesa}{Cantidad}
	\requiere{h \in sandwichesDelLocal(l)}

	\observador{bebidasDelLocal}{l:Local}{[Bebida]} 
	\observador{sandwichesDelLocal}{l:Local}{[Hamburguesa]} 
	\observador{empleados}{l: Local}{[Empleado]}
	\observador{desempleados}{l: Local}{[Empleado]}
	\observador{energiaEmpleado}{l: Local, e: Empleado}{Energia}
	\requiere{e \in empleados(l)}

	\observador{ventas}{l: Local}{[Pedido]}
	
	\medskip

	\invariante[hayBebidasySonDistintas]{|bebidasDelLocal(l)|>0 \land distintos(bebidasDelLocal(l))}
	\invariante[haySandwichesySonDistintos]{|sandwichesDelLocal(l)|>0 \land distintos(sandwichesDelLocal(l))}
	\invariante[stockBebidasPositivo]{(\forall b \selec bebidasDelLocal(l)) stockBebidas(l,b) \geq 0 }
	\invariante[stockSandwichesPositivo]{(\forall h \selec sandwichesDelLocal(l)) stockSandwiches(l,h) \geq 0 }
	
	\invariante[empleadosDistintos]{distintos(empleados(l)++desempleados(l))}
	\invariante[energiaHasta100]{(\forall e \selec empleados(l)) energiaEnRango(energiaEmpleado(l,e))}
	\invariante[empleadosQAtendieronDelLocal]{(\forall v \selec ventas(l)) atendio(v) \in empleados(l)++desempleados(l)}
	\invariante[ventasCorrelativas]{correlativos([numero(p) | p \leftarrow ventas(l)])}
	\invariante[combosDeLocal]{(\forall x \leftarrow [ c | p \leftarrow ventas(l), c \leftarrow combos(p)]) \\ (bebida(x) \in bebidasDelLocal(l)) \wedge (sandwich(x) \in sandwichesDelLocal(l)) }

\end{tipo}


\newpage

\section{Funciones Auxiliares}

\aux{distintos}{ls:[T]}{\bool}{ 
  (\forall \ i,j \selec [0..|ls|), i \neq j) \ ls_i \neq ls_j
}

\aux{energiaEnRango}{e: Energia} {\bool}{
        0 \leq e \leq 100
}

\aux{correlativos}{ls:[\ent]}{\bool}{
(((\forall \ n \leftarrow ls) (n+1) \in ls) \ \vee (esmax(n,ls))) \wedge (distintos(ls))
}

\aux{esmax}{n:\ent,ls:[\ent]}{\bool}{
(\forall \ x \leftarrow ls) \ n \geq x
}

\aux{sacarRepeticiones}{ls:[T]}{[T]}{
[ls_i \ \vert \ i \leftarrow [ 0.. \vert ls \vert) , ls_i \not \in ls(i..\vert ls \vert)] ] 
}

\aux{cuenta}{x:T, y:[T]}{\ent } {
\vert [i \ \vert \ i \selec y, i == x] \vert
}

\aux {maximo}{ls:[\ent]}{\ent}{
cab([x \ \vert \ x \leftarrow ls , (\forall \ y \leftarrow ls) \  x \geq y])
}

\subsection{Combo}
% los aux del tipo combo

\subsection{Pedido}
% los aux del tipo pedido
\aux{bebidasDelPedido}{p:Pedido}{[Bebida]}{
[bebida(combos(p)_i) \ \vert \ i \leftarrow [0..\vert combos(p) \vert) ]  
}
\aux{bebidasUsadas}{p:Pedido}{[Bebida]}{
sacarRepeticiones(bebidasDelPedido(p))  
}
\aux{sandwichesDelPedido}{p:Pedido}{[Hamburguesa]}{
[sandwich(combos(p)_i) \ \vert \ i \leftarrow [0..\vert combos(p) \vert) ]  
}
\aux{sandwichesUsados}{p:Pedido}{[Hamburguesa]}{
sacarRepeticiones(sandwichesDelPedido(p))  
}
\aux{dificultadPedido}{p:Pedido}{ \ent }{
\vert [ 1 \ \vert \ c \selec combos(p), i \selec [0.. dificultad(c)) ] \vert
}
\aux{cantidadMovimientos}{cs:[Combo], cs2:[Combo]}{\ent}{
\vert [1 \ \vert \ i \leftarrow [0..\vert cs \vert ), \neg combosIguales(cs_i ,cs2_i)] \vert 
}
\aux {combosIguales}{c1:Combo, c2:Combo}{\bool}{
(bebida(c1) == bebida(c2)) \wedge (sandwich(c1) == sandwich(c2))
}
\aux {vuelveAAparecer}{i: \ent , cs:[Combo]}{\bool}{
\vert [cs_x \ \vert \ x \selec (i.. \vert cs \vert ), combosIguales(cs_i,cs_x)] \vert > 0
}
\aux{cantRepeticionesDeCombos}{cs:[Combo]}{\ent}{
\vert [cs_i \ \vert \ i \selec [0.. \vert cs \vert ), vuelveAAparecer(i, cs)] \vert
}
\aux{distintosSyB}{cs:[Combo]}{\bool}{
(\forall \ i,j \selec [0..|cs|), i \neq j) \ \neg combosIguales(cs_i, cs_j)
}

\subsection{Local}
% los aux del tipo local
\aux{atendioEmpleado}{l:Local}{[Empleado]}{
[atendio(p) \ \vert \ p \leftarrow ventas(l), atendio(p) \in empleado(l)]
}
\aux{subventas}{l:Local}{[Empleado]}{
atendioEmpleado[0.. \vert empelados(l) \vert )
}

\aux{empleadosConMasVentas}{l:Local}{[Empleado]}{ 
[ \ e \  \vert \ e \leftarrow sacarRepeticiones(atendioEmpleado(l)),  \\ ( \forall \ e2 \leftarrow sacarRepeticiones(atendioEmpleado(l))) \ cuenta(e,atendioEmpleado(l)) \geq cuenta(e2,atendioEmpleado(l)) ]
}

\aux{cantidadDeCombosDeEmpleado}{e:Empleado, l:Local}{ \ent }{
 \\ \vert [ \ c \ \vert \ p \leftarrow ventas(l), \ c \leftarrow combos(p), \ atendio(p) == e] \vert
}

\aux{numVentaL}{l:Local}{[\ent]}{ 
[ \ numero(m) \  \vert \ m \leftarrow ventas(l)]
}

\aux{ventaN}{l:Local, n: \ent}{Pedido}{ 
cab([\ m \  \vert \ m \leftarrow ventas(l), numero(m) == n])
}

\aux{bebidasN}{l:Local, n:\ent}{[Bebida]}{
[bebida(i) \vert i \leftarrow combos(ventaN(l,n))]
}

\aux{bebidasUsadasN}{l:Local, n:\ent}{[Bebida]}{
sacarRepeticiones(bebidasN(l, n))
}

\aux{sandwichesN}{l:Local, n:\ent}{[Hamburguesa]}{
[sandwich(i) \vert i \leftarrow combos(ventaN(l,n)]
}

\aux{sandwichUsadosN}{l:Local, n:\ent}{[Hamburguesa]}{
sacarRepeticiones(sandwichesN(l, n))
}

\aux{ventasMenosN}{l:Local, n:\ent}{[Pedido]}{ 
[ p \vert p \leftarrow ventas(l), numero(p) \neq n]
}

\aux{mismosVenta}{l1:[Pedido], l2:[Pedido]}{\bool}{ 
|l1| == |l2| \wedge (\forall i \leftarrow l1) cuentaVenta(i, l1) == cuentaVenta(i, l2));
}

\aux{cuentaVenta}{x:Pedido, y:[Pedido]}{\ent } {
\vert [i \vert i \selec y, atendio(i) == atendio(x) \wedge combos(i) == combos(x)] \vert
}

\aux{ventasIguales}{v1:Pedido, v2:Pedido}{\bool}{
atendio(v1) == atendio(v2) \wedge \vert combos(v1) \vert == \vert combos(v2) \vert \wedge (\forall i \selec [0.. \vert combos(v1) \vert) \  combos(v1)_i == combos(v2)_i
}

\aux{NumeroPedidoUltimaVenta}{l:Local}{\ent}{
maximo([numero(p) \ \vert \ p \leftarrow ventas(l)])
}

\aux{posicionDeLaVenta}{e:Empleado, v:[Pedido]}{[\ent]}{
0:[(i+1) \ \vert \ i \selec [0..\vert v \vert), atendio(v_i) == e] ++ [\vert v \vert + 1]
}

\aux{distanciaEntreVentasDelEmpleado}{e:Empleado, v:[Pedido]}{[\ent]}{if \ \vert posicionDeLaVenta(e, v) \vert == 2 \ \\
then \ [\vert v \vert] \
else \ [(posicionDeLaVenta(e, v)_{i+1} - posicionDeLaVenta(e, v)_i -1) \vert i \selec [0..|posicionDeLaVenta(e, v)|-1)]}

\aux{distanciaMayorEntreVentasDelEmpleado}{e:Empleado, v:[Pedido]}{\ent}{ \\
maximo(distanciaEntreVentasDelEmpleado(e, v))
}


\aux{sacarVentaN}{l:Local,n:\ent}{[\ent]}{
[numero(i) \vert i \selec (numVentaL(l), numero(i) < n] ++ [numero(i)-1 \vert i \selec numVentaL(l), numero(i) > n]
}

\end{document} %Termin�!

