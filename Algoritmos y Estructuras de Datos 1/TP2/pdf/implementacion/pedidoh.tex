\begin{lstlisting}
#ifndef PEDIDO_H_INCLUDED
#define PEDIDO_H_INCLUDED

#include <vector>
#include "tipos.h"
#include "combo.h"

class Pedido {

	public:

		Pedido();
		Pedido(const int n, const Empleado e, const vector<Combo> cs);
 
		int                 numeroP() const;
		Empleado            atendioP() const;
		vector<Combo>       combosP() const;
		Energia             dificultadP() const;

		void                agregarComboP(const Combo c);
		void                anularComboP(int i);
		void                cambiarBebidaComboP(const Bebida b, int i);
		void                elMezcladitoP();

		void                mostrar(std::ostream& os) const;
		void                guardar(std::ostream& os) const;
		void                cargar (std::istream& is);

		bool                operator==(const Pedido& otroPedido) const;

	private:

                vector<Combo>       _combos;
                Empleado            _atendio;
                int                 _numero;


                vector< pair<Bebida, Hamburguesa> >     _posiblesParesNoUsados() ;
                bool                                    _estaRepetido(unsigned int);
                vector<Bebida>                          _bebidasUsadas();
                vector<Hamburguesa>                     _sandwichesUsados();
                vector< pair<Bebida, Hamburguesa> >     _quitarUsados(vector< pair<Bebida, Hamburguesa> >);
                bool                                    _mismosCombos(const Pedido& otroPedido) const;
                int                                     _cuenta(Combo) const;

                enum {ENCABEZADO_ARCHIVO = 'P'};
};

// Definirlo usando mostrar, para poder usar << con este tipo.
std::ostream & operator<<(std::ostream & os,const Pedido & p);

#endif // PEDIDO_H_INCLUDED

\end{lstlisting}