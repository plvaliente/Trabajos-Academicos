\subsubsection{Primer Ciclo}
\begin{lstlisting}
void Local::agregarComboAlPedidoL(const Combo c, int n){
	int i = 0 ;
	_actualizaStockBebida( c.bebidaC() , -1 ) ;
	_actualizaStockSandwich( c.sandwichC() , -1 ) ;
	while( i < _ventas.size() && _ventas.at(i).numeroP() != n ){
		i++ ;
	}
\end{lstlisting}

\hspace{15pt}\textbf{Pc, Qc, I, B, cota y fv.} \\
Llamo $vN = |ventas(l)| $;\\
Vale $vN \geq 1$ (por requiere); \\
Llamo $nr(i) = numero(ventas(l)_i) $; \\
Ademas por requiere tambien se que n tiene que ser un numero de pedido valido ;\\
Pc: $i == 0$; \\
Qc: $nr(i) == n \vee vN == i$ ; \\
B: $i < vN \wedge nr(i) \neq n$ ; \\
$\neg$B: $i \geq vN \vee nr(i) == n$ ; \\
Fv: $vN - i $; \\
cota: 0 ; \\
I: $0 \leq i \leq vN $ ; \\
	
\hspace{15pt}\textbf{Pc $\wedge$ B $\Rightarrow$ I:} \\
Pc $\wedge$ B: $i == 0 \wedge i < vN \wedge nr(i) \neq n$ ; \\
Sabiendo que $i == 0$ verifico que se cumple el invariante: \\
Reemplazo i = 0 en I:$ 0 \leq 0 \leq vN $ (Verdadero porque $vN \geq 1$ );  \\

\hspace{15pt}\textbf{I $\wedge \neg $ B $\Rightarrow$ Qc:} \\
I $\wedge \neg$B: $0 \leq i \leq vN \wedge i \geq vN \vee nr(i) == n$ ; \\
Implica: $vN == i \vee nr(i) == n  $; \\
Ya que: $0 \leq i \leq vN \wedge i \geq vN \Rightarrow vN == i $; \\
Y $0 \leq i \leq vN \wedge nr(i) == n$. En particular implica $nr(i) == n$ ; \\

\hspace{15pt}\textbf{I $\wedge$ fv $\leq$ Cota $\Rightarrow \neg$ B:} \\
I $\wedge$ Fv $\leq$ C: $0 \leq i \leq vN \wedge vN - i \leq 0$ ; \\
Implica: $ i \leq vN \wedge vN \leq i$ ; \\
Implica: $i == vN$; \\
Implica: $i \geq vN \vee nr(i) == n$; \\
Equivale a $\neg B$; \\

\newpage %Salto de página

