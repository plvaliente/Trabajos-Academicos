\subsubsection{Primer ciclo}
\begin{lstlisting}
vector<Empleado> Local::candidatosAEmpleadosDelMesL() const{
        return _candidatosAEmpleadosDelMes();
}

vector <Empleado> Local::_candidatosAEmpleadosDelMes() const{
	return _empleadosConMasCombos(_empleadosConMasVentas(empleadosL()));
}

vector < Empleado > Local::_empleadosConMasCombos(const vector <Empleado> empleados) const{
	int i = 0;
	vector < Empleado > emp = vector < Empleado >();

	//**Busco el maximo de Combos vendidos por todos los empleados.
	int maxCombos = 0;
	while (i < empleados.size()) {
		if ( _combosVendidosPorElEmpleado(empleados.at(i)).size() > maxCombos) {
			maxCombos = _combosVendidosPorElEmpleado(empleados.at(i)).size();
		}
		i++;
	}
\end{lstlisting} 


\hspace{15pt}\textbf{Pc, Qc, I, B, cota y fv.} \\
Llamo $sizeE = |empleados(l)| $; \\
Vale $sizeE > 0$ por invariante de Local. ;\\
Implica  $|empleados(l)| > 0$ ;\\
Llamo $sizeCombosVendidosE(i) = | combosVendidosPorElEmpleado(l, empleados(l)_i) | $; \\
Llamo $combosVendidosPorElEmpleado = concat[combos(p)|p \leftarrow ventas(l), atendio(p) == e]$ ;\\

Pc: $i == 0 \wedge maxCombos == 0 \wedge emp == [] \wedge sizeE > 0$ ;\\
Qc: $(\forall j \leftarrow [0..sizeE)) |combosVendidosPorElEmpleado(l, empleados(l)[j])| \leq maxCombos$ ;\\
B: $i < sizeE$ ;\\
I: $0 \leq i \leq sizeE \wedge (\forall j \leftarrow [0..i)) |combosVendidosPorElEmpleado(l, empleados(l)[j])| \leq maxCombos$ ;\\ 	
fv: $sizeE - i$ ;\\
cota: 0 ;\\

\hspace{15pt}\textbf{Pc $\wedge$ B $\Rightarrow$ I:} \\
Por Pc $\wedge$ B : $i == 0 \wedge maxCombos == 0 \wedge emp == [] \wedge sizeE > 0 \wedge i < sizeE $ ;\\
Implica $ i == 0 \wedge i < sizeE \wedge sizeE > 0 \Leftrightarrow 0 == i < sizeE $ ; \\
Implica $ 0 \leq i \leq sizeE $ ;\\

Luego usando que $i == 0$ y por Pc sabemos que $maxCombos == 0$ ;\\
Y $(\forall j \leftarrow [0..0)) |combosVendidosPorElEmpleado(l, empleados(l)_j)| == 0 $ ;\\
Implica que $(\forall j \leftarrow [0..i)) |combosVendidosPorElEmpleado(l, empleados(l)_j)| == maxCombos $ ;\\
Implica que $(\forall j \leftarrow [0..i)) |combosVendidosPorElEmpleado(l, empleados(l)_j)| \leq maxCombos $ ;\\

Luego sabemos que $0 \leq i \leq sizeE \wedge (\forall j \leftarrow [0..i)) |combosVendidosPorElEmpleado(l, empleados(l)_j)| \leq maxCombos$ ;\\
Equivale a I ;\\

\hspace{15pt}\textbf{I $\wedge \neg $ B $\Rightarrow$ Qc:} \\
Usando que $i \leq sizeE$ por el Invariante y $i >= sizeE$ por la negacion de la guarda ;\\
Entonces $i \leq sizeE \wedge  i \geq sizeE \Rightarrow i == sizeE $ ;\\
Reemplazo en I: $(\forall j \leftarrow [0..sizeE)) |combosVendidosPorElEmpleado(l, empleados(l)_j)| \leq maxCombos $ ;\\
Lo cual es equivalente a Qc. ;\\

\hspace{15pt}\textbf{I $\wedge$ fv $\leq$ Cota $\Rightarrow \neg$ B:} \\
I $\wedge$ fv $\leq$: $0 \leq i \leq sizeE \wedge sizeE - i \leq cota$ ;\\
Implica $ 0 \leq i \leq sizeE \wedge sizeE \leq i.$ ;\\
Implica $i == sizeE.$ ;\\
Implica $i \geq sizeE.$ Lo cual es equivalente a $ \neg$ B. ;\\
\\
\\

