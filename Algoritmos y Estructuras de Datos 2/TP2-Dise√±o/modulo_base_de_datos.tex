\section{M\'odulo Base de Datos}

\Encabezado{Notas Preliminares}
  En todos los casos, al indicar las complejidades de los algoritmos, las variables que se utilizan corresponden a:
  \vspace{-0.5em}\begin{itemize}
    \item $n$: N\'umero de registros en la tabla pasada por par\'ametro.
    \item $m$: N\'umero de registros en la tabla pasada por par\'ametro.
    \item $L$: M\'axima longitud de un valor STRING de un registro en la tabla pasada por par\'ametro.
    \item $T$: Cantidad de tablas en la base de datos
    \item $R$: Cantidad de elementos en la lista de modificaciones de un join.
  \end{itemize}


\begin{Interfaz}
	\seExplicaCon{BaseDeDatos}
    
    \generos{\tipo{bd}} 
    
 		\servUsados{tbl, nombreTabla, itTablas, reg, campo, bool, conj, tupla, diccStr, datosTabla}
    
    \Encabezado{Operaciones de Base De Datos}
    
    \InterfazFuncion{nuevaDB}{}{bd}%
    [true]%pre
    {res $\igobs$ nuevaDB()}%pos
    [$O(1)$]%complejidad
    [Genera una nueva Base De Datos. ]%descripcion
    []%aliasing
    
    \InterfazFuncion{agregarTabla}{\In{t}{tbl}, \Inout{b}{bd}}{}%
    [b $\igobs$ $b_{0}$ $\land$ $\emptyset$?(registros(t))]%pre
    {b $\igobs$ agregarTabla(t, $b_{0}$)}%pos
    [$O(1)$]%complejidad
    [Agrega una nueva tabla sin registros a la base de datos]%descripcion
    []%aliasing
    
    \InterfazFuncion{insertarEntrada}{\In{reg}{reg}, \In{t}{nombreTabla}, \Inout{b}{bd}}{}%
    [b $\igobs$ $b_{0}$ $\land$ t $\in$ tablas(b) $\yluego$ puedoInsertar?(reg, t)]%pre
    {b $\igobs$ insertarEntrada(reg, t, $b_{0}$)}%pos
    [$O(T*L + lg(n))$]%complejidad
    [Inserta un nuevo registro en la tabla cuyo nombre es pasado por par\'ametro]%descripcion
    []%aliasing
    
    \InterfazFuncion{borrar}{\In{cr}{reg}, \In{t}{nombreTabla}, \Inout{b}{bd}}{}%
    [b $\igobs$ $b_{0}$ $\land$ t $\in$ tablas(b) $\land$ \#(campos(cr))=1]%pre
    {b $\igobs$ borrar(reg, t, $b_{0}$)}%pos
    [$O(T*L + n *(n + L)$]%complejidad
    [Borra todos los registros de la tabla, cuyo nombre es pasado por par\'ametro, cuyos datos del campo del registro pasado por par\'ametro coincidan con el dato del mismo registro]%descripcion
	[]%aliasing
    
    \InterfazFuncion{generarVistaJoin}{\In{t1}{nombreTabla}, \In{t2}{nombreTabla}, \In{c}{campo} \Inout{b}{bd}}{}%
    [b $\igobs$ $b_{0} \land$ t1 $\neq$ t2 $\land$ $\lbrace$t1, t2$\rbrace$ $\subseteq$ tablas(b) $\yluego$ c $\in$ claves(dameTabla(t1, b)) $\land$ c $\in$ claves(dameTabla(t2,b)) $\land$ $\lnot$hayJoin?(t1,t2,b)]%pre
    {b $\igobs$ generarVistaJoin(t1, t2, c, $b_{0}$)}%pos
    [$O((n+m)*(L + log(n + m)))$]%complejidad
    [Genera un join entre las tablas cuyos nombres son pasados por par\'ametro]%descripcion
    []%aliasing 
  
    \InterfazFuncion{borrarJoin}{\In{t1}{nombreTabla} \In{t2}{nombreTabla} \Inout{b}{bd}}{}%
    [b $\igobs$ $b_{0} \land$ hayJoin?(t1,t2,b)]%pre
    {b $\igobs$ borrarJoin(t1,t2,$b_{0}$}%pos
    [$O(1)$]%complejidad
    [Borra el join entre las tablas cuyos nombres son pasados por par\'ametro]%descripcion
    []%aliasing 
  
    \InterfazFuncion{tablas}{\In{b}{bd}}{itTablas}%
    [true]%pre
    {esPermutacion?(SecuSuby(res), tablas(b)) }%pos
    [$O(1)$]%complejidad
    [Devuelve un iterador de Tablas con siguiente ]%descripcion
    []%aliasing     

    \InterfazFuncion{dameTabla}{\In{t}{nombreTabla} \In{b}{bd}}{tbl}%
    [t $\in$ tablas(b)]%pre
    {res $\igobs$ dameTabla(t,b) }%pos
    [$O(1)$]%complejidad
    [Devuelve la tabla de la base de datos cuyo nombre es $t$]%descripcion
    [tbl se devuelve por referencia]%aliasing     

    \InterfazFuncion{hayJoin?}{\In{t1}{nombreTabla} \In{t2}{nombreTabla} \In{b}{bd}}{bool}%
    [true]%pre
    {res $\igobs$ hayJoin(t1,t2,b) }%pos
    [$O(1)$]%complejidad
    [Devuelve si hay join entre las dos tablas cuyos nombres son los pasados como parametro]%descripcion
    []%aliasing   

    \InterfazFuncion{campoJoin}{\In{t1}{nombreTabla} \In{t2}{nombreTabla} \In{b}{bd}}{campo}%
    [hayJoin?(t1,t2,b)]%pre
    {res $\igobs$ campoJoin(t1,t2,b)}%pos
    [$O(1)$]%complejidad
    [Devuelve el campo por el cual hay join en las tablas cuyos nombres son pasados como parametro]%descripcion
    []%aliasing   

    \InterfazFuncion{registros}{\In{t}{nombreTabla} \In{b}{bd}}{conj(reg)}%
    [t $\in$ tablas(b)]%pre
    {res $\igobs$ registros(t,b)}%pos
    [$O(n)$]%complejidad
    [Devuelve el conjunto de registros de la tabla cuyo nombre es pasado como parametro]%descripcion
    [El conjunto de registros de pasa por referencia no modificable]%aliasing   

    \InterfazFuncion{vistaJoin}{\In{t1}{nombreTabla} \In{t2}{nombreTabla} \In{b}{bd}}{conj(reg)}%
    [hayJoin?(t1,t2,b)]%pre
    {res $\igobs$ vistaJoin(t1,t2,b)}%pos
    [$O(R*(L+log(n+m))$]%complejidad
    [Actualiza y devuelve el join entre las dos tablas cuyos nombres son pasados como parametro.]%descripcion
    [El conjunto de registros de devuelve por referencia no modificable]%aliasing   

    \InterfazFuncion{cantidadDeAccesos}{\In{t}{nombreTabla} \In{b}{bd}}{conj(reg)}%
    [t $\in$ tablas(b)]%pre
    {res $\igobs$ cantidadDeAccesos(t,b)}%pos
    [$O(1)$]%complejidad
    [Devuelve la cantidad de accesos de la tabla cuyo nombre es pasado como parametro]%descripcion
    []%aliasing   
    
    \InterfazFuncion{tablaMaxima}{\In{b}{bd}}{nombreTabla}%
    [tablas(b) $\neq \emptyset$]%pre
    {res $\igobs$ tablaMaxima(t1,t2,b)}%pos
    [$O(1)$]%complejidad
    [Devuelve el nombre de la tabla m\'axima de la base de datos]%descripcion
    []%aliasing   

    \InterfazFuncion{encontrarMaximo}{\In{t}{nombreTabla} \In{ct}{conj(nombreTabla)} \In{b}{bd}}{nombreTabla}%
    [\{t\} $\cup$ ct $\subseteq$ tablas(b)]%pre
    {res $\igobs$ encontrarMaximo(t,ct,b)}%pos
    [$O(T)$]%complejidad
    [Devuelve el nombre de la tabla m\'axima de entre una tabla y conjunto de tablas cuyos nombres son pasados por parametro]%descripcion
    []%aliasing   
    
    \InterfazFuncion{buscar}{\In{criterio}{reg} \In{t}{nombreTabla} \In{b}{bd}}{conj(reg)}%
    [t $\in$ tablas(b)]%pre
    {res $\igobs$ buscar(criterio,t,b)}%pos
    [$O(n*L)$ en peor caso, si alguno de los campos de criterio es un campo indexado y clave en la tabla de nombre t la complejidad es $O(log(n) + L)$]%complejidad
    [Devuelve un conjunto con los registros de la tabla con nombre $t$ que coincidan en campo y dato para cada campo de $r$ utilizando los campos indexados de t (si existen y si son campos de criterio) para realizar la busqueda en menor tiempo]%descripcion
    [El conjunto de registros se pasa por referencia no modificable]%aliasing   
\end{Interfaz}    

\begin{Representacion}
	\begin{Estructura}{bd}[estrBD]
    	\begin{Tupla}[estrBD]
		\tupItem{NombresTablas}{Conj<nombreTabla>)}
        \tupItem{Tablas}{DiccStr<nombreTabla, datosTabla>}
        \tupItem{tablaMaxima}{nombreTabla}
        \end{Tupla}
	\end{Estructura}
    
    
\textbf{Invariante de Representaci\'on en castellano}

	\begin{enumerate}
		\item TablaMaxima es el nombre de una tabla en las tablas de la base de datos
        \item Para cada tabla en Tablas, la cantidad de accesos es menor o igual a la cantidad de accesos de la tabla cuyo nombre es el de tablaMaxima
        \item Los nombres en Tablas son los mismos que el conjunto NombresTablas, y para cada nombre en Tablas, la Tabla en su significado tiene el mismo nombre.
        \item En cada Tabla, las tablas en Joins pertenecen a las tablas de la base de datos.
		\item En cada Tabla, el nombre de la tabla, no aparece en sus Joins en datosTabla.
        \item En cada Tabla, las claves de todas las tablas con las que tiene Join, estan incluidas en las claves de Tabla.
        \item En cada Tabla, y cada Join, el campo del Join es un campo de las dos tablas.
        \item En cada Tabla, los registros de Joins de datosTabla, son el resultado de combinar las dos claves por el campo del Join.
        \item En cada Tabla, si el campo del Join es de tipo String, entonces en ConjJoin en Joins de datosTabla, cada DatoString de IndiceString es equivalente a obtener el CampoIndiceString de Registros, y su significado es una tupla cuyo primer elemento es un iterador que apunta al registro que tiene ese dato en el campo CampoIndiceString, y el segundo es NULL.
        \item En cada Tabla, si el campo del Join es de tipo Nat, entonces en ConjJoin en Joins de datosTabla, cada DatoNat de IndiceNat es equivalente a obtener el CampoIndiceNat de Registros, y su significado es una tupla cuyo primer elemento es un iterador que apunta al registro que tiene ese dato en el campo CampoIndiceString, y el segundo es NULL.
	\end{enumerate}
    
 \textbf{Invariante de Representaci\'on}
     \Rep[bd][e]{
     \begin{enumerate}
		\item ($\exists$ n $\in$ e.NombresTablas) n = e.tablaMaxima $\yluego$ 
		\item ($\forall$ n $\in$ e.NombresTablas) cantidadDeAccesos(Obtener(n , e.Tablas).Tabla) $\leq$ cantidadDeAccesos(Obtener( e.TablaMaxima, e.Tablas).Tabla)  $\land$ 
		\item e.NombresTablas = Claves(e.Tablas) $\yluego$ ($\forall$ n $\in$ e.NombresTablas)) n $=$ nombre(Obtener(n, e.Tablas).Tabla) $\land$ 
		\item ($\forall$ n1 $\in$ e.NombresTablas) ($\forall$ n2 $\in$ Claves(Obtener(n1, e.Tablas).Joins)) n2 $\in$ e.NombresTablas $\yluego$ 
        \item ($\forall$ n1 $\in$ e.NombresTablas) ($\forall$ n2 $\in$ Claves(Obtener(n1,	 e.Tablas).Joins)) n1 $\neq$ n2 $\land$ 
        \item ($\forall$ n1 $\in$ e.NombresTablas) ($\forall$ n2 $\in$ Claves(Obtener(n1, e.Tablas).Joins)) claves(Obtener(n2, e.Tablas).Tabla) $\in$ claves(Obtener(n1, e.Tablas).Tabla) $\land$ 
        \item ($\forall$ n1 $\in$ e.NombresTablas) ($\forall$ n2 $\in$ Claves(Obtener(n1, e.Tablas).Joins)) Obtener(n2, Obtener(n1, e.Tablas).Joins).Campo $\in$ campos(Obtener(n1, e.Tablas).Tabla) $\land$ Obtener(n2, Obtener(n1, e.Tablas).Joins).Campo $\in$ campos(Obtener(n2, e.Tablas).Tabla) $\land$ 
		\item ($\forall$ n1 $\in$ e.NombresTablas) ($\forall$ n2 $\in$ Claves(Obtener(n1, e.Tablas).Joins)) Obtener(n2, Obtener(n1, e.Tablas).Joins).ConjJoin.Registros $=$ CombinarRegistros(Obtener(n2, Obtener(n1, e.Tablas).Joins).ConjJoin.Campo , registros(Obtener(n1, e.Tablas).Tabla) , registros(Obtener(n1, e.Tablas).Tabla) ) $\land$
		\item ($\forall$ n1 $\in$ e.NombresTablas) ($\forall$ n2 $\in$ Claves(Obtener(n1, e.Tablas).Joins)) String?(Obtener(n2, Obtener(n1, e.Tablas).Joins).Campo) $\impluego$ 
        ($\forall$ s$\in$ Claves(Obtener(n2, Obtener(n1, e.Tablas).Joins).ConjJoin.IndiceString)) Obtener(Obtener(n2, Obtener(n1, e.Tablas).Join).Campo , Siguiente(Obtener(s , Obtener(n2, Obtener(n1, e.Tablas).Join).ConjJoin.IndiceString))) $=$ s
		\item ($\forall$ n1 $\in$ e.NombresTablas) ($\forall$ n2 $\in$ Claves(Obtener(n1, e.Tablas).Joins)) Nat?(Obtener(n2, Obtener(n1, e.Tablas).Joins).Campo) $\impluego$ 
        ($\forall$ s$\in$ Claves(Obtener(n2, Obtener(n1, e.Tablas).Joins).ConjJoin.IndiceNat)) Obtener(Obtener(n2, Obtener(n1, e.Tablas).Joins).Campo , Siguiente(Obtener(s ,Obtener(n2, Obtener(n1, e.Tablas).Joins).ConjJoin.IndiceNat))) $=$ s.
      \end{enumerate}  
      }


\textbf{Funcion de Abstracci\'on}

\Abs[estrBD]{BaseDeDatos}[e]{bd}{
\begin{enumerate}
	\item tablas(bd) $=$ e.NombresTablas $\yluego$ 
    \item ($\forall$ s $\in$ tablas(bd)) s $\in$ tablas(bd) $\impluego$ dameTabla(s, bd) $=$ Obtener(s, e.Tablas).Tabla $\yluego$ 
    \item ($\forall$ t1 $\in$ tablas(bd)) ($\forall$ t2 $\in$ tablas(b)) hayJoin?(t1,t2,bd) $=$ Def?(t2, Obtener(t1, e.Tablas).Joins) $\land$ 
    \item ($\forall$ t1 $\in$ tablas(bd)) ($\forall$ t2 $\in$ tablas(b)) hayJoin?(t1,t2,bd) $\impluego$ campoJoin(t1,t2,bd) $=$ Obtener(t2, Obtener(t1, e.Tablas).Join).Campo   
    \end{enumerate}
}
    
\end{Representacion}  

\begin{Algoritmos}

\begin{algoritmo}{iNuevaBD}{}{estrBD}
	$res.NombreTablas \gets Vacio()$ \com*{$O(1)$}
  	$res.Tablas \gets Vacio()$ \com*{$O(1)$}
    $res.tablaMaxima \gets "" $ \com*{$O(1)$}
    
\end{algoritmo}
\datosAlgoritmo{} % Descripción
{} % Pre
{} % Post
{$O(1)$} % Complejidad
{$O(1) + O(1) + O(1) = O(1) $} % Justificación

\begin{algoritmo}{iAgregarTabla}{\Inout{b}{estrBD}, \In{t}{tbl}}{bd}
	\If(\com*[f]{$O(|t.Nombre|)$}){$\neg$ Definido?(b.Tablas, t.Nombre)}{
    	$AgregarRapido(b.NombreTablas, t.Nombre)$ \com*{$O(copy(t.Nombre))$}
        $datos \gets < t , Vacio() > $ \com*{$O(1) + O(1)$}
        $Definir(b.Tablas, t.Nombre, datos) $ \com*{$O(copy(t.Nombre))$}
        \If(\com*[f]{$O(equals(b.tablaMaxima, t.Nombre)$}){b.tablaMaxima = $""$}{
            $b.tablaMaxima \gets t.Nombre$ \com*{$O(copy(t.Nombre))$}
        }
    }	 
\end{algoritmo}
\datosAlgoritmo{} % Descripción
{} % Pre
{} % Post
{$O(copy(t.Nombre))$} % Complejidad
{El algoritmo tiene llamada a funciones con costo $O(copy(t.Nombre))$, $O(|t.Nombre|)$ , $O(equals(b.tablaMaxima, t.Nombre)$ , $O(1)$ y $O(1)$. Aplicando \'algebra de \'ordenes: \\ $O(|t.Nombre|)$ + $O(copy(t.Nombre))$ + $O(1)$ + $O(1)$ + $O(copy(t.Nombre))$ + $O(equals(b.tablaMaxima, t.Nombre)$ + $O(copy(t.Nombre))$ = $O(copy(t.Nombre))$

Como el nombre de las tablas esta acotado, el costo de copiado es $O(1)$} % Justificación

\begin{algoritmo}{iInsertarEntrada}{\In{reg}{reg}, \In{t}{string}, \Inout{b}{estrBD}}{}
	$tablaMaxima \gets DameTabla(b.tablaMaxima, b)$ \com*{$O(|b.tablaMaxima|)$}
   	$tabla \gets DameTabla(t, b)$ \com*{$O(|t|)$}
   	$AgregarRegistro(reg, tabla)$ \com*{$O(L) + O(in)$}
    \If(\com*[f]{$O(1)$}){$CantidadAccesos(tabla) > CantidadAccesos(tablaMaxima)$}{		    	    			$b.tablaMaxima \gets t$ \com*{$O(1)$}
    }
	$ActualizarJoin(true, reg,t,b)$ \com*{$O(T*L)$}
\end{algoritmo}
\datosAlgoritmo{} % Descripción
{} % Pre
{} % Post
{$O(T*L + lg(n))$} % Complejidad
{Por algebra de ordenes
$O(|b.tablaMaxima|) + O(|t|) + O(L) + O(in) + O(1) + O(T*L) =$

$O(1) + O(1) + O(L) + O(in) + O(1) + O(T*L) =$

$O(T*L) + O(in)$ Ya que los nombres de las tablas son acotados.

Donde in es $O(|L|)$, si no hay indices o si solo hay indice string, y $O(lg(n))$ si hay indice NAT} % Justificación
		
\begin{algoritmo}{iBorrar}{\In{cr}{reg}, \In{t}{string}, \Inout{b}{estrBD}}{}
	$tablaMaxima \gets DameTabla(b.tablaMaxima, b)$ \com*{$O(|b.tablaMaxima|)$}
   	$tabla \gets DameTabla(t, b)$ \com*{$O(|t|)$}
   	$BorrarRegistro(reg, tabla)$ \com*{$O(L) + O(in)$}
    \If(\com*[f]{$O(1)$}){$CantidadAccesos(tabla) > CantidadAccesos(tablaMaxima)$}{		    	    			$b.tablaMaxima \gets t$ \com*{$O(1)$}
    }
	$ActualizarJoin(false, cr,t,b)$ \com*{$O(T*L)$}
    
\end{algoritmo}
\datosAlgoritmo{} % Descripción
{} % Pre
{} % Post
{$O(T*L + n*(n+L)) $} % Complejidad
{$O(T) + O(L) + O(in) + O(T*L)$ = $O(T*L + in)$ Por algebra de ordenes.

Donde in es $O(n *(n + L))$ en peor caso, y $O(log(n) + L)$ si el criterio de borrado es indice sobre campo tipo nat. } % Justificación 

\begin{algoritmo}{iGenerarVistaJoin}{\In{t1}{string}, \In{t2}{string}, \In{c}{campo}, \Inout{b}{estrBD}}{}
    $datosT \gets Obtener(b.Tablas, t1)$ \com*{$O(|t1.Nombre|)$}
   	$contenedor \gets <Vacio(), Vacio(), Vacio() > $ \com*{$O(3)$}
	$join \gets <c, Vacio(), contenedor > $ \com*{$O(3)$}
	$Definir(datosT.Joins, t2, join) $ \com*{$O(copy(t1.Nombre))$}
    $regsT1 \gets registros(t1, bd)$ \com*{$O(1)$}
    $regsComb \gets combinarRegistrosRap(c, regsT1, dameTabla(t2))$ \com*{$O(A)$}
    $itRegsComb \gets CrearIt(regsComb) $ \com*{$O(1)$}
    \While(\com*[f]{$O((n+m)*(3*L+(3*log(n+m)))$}){($HaySiguiente(itRegsComb)$)}{
    
		$regComb \gets Siguiente(itRegsComb) $ \com*{$O(1)$} 
        $itRegComb \gets AgregarRapido(contenedor.Registros, regComb) $ \com*{$O(1)$} 
    	\eIf(\com*[f]{$O(|campos(regComb)|)$}){esNat?(Significado(regComb,c))}{
        	$DatoNat \gets ValorN(Significado(regComb, c))$ \com*{$O(|campos(regComb)|)$} 
            \eIf(\com*[f]{$O(log(n+m))$}){Definido?(contenedor.IndiceNat, DatoNat)}{
            	$ConjIts \gets Significado(contenedor.IndiceNat, DatoNat) $ \com*{$O(log(n+m))$}
                $AgregarRapido(ConjIts, $<$itRegComb,NULL$>$) $ \com*{$O(1)$}
            }{
            	$ConjIts \gets Vacio() $ \com*{$O(1)$}
				$AgregarRapido(ConjIts, $<$itRegComb,NULL$>$) $ \com*{$O(1)$}
				$DefinirRapido(contenedor.IndiceNat, DatoNat, ConjIts) $ \com*{$O(log(n+m))$}             	
            }
        }
        {
        	$DatoString \gets ValorS(Significado(regComb, c))$ \com*{$O(|campos(regComb)|)$} 
            \eIf(\com*[f]{$O(L)$}){Definido?(contenedor.IndiceString, DatoString)}{
            	$ConjIts \gets Significado(contenedor.IndiceString, DatoString) $ \com*{$O(L)$}
                $AgregarRapido(ConjIts, $<$itRegComb,NULL$>$) $ \com*{$O(1)$}
            }{
            	$ConjIts \gets Vacio() $ \com*{$O(1)$}
				$AgregarRapido(ConjIts, $<$itRegComb,NULL$>$) $ \com*{$O(1)$}
				$DefinirRapido(contenedor.IndiceString, DatoString, ConjIts) $ \com*{$O(L)$}             	
            }
		}
        $Avanzar(itRegsComb)$ \com*{$O(1)$}
    }
\end{algoritmo}

\datosAlgoritmo{} % Descripción
{} % Pre
{} % Post
{$O((n+m)*(L+log(n+m))$} % Complejidad
{\begin{enumerate}
\item $n$ es la cantidad de registros de la tabla 1
\item $m$ es la cantidad de registros de la tabla 2
\item $L$ es la maxima longitud de algun campo string de algun registro que este en alguno de los dos conjuntos de registros de las tablas.
\item $A$ es la complejidad de combinarRegistrosRap. Y como el campo es Clave e Indice. Entonces esta complejidad es $O(n*(log(m) + L))$
\item El nombre de las tablas es acotado por lo tanto $O(t1.Nombre) \in O(1)$
\item La cantidad de campos de un registro es acotado por lo tanto $O(|campos(regComb)|) \in O(1)$
\item Por los items anteriores y las justificaciones del codigo se deduce que la complejidad del while es $O((n+m)*(3*L+(3*log(n+m)))$
\item Por lo tanto la complejidad del algoritmo es $O((n+m)*(3*L+(3*log(n+m)))) + O(n*(log m + L))$
\item Aplicando algebra de ordenes se obtiene $O(n*L + m*L + n*log(n+m) + m*log(n+m) + n*log(m) + n*L)$
\item Siguiendo $O(n*L +n*L + m*L + n*log(n+m) + m*log(n+m) + n*log(m))$
\item Siguiendo $O(2*n*L + m*L + n*log(n+m) + m*log(n+m) + n*log(m))$
\item Siguiendo $O(2*n*L + m*L + n*log(n+m) + m*log(n+m) + n*log(m+n))$
\item Siguiendo $O(2*n*L + m*L + 2*n*log(n+m) + m*log(n+m))$
\item Siguiendo $O(n*L + m*L + n*log(n+m) + m*log(n+m))$
\item Finalmente queda que la complejidad es $O((n+m)*(L + log(n+m)))$
\end{enumerate}} % Justificación

\begin{algoritmo}{iborrarJoin}{\In{t1}{nombreTabla} \In{t2}{nombreTabla} \Inout{b}{estrBD}}{}

    $datosT \gets Obtener(b.Tablas, t1)$ \com*{$O(|t1|)$}
    $borrar(datosT.Joins, t2)$\com*{$O(1)$}
    
\end{algoritmo}

\datosAlgoritmo{} % Descripción
{} % Pre
{} % Post
{$O(1)$} % Complejidad
{$O(|t1|) + O(1) = O(1) $porque los nombres de las tablas estan acotados.} % Justificación

\begin{algoritmo}{itablas}{\In{b}{estrBD}}{itTablas}
    $itTbl \gets CrearIt(b.NombresTablas)$ \com*{$O(1)$}
	$res \gets itTbl$ \com*{$O(1)$}  	    
\end{algoritmo}

\datosAlgoritmo{} % Descripción
{} % Pre
{} % Post
{$O(1)$} % Complejidad
{$O(1) + O(1) = O(1)$} % Justificación

\begin{algoritmo}{idameTabla}{\In{t}{nombreTabla} \In{b}{estrBD}}{tbl}

	$datosT \gets Obtener(b.Tablas,t)$ \com*{$O(|t|)$}
	$res \gets datosT.Tabla$ \com*{$O(1)$}
    
\end{algoritmo}

\datosAlgoritmo{} % Descripción
{} % Pre
{} % Post
{$O(1)$} % Complejidad
{$O(|t|) + O(1) = O(1)$ porque los nombres de las tablas estan acotados} % Justificación

\begin{algoritmo}{ihayJoin?}{\In{t1}{nombreTabla} \In{t2}{nombreTabla} \In{b}{estrBD}}{bool}

	$datosT \gets Obtener(b.Tablas, t1)$ \com*{$O(|t1|)$}
	$res \gets Def?(datosT.Joins, t2)$ \com*{$O(|t2|)$}

\end{algoritmo}

\datosAlgoritmo{} % Descripción
{} % Pre
{} % Post
{$O(1)$} % Complejidad
{$O(|t1|) + O(|t2|) = O(1)$ porque los nombres de las tablas estan acotados} % Justificación

\begin{algoritmo}{icampoJoin}{\In{t1}{nombreTabla} \In{t2}{nombreTabla} \In{b}{estrBD}}{campo}%

	$datosT \gets Obtener(b.Tablas,t1)$ \com*{$O(|t1|)$}
    $res \gets Obtener(datosT.Joins, t2).Campo$ \com*{$O(|t2|)$}
    
\end{algoritmo}

\datosAlgoritmo{} % Descripción
{} % Pre
{} % Post
{$O(1)$} % Complejidad
{$O(|t1|) + O(|t2|) = O(1)$ porque los nombres de las tablas estan acotados} % Justificación

\begin{algoritmo}{iregistros}{\In{t}{nombreTabla} \In{b}{estrBD}}{conj(reg)}%

	$datosT \gets Obtener(b.Tablas,t1)$ \com*{$O(|t|)$}
    $res \gets registros(datosT.Tabla)$ \com*{$O(registros(t))$}
    
\end{algoritmo}

\datosAlgoritmo{} % Descripción
{} % Pre
{} % Post
{$O(registros(t))$} % Complejidad
{$O(|t|) + O(registros(t)) = O(registros(t))$ porque los nombres de las tablas estan acotados} % Justificación


\begin{algoritmo}{iVistaJoin}{\In{t1}{string}, \In{t2}{string}, \In{b}{estrBD}}{conj(reg)}
    $datosT \gets Obtener(b.Tablas, t1)$ \com*{$O(|t1|)$}
    $join \gets Obtener(datosT.Joins, t2) $ \com*{$O(|t2|)$}
    $itMod \gets CrearIt(join.Modifiaciones) $ \com*{$O(1)$}
	\While(\com*[f]{$O(R*(L+log(n*m)))$}){HaySiguiente(itMod)}{
    	$ mod \gets Siguiente(itMod) $ \com*{$O(1)$}
        $conjReg2 \gets buscar(mod.Reg, t2, b)$ \com*{$O(buscar(r,t,b))$}
        $conjReg1 \gets Vacio() $ \com*{$O(1)$}
        $AgregarRapido(conjReg1, mod.Reg) $ \com*{$O(1)$}
        $regsComb \gets combinarRegistros(join.Campo, conjReg1, conjReg2)$ \com*{$O(A)$}
        $itRegsComb \gets CrearIt(regsComb) $ \com*{$O(1)$}
        \While(\com*[f]{$O(\#regsComb)$}){($HaySiguiente(itRegsComb)$)}{
            $regComb \gets Siguiente(itRegsComb) $ \com*{$O(1)$} 
        	\eIf(\com*[f]{$O(|campos(regComb)|)$}){Nat?(Significado(regComb,join.Campo))}{
				$AuxActContenedor(join.ConjJoin, regComb, mod, join.Campo, true)$ \com*{$O(log(n+m) + L)$}
        	}
        	{
				$AuxActContenedor(join.ConjJoin, regComb, mod, join.Campo, false)$ \com*{$O(log(n+m) + L)$}
			}
		}
    }
    $ res \gets join.Reg $ \com*{$O(1)$}
\end{algoritmo}
\datosAlgoritmo{} % Descripción
{} % Pre
{} % Post
{$O(R*(L+log(n+m))$} % Complejidad
{\begin{enumerate}
\item $R$ es la cantidad de elementos en la lista de modificaciones del Join
\item $n$ es la cantidad de elementos de la tabla 1
\item $m$ es la cantidad de elementos de la tabla 2
\item El nombre de las tablas es acotado por lo tanto $O(t1.Nombre) \in O(1)$
\item La cantidad de campos de un registro es acotado por lo tanto $O(|campos(regComb)|) \in O(1)$
\item $O(buscar(r,t,b)$ es $O(log(m) + L)$ pues el campo del join es un campo clave con indice. Y todos los registros que se agreguen al la lista de modificaciones tiene al dicho campo.
\item $A$ es la complejidad de combinarRegistros(c, cr1, cr2) la cual es $O(\#cr1*\#cr2*L*min\lbrace \#cr1, \#cr2\rbrace)$. Pero como $\#cr1 = \#cr2 = 1$. Entonces la complejidad queda O(1*1*L*1), entonces por algebra de ordenes queda O(L)
\item La complejidad de la funcion AuxActContenedor es O((log(m+n) + L) + (A*L)) pero como el campo del join es indice y clave. Entonces queda O(log(m+n) + 2L) pues (A) es la cantidad de elementos que puede haber en un algun Indice. Y, como el campo es clave, esa cantidad es 1.
\item La complejidad del algoritmo es : O(R*((log(m) + L) + log(m+n) + 2L))
\item Aplicando algebra de Ordenes queda : O(R*(log(m+n)*L))
\end{enumerate}} % Justificación


\begin{algoritmo}{iAuxActContenedor}{\Inout{cont}{estrCont}, \In{reg}{reg}, \In{mod}{estrMod}, \In{c}{string},\In{nat?}{bool}}{}
	\eIf(\com*[f]{$O(1)$}){nat?}{
 		$DatoNat \gets ValorN(Significado(reg, c))$ \com*{$O(|campos(regComb)|)$} 
        \eIf(\com*[f]{$O(log(n+m))$}){Definido?(cont.IndiceNat, DatoNat)}{
        	$ ConjIts \gets Significado(cont.IndiceNat, DatoNat) $ \com*{$O(log(n+m))$}
        	\eIf(\com*[f]{$O(1)$}){mod.Inserto?}{
	    	    $itRegComb \gets AgregarRapido(cont.Registros, reg) $ \com*{$O(1)$}
    	    	$AgregarRapido(ConjIts, $<$itRegComb,NULL$>$) $ \com*{$O(1)$}
			}{
				$AuxBorrarDelJoin(ConjIts,c,DatoNat,nat?)$ \com*{$O(1)$}
	        }
        }{
        	$ConjIts \gets Vacio() $ \com*{$O(1)$}
	        $DefinirRapido(cont.IndiceNat, DatoNat, ConjIts) $ \com*{$O(log(n+m))$}
    	    \If(\com*[f]{$O(1)$}){mod.Inserto?}{
    	    	$itRegComb \gets AgregarRapido(cont.Registros, reg) $ \com*{$O(1)$}
	    	    $AgregarRapido(ConjIts, $<$itRegComb,NULL$>$) $ \com*{$O(1)$}
	        }

		}
	}{
          $DatoString \gets ValorS(Significado(regComb, c))$ \com*{$O(|campos(regComb)|)$}
	      \eIf(\com*[f]{$O(log(n+m))$}){Definido?(contenedor.IndiceString, DatoString)}{
          $ ConjIts \gets Significado(contenedor.IndiceString, DatoString) $ \com*{$O(log(n+m))$}
          \eIf(\com*[f]{$O(1)$}){mod.Inserto?}{
              $itRegComb \gets AgregarRapido(contenedor.Registros, regComb) $ \com*{$O(1)$}
              $AgregarRapido(ConjIts, $<$itRegComb,NULL$>$) $ \com*{$O(1)$}
          }{
				$AuxBorrarDelJoin()$ \com*{$O(1)$}
          }
          }{
              $ConjIts \gets Vacio() $ \com*{$O(1)$}
              $DefinirRapido(contenedor.IndiceString, DatoString, ConjIts) $ \com*{$O(log(n+m))$}
              \If(\com*[f]{$O(1)$}){mod.Inserto?}{
                  $itRegComb \gets AgregarRapido(contenedor.Registros, regComb) $ \com*{$O(1)$}
                  $AgregarRapido(ConjIts, $<$itRegComb,NULL$>$) $ \com*{$O(1)$}
              }  		             	
          }
   
    }
\end{algoritmo}
\datosAlgoritmo{Use este algoritmo auxiliar porque actualizar vista join no entraba en la pagina}
{}
{}
{$O((log(n+m) + L) + (A)*L) $}
{\begin{enumerate}
\item Busca el registro en el indice que corresponda. O(log(m+n) + L)
\item Si la modificacion fue de Borrar. Borrar en los indices segun corresponda. O(A*L)
\item A es la cantidad de elementos en el significado de la clave del indice a borrar.
\item Finalmente la complejidad del algoritmo es O((log(m+n) + L) + (A)*L)
\end{enumerate}}

\begin{algoritmo}{iAuxBorrarDelJoin}{\Inout{its}{conj}, \In{c}{string}, \In{dato}{Dato}, \In{nat?}{bool}}{}
    $itConjIts \gets CrearIt(its) $ \com*{$O(1)$}
    \While(\com*[f]{$O(|ConjIts|)$}){HaySiguiente(itConjIts)}{
	    $ itReg \gets Siguiente(itConjIts).Reg $\com*{$O(1)$}
    	$ regB \gets Siguiente(itReg) $ \com*{$O(1)$}
	    $ datoR \gets Significado(regB, c) $ \com*{$O(|campos(regB)|$}
		\If(\com*[f]{$O(L)$}){(nat? $\wedge$ ValorN(datoR) = dato)) $\vee$ (ValorS(datoR) = dato)}{
	        $EliminarSiguiente(itReg) $ \com*{$O(1)$}
            $EliminarSiguiente(itConjIts, it) $ \com*{$O(1)$}
        }
		$ Avanzar(itConjIts) $ \com*{$O(1)$}
    }
\end{algoritmo}
\datosAlgoritmo{Use este algoritmo auxiliar porque actualizar vista join no entraba en la pagina}
{}
{}
{O(A*L)}
{\begin{enumerate}
\item A es la cantidad de elementos en el significado de la clave del indice a borrar, o sea $\#ConjIts$.
\end{enumerate}}
\begin{algoritmo}{iActualizarJoin}{\In{inserta}{bool},\In{r}{reg}, \In{t}{string}, \Inout{b}{estrBD}}{}
    $itTbl \gets CrearIt(b.NombresTablas)$ \com*{$O(1)$}
    $datosT \gets Obtener(b.Tablas, t)$ \com*{$O(|t|)$}
    \While(\com*[f]{$O(|NombreTablas|*(2*copy(reg)))$}){($HaySiguiente(itTbl)$)}{
    	$nombreT \gets Siguiente(itTbl)$ \com*{$O(1)$}
        \If(\com*[f]{$O(|nombreT|)$}){HayJoin?(t, nombreT)}{
        	$join \gets Obtener(datosT.Joins, nombreT) $ \com*{$O(|nombreT|)$}
            $tup \gets <$inserta$, $copiar(reg)$> $ \com*{$O(copy(reg))$}
	        $AgregarAdelante(join.modificaciones, tup) $ \com*{$O(copy(reg))$}
        }
        $Avanzar(itTbl) $ \com*{$O(1)$}        
    }
    $itTbl \gets CrearIt(b.NombresTablas)$ \com*{$O(1)$}
    \While(\com*[f]{$O(|NombreTablas|*(2*copy(reg)))$}){($HaySiguiente(itTbl)$)}{
    	$nombreT \gets Siguiente(itTbl)$ \com*{$O(1)$}
        \If(\com*[f]{$O(1)$}){HayJoin?(nombreT, t)}{
            $datosT \gets Obtener(b.Tablas, nombreT)$ \com*{$O(|nombreT|)$}
        	$join \gets Obtener(datosT.Joins, t) $ \com*{$O(|t|) $}
            $tup \gets <$inserta$, $copiar(reg)$> $ \com*{$O(copy(reg))$}
	        $AgregarAdelante(join.modificaciones, tup) $ \com*{$O(copy(reg))$}
        }
        $Avanzar(itTbl) $ \com*{$O(1)$}    
    }
\end{algoritmo}

\datosAlgoritmo{} % Descripción
{$t \in tablas(b) $} % Pre
{$ (\forall t_{1}, t_{2} \in tablas(b)) \ hayJoin?(t1,t2,b) \impluego vistaJoin(t1,t2,b) \equiv combinarRegistros(campoJoin(t1,t2,b), registros(t1), registros(t2)) $} % Post
{$O(n*L)$} % Complejidad
{\begin{enumerate}
	\item $n$ es la cantidad de Tablas de la base de datos.
    \item $L$ es la longitud maxima de algun dato string del registro a copiar
	\item Recorre los joins de la tabla. O(n)
    \item Para cada join de la tabla. Agrega una copia del registro. O(L)
    \item Por los 2 items anteriores el costo total del while es O(n*L)
    \item Luego, recorre todas las tablas de la base de datos. O(n)
    \item Luego para cada tabla se pregunta si existe un join con la tabla $t$. Como los nombre de las tablas estan acotados, esta operaci\'on es O(1), por su definici\'on
    \item Si hay un join, lo recupera, y como el nombre de las tablas es acotado y las tablas estan implementadas sobre un DiccString, el Obtener es O(1)
    \item Luego, genera la modificacion para eso copia el registro. Copiar un registro cuesta O(L)
    \item Por ultimo, lo agrega a la lista de modificaciones del join. Esta operacion tambien copia el registro por lo tanto cuesta O(L)
    \item Finalmente, por lo items anteriores, el segundo while cuesta $O(n*(1+1+L+L) \in O(n*L)$
    \item Sino hay join, no se hace nada.
    \item Luego, la complejidad total del algoritmo es $O(n*L + n*L) \in O(n*L)$
\end{enumerate}} % Justificación



\begin{algoritmo}{icantidadDeAccesos}{\In{t}{nombreTabla} \In{b}{estrBD}}{conj(reg)}

	$datosT \gets Obtener(b.Tablas,t)$ \com*{$O(|t|)$}
	$res \gets cantidadDeAccesos(datosT.Tabla)$ \com*{$O(1)$}
    
\end{algoritmo}

\datosAlgoritmo{} % Descripción
{} % Pre
{} % Post
{$O(1)$} % Complejidad
{$O(|t|) + O(1) = O(1)$ porque los nombres de las tablas estan acotados} % Justificación

\begin{algoritmo}{itablaMaxima}{\In{b}{estrBD}}{nombreTabla}%

	$res \gets b.tablaMaxima$ \com*{$O(1)$}
   
\end{algoritmo}

\datosAlgoritmo{} % Descripción
{} % Pre
{} % Post
{$O(1)$} % Complejidad
{} % Justificación

\begin{algoritmo}{iencontrarMaximo}{\In{t}{nombreTabla} \In{ct}{conj(nombreTabla)} \In{b}{estrBD}}{nombreTabla}%
		
	$itTbl \gets CrearIt(ct)$ \com*{$O(1)$}
	$res \gets t$ \com*{$O(1)$}
	\While(\com*[f]{$O(1)$}){($HaySiguiente(itTbl)$)}{
		$nombreT \gets Siguiente(itTbl)$ \com*{$O(1)$}
		\If(\com*[f]{$O(1)$}){cantidadDeAccesos(dameTabla(res)) $\leq$ cantidadDeAccesos(dameTabla(nombreT)) }{
            $res \gets nombreT$ \com*{$O(1)$}
        }
		$Avanzar(itTbl)$ \com*{$O(1)$}
    }
\end{algoritmo}
\datosAlgoritmo{} % Descripción
{} % Pre
{} % Post
{$O(\#(ct))$} % Complejidad
{} % Justificación

\begin{algoritmo}{ibuscar}{\In{criterio}{registro} \In{t}{nombreTabla} \In{b}{estrBD}}{conj(reg)}%

	$res \gets coincidenciasRap(dameTabla(t,b),criterio)$ \com*{$O(coincidenciasRap)$}
    
\end{algoritmo}

\datosAlgoritmo{} % Descripción
{} % Pre
{} % Post
{$O(coincidenciasRap)$} % Complejidad
{Hereda la complejidad de coincidenciasRap: \\ En el caso de que la tabla de nombre $t$ no tenga indices o que ningun campo de $criterio$ sea indice en $t$, coincidenciasRap debe recorrer linealmente los registros de $t$ teniendo la complejidad de coincidencias: $O(n*L)$ \\ Si alguno de los campos de $criterio$ es indice solo debe buscar las coincidencias en el conjunto de registros que coincidan en el dato del campo indexado, los cuales obtengo realizando un $"Obtener"$ de ese dato en un diccStr o en un diccNat. Si alguno de los campos de $criterio$ ademas de ser indice es clave en t puede haber a lo sumo un registro en t que coincide con el valor de ese campo en $criterio$ por lo que al buscar los registros a traves del campo indexado voy a obtener un conjunto de 1 elemento por lo que la complejidad sera solamente la del $Obtener$ de los diccionarios: $O(log(n) + L)$} % Justificación

	
\end{Algoritmos}
    
    
    