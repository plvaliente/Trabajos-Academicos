\documentclass[a4paper,10pt]{article}
\usepackage[paper=a4paper, hmargin=1.5cm, bottom=1.5cm, top=3.5cm]{geometry}
\usepackage[latin1]{inputenc}
\usepackage[T1]{fontenc}
\usepackage[spanish,activeacute]{babel}
\usepackage{indentfirst}
\usepackage{fancyhdr}
\usepackage{latexsym}
\usepackage{lastpage}
\usepackage{xspace}
\usepackage{xargs}
\usepackage{ifthen}
\usepackage{aed2-symb,aed2-itef,aed2-tad,caratula}
\usepackage[colorlinks=true, linkcolor=blue]{hyperref}
\usepackage{calc}

\newcommand{\f}[1]{\text{#1}}
\newcommand{\fAux}{$_{\text{aux}}$}
\renewcommand{\paratodo}[2]{\ensuremath{\forall~#2: \text{#1}}}

% Macros de diseño provistas por la cátedra %

\usepackage{xspace}
\usepackage{xargs}
\usepackage{ifthen}

\newcommand{\moduloNombre}[1]{\textbf{#1}}

\let\NombreFuncion=\textsc
\let\TipoVariable=\texttt
\let\tipo=\texttt
\let\ModificadorArgumento=\textbf
\newcommand{\res}{$res$\xspace}
\newcommand{\tab}{\hspace*{7mm}}

\newcommandx{\TipoFuncion}[3]{%
  \NombreFuncion{#1}(#2) \ifx#3\empty\else $\to$ \res\,: \TipoVariable{#3}\fi%
}
\newcommand{\In}[2]{\ModificadorArgumento{in} \ensuremath{#1}\,: \TipoVariable{#2}\xspace}
\newcommand{\Out}[2]{\ModificadorArgumento{out} \ensuremath{#1}\,: \TipoVariable{#2}\xspace}
\newcommand{\Inout}[2]{\ModificadorArgumento{in/out} \ensuremath{#1}\,: \TipoVariable{#2}\xspace}
\newcommand{\Aplicar}[2]{\NombreFuncion{#1}(#2)}

\newlength{\IntFuncionLengthA}
\newlength{\IntFuncionLengthB}
\newlength{\IntFuncionLengthC}
%InterfazFuncion(nombre, argumentos, valor retorno, precondicion, postcondicion, complejidad, descripcion, aliasing)
\newcommandx{\InterfazFuncion}[9][4=true,6,7,8,9]{%
  \hangindent=\parindent
  \TipoFuncion{#1}{#2}{#3}\\%
  \textbf{Pre} $\equiv$ \{#4\}\\%
  \textbf{Post} $\equiv$ \{#5\}%
  \ifx#6\empty\else\\\textbf{Complejidad:} #6\fi%
  \ifx#7\empty\else\\\textbf{Descripci\'on:} #7\fi%
  \ifx#8\empty\else\\\textbf{Aliasing:} #8\fi%
  \ifx#9\empty\else\\\textbf{Requiere:} #9\fi%
}

\newenvironment{Interfaz}{%
  \parskip=2ex%
  \noindent\textbf{\Large Interfaz}%
  \par%
}{}

\newenvironment{Representacion}{%
  \vspace*{2ex}%
  \noindent\textbf{\Large Representaci\'on}%
  \vspace*{2ex}%
}{}

\newenvironment{Algoritmos}{%
  \vspace*{2ex}%
  \noindent\textbf{\Large Algoritmos}%
  \vspace*{2ex}%
}{}


\newcommand{\Encabezado}[1]{
  \vspace*{1ex}\par\noindent\textbf{\large #1}\par
}

\newenvironmentx{Estructura}[2][2={estr}]{%
  \par\vspace*{2ex}%
  \TipoVariable{#1} \textbf{se representa con} \TipoVariable{#2}%
  \par\vspace*{1ex}%
}{%
  \par\vspace*{2ex}%
}%

\newboolean{EstructuraHayItems}
\newlength{\lenTupla}
\newenvironmentx{Tupla}[1][1={estr}]{%
    \settowidth{\lenTupla}{\hspace*{3mm}donde \TipoVariable{#1} es \TipoVariable{tupla}$($}%
    \addtolength{\lenTupla}{\parindent}%
    \hspace*{3mm}donde \TipoVariable{#1} es \TipoVariable{tupla}$($%
    \begin{minipage}[t]{\linewidth-\lenTupla}%
    \setboolean{EstructuraHayItems}{false}%
}{%
    $)$%
    \end{minipage}
}

\newcommandx{\tupItem}[3][1={\ }]{%
    %\hspace*{3mm}%
    \ifthenelse{\boolean{EstructuraHayItems}}{%
        ,#1%
    }{}%
    \emph{#2}: \TipoVariable{#3}%
    \setboolean{EstructuraHayItems}{true}%
}

\newcommandx{\tupItemNL}[3][1={\ }]{%
    %\hspace*{3mm}%
    \ifthenelse{\boolean{EstructuraHayItems}}{%
        ,\\#1%
    }{}%
    \emph{#2}: \TipoVariable{#3}%
    \setboolean{EstructuraHayItems}{true}%
}

\newcommandx{\RepFc}[3][1={estr},2={e}]{%
  \tadOperacion{Rep}{#1}{bool}{}%
  \tadAxioma{Rep($#2$)}{#3}%
}%

\newcommandx{\Rep}[3][1={estr},2={e}]{%
  \tadOperacion{Rep}{#1}{bool}{}%
  \tadAxioma{Rep($#2$)}{true \ssi #3}%
}%

\newcommandx{\Abs}[5][1={estr},3={e}]{%
  \tadOperacion{Abs}{#1/#3}{#2}{Rep($#3$)}%
  \settominwidth{\hangindent}{Abs($#3$) \igobs #4: #2 $\mid$ }%
  \addtolength{\hangindent}{\parindent}%
  Abs($#3$) \igobs #4: #2 $\mid$ #5%
}%

\newcommandx{\AbsFc}[4][1={estr},3={e}]{%
  \tadOperacion{Abs}{#1/#3}{#2}{Rep($#3$)}%
  \tadAxioma{Abs($#3$)}{#4}%
}%

\newcommand{\DRef}{\ensuremath{\rightarrow}}

% Macros de diseño propias %

\usepackage{scrextend} % Para poder indentar bloques

\newenvironment{paramFormales}{
  \textbf{par\'ametros formales}
  \vspace{-0.5em}
  \list{}{\leftmargin8em \topsep0.2em \itemsep0.25em \labelsep2em}
}{
  \endlist 
}

\newcommand{\servUsados}[1]{\textbf{Servicios usados:} #1 \\}

\newcommand{\paramGeneros}[1]{\item[\textbf{g\'eneros}] #1}

\newcommand{\paramFuncion}[1]{\item[\textbf{funci\'on}] \parbox[t]{\textwidth-2\parindent-1.7cm}{#1}}

\newcommand{\seExplicaCon}[1]{\parbox{3cm}{\textbf{se explica con}:} \tadNombre{#1}}

\newcommand{\generos}[1]{\parbox{3cm}{\textbf{g\'eneros}:} #1}

\newcommand{\campoTupla}[2]{\textrm{\textit{#1:}} \TipoVariable{#2}}

\usepackage[noresetcount]{algorithm2e}
\usepackage{float}

\NoCaptionOfAlgo\LinesNumbered\RestyleAlgo{ruled}\IncMargin{1em}\DontPrintSemicolon\SetArgSty{}\SetCommentSty{textsf}\SetFuncSty{textsf}

\newenvironment{algoritmo}[3]{
  \setcounter{AlgoLine}{0}
  \begin{algorithm}[H]\SetAlgoLined\SetAlgoLongEnd
  \caption{\TipoFuncion{#1}{#2}{#3}}
}{
  \end{algorithm}
  \vspace{0em}
}

\newenvironment{contAlgoritmo}[1]{
  \begin{algorithm}[H]\SetAlgoLined\SetAlgoLongEnd
  \caption{\NombreFuncion{#1} \emph{(cont.)}}
}{
  \end{algorithm}
}

\newcommand{\datosAlgoritmo}[5]{
  \ifx#1\empty\else \textbf{Descripci\'on:} #1

  \fi \ifx#2\empty\else\textbf{Pre} $\equiv$ \{#2\}

  \fi \ifx#3\empty\else\textbf{Post} $\equiv$ \{#3\}

  \fi \textbf{Complejidad:} #4 

  \ifx#5\empty\else\textbf{Justificaci\'on:} #5 \fi \vspace{1em}
}

\SetKwComment{com}{ $\triangleright$ }{}
\def\new{\textbf{\&}}
\def\NULL{\textrm{NULL}}

% Operaciones básicas para algoritmos
\SetKwFunction{copiar}{Copiar}
\SetKwFunction{delete}{delete}
\SetKwFunction{crearArr}{CrearArreglo}

\SetKwFunction{maximo}{max}

\SetKwFunction{vacio}{Vacio}
\SetKwFunction{vacia}{Vacia}

\SetKwFunction{esVacio}{EsVacio?}
\SetKwFunction{esVacia}{EsVacia?}

\SetKwFunction{prim}{Primero}
\SetKwFunction{ult}{Ultimo}
\SetKwFunction{comienzo}{Comienzo}
\SetKwFunction{fin}{Fin}
\SetKwFunction{esta}{Esta?}
\SetKwFunction{longitud}{Longitud}

\SetKwFunction{pert}{Pertenece}
\SetKwFunction{ag}{Agregar}
\SetKwFunction{agRap}{AgregarRapido}
\SetKwFunction{agAtras}{AgregarAtras}
\SetKwFunction{agAdelante}{AgregarAdelante}
\SetKwFunction{card}{Cardinal}
\SetKwFunction{union}{Union}

\SetKwFunction{definido}{Definido}
\SetKwFunction{definir}{Definir}
\SetKwFunction{defRap}{DefinirRapido}
\SetKwFunction{isDef}{Definido?}
\SetKwFunction{signif}{Significado}
\SetKwFunction{obtener}{Obtener}
\SetKwFunction{borrar}{Borrar}

\SetKwFunction{crearIt}{CrearIt}
\SetKwFunction{haySig}{HaySiguiente?}
\SetKwFunction{sig}{Siguiente}
\SetKwFunction{avanzar}{Avanzar}
\SetKwFunction{sigClave}{SiguienteClave}
\SetKwFunction{sigSignif}{SiguienteSignificado}
\SetKwFunction{elimSig}{EliminarSiguiente}

\SetKwFunction{encolar}{Encolar}
\SetKwFunction{desencolar}{Desencolar}

\usepackage[tikz]{bclogo}
\newcommand{\nuevoAlgo}[0]{
  \begin{bclogo}[logo=\bcattention, noborder=true, barre=none]{ Algoritmo modificado}
  \end{bclogo}
}

\sloppy

\hypersetup{%
 % Para que el PDF se abra a pagina completa.
 pdfstartview= {FitH \hypercalcbp{\paperheight-\topmargin-1in-\headheight}},
 pdfauthor={Grupo 11},
 pdfkeywords={trabajo, pr\'actico, algoritmos, dise\~no},
 pdfsubject={Trabajo pr\'actico 2 - Dise\~no - DCNet}
}

\parskip=5pt % 10pt es el tamaño de fuente

% Pongo en 0 la distancia extra entre ítemes.
\let\olditemize\itemize
\def\itemize{\olditemize\itemsep=0pt}

% Acomodo fancyhdr.
\pagestyle{fancy}
\thispagestyle{fancy}
\addtolength{\headheight}{1pt}
\lhead{Algoritmos y Estructuras de Datos II}
\rhead{Trabajo Pr\'actico 2 - Dise\~no - Base de Datos}
\cfoot{\thepage /\pageref{LastPage}}
\renewcommand{\footrulewidth}{0.4pt}

\author{Grupo 14}
\date{}
\title{Trabajo pr\'actico 2 - Dise\~no - Base de Datos}

\def\Materia{Algoritmos y Estructuras de Datos II}
\def\Titulo{Trabajo pr\'actico 2}
\def\Subtitulo{Dise\~no - Base de Datos}
\def\Grupo{Grupo 14}
\integrante{Zar Abad, Ciro Rom\'an}{129/15}{ciromanzar@gmail.com}
\integrante{Lopez Valiente, Patricio}{457/15}{patricio454@gmail.com}
\integrante{Delmagro, Agust\'in}{596/14}{agustin.delmagro@gmail.com}
\integrante{Vercinsky, Iv\'an}{141/15}{ivan9074@gmail.com}

\begin{document}

\maketitle
%\newpage\null\thispagestyle{empty}\newpage

\tableofcontents
%\newpage\null\thispagestyle{empty}\newpage

\section{Extension de Modulos basicos}

% \Encabezado{Diccionario Lineal($\kappa$, $\sigma$)}

% \begin{Interfaz}
  
%   \Encabezado{Funciones Extendidas}

% %%%%%%%%%%%%%%%%%%%%%%%%%%%%%%%%%%%%%%%%%%%
% %%%%%%%%%        FUNCIONES       %%%%%%%%%%
% %%%%%%%%%%%%%%%%%%%%%%%%%%%%%%%%%%%%%%%%%%%

% \tadAlinearFunciones{coincidenTodosAux}{$\kappa$ \ $c$, {{dicc($\kappa$, $\sigma$)}} \ $d$, conj({{dicc($\kappa$, $\sigma$)}}) \ $cd$}

% \tadOperacion{campos}{{{dicc($\kappa$, $\sigma$)}}}{conj($\kappa$)}{}

% \tadOperacion{borrar?}{{{dicc($\kappa$, $\sigma$)}} /$crit$, {{dicc($\kappa$, $\sigma$)}}}{bool}{$\#$campos($crit$) $\equiv$ 1}

% \tadOperacion{agregarCampos}{{{dicc($\kappa$, $\sigma$)}} \ $d_1$, {{dicc($\kappa$, $\sigma$)}} \ $d_2$}{dicc($\kappa$, $\sigma$)}{}

% \tadOperacion{copiarCampos}{conj($\kappa$) \ $cc$, {{dicc($\kappa$, $\sigma$)}} \ $d_1$ , {{dicc($\kappa$, $\sigma$)}}  \ $d_2$}{dicc($\kappa$, $\sigma$)}{$cc$ $\in$ campos($d_2$)}

% \tadOperacion{coincideAlguno}{{{dicc($\kappa$, $\sigma$)}} \ $d_1$, conj($\kappa$) \ $cc$, {{dicc($\kappa$, $\sigma$)}} \ $d_2$}{bool}{$cc$ $\subseteq$ campos($d_1$) $\cap$ campos($d_2$)}

% \tadOperacion{coincidenTodos}{{{dicc($\kappa$, $\sigma$)}} \ $d_1$, conj($\kappa$) \ $cc$, {{dicc($\kappa$, $\sigma$)}} \ $d_2$}{dicc$(\kappa, \sigma)$}{$cc$ $\subseteq$ campos($d_1$) $\cap$ campos($d_2$)}

% \tadOperacion{coincidenTodosAux}{{{dicc($\kappa$, $\sigma$)}} \ $d_1$, conj($\kappa$) \ $cc$, {{dicc($\kappa$, $\sigma$)}} \ $d_2$}{dicc$(\kappa, \sigma)$}{$cc$ $\subseteq$ campos($d_1$)}

% \tadOperacion{enTodos}{$\kappa$ \ $c$, conj({{dicc($\kappa$, $\sigma$)}}) \ $cd$}{bool}{}

% \tadOperacion{combinarTodos}{$\kappa$ \ $c$, {{dicc($\kappa$, $\sigma$)}} \ $d$, conj({{dicc($\kappa$, $\sigma$)}}) \ $cd$}{conj({{dicc($\kappa$, $\sigma$)}})}{$c$ $\in$ campos($d_1$) $\land$ enTodos($c$,$cd$)}

% %%%%%%%%%%%%%%%%%%%%%%%%%%%%%%%%%%%%%%%%%%%
% %%%%%%%%%         AXIOMAS        %%%%%%%%%%
% %%%%%%%%%%%%%%%%%%%%%%%%%%%%%%%%%%%%%%%%%%%

% \tadAxiomas[\paratodo{dicc($\kappa$, $\sigma$)}{d, d_1, d_2, crit}, \paratodo{conj(dicc($\kappa$, $\sigma$))}{cd}, \paratodo{conj($\kappa)$}{cc}, \paratodo{$\sigma$}{d_1, d_2}, \paratodo{$\kappa$}{c} ]

% \tadAlinearAxiomas{coicidenTodosAux($d_1$, $cc$, $d_2$)cccc}

% \tadAxioma{campos($d$)}{claves($d$)}

% \tadAxioma{borrar?($crit$, $d$)}{coincidenTodos($crit$, campos($crit$), $d$)}

% \tadAxioma{agregarCampos($d_1$, $d_2$)}{copiarCampos(campos($d_2$) - campos($d_1$), $d_1$, $d_2$)}

% \tadAxioma{copiarCampos($cc$, $d_1$, $d_2$)}{
% $\textbf{if}$ ($\emptyset$?($cc$)) $\emph{then}$ \\
% 	$\hspace*{10px}$ $r_1$ \\
% $\textbf{else}$ \\
% 	$\hspace*{10px}$ copiarCampos(sinUno($cc$), definir(dameUno($cc$), \\
% 	$\hspace*{10px}$ obtener(dameUno($cc$), $d_2$),$d_1$), $d_2$ ) \\
% $\textbf{fi}$}

% \tadAxioma{coicideAlguno($d_1$, $cc$, $d_2$)}{$\neg \emptyset$?($cc$) $\yluego$ ((obtener(dameUno($cc$),$d_1$) $=$ obtener(dameUno($cc$), $d_2$)) $\lor$ \\ coincideAlguno($d_1$, sinUno($cc$), $d_2$))} 

% \tadAxioma{coincidenTodos($d_1$, $cc$, $d_2$)}{$\emptyset$?($cc$) $\oluego$ ((obtener(dameUno($cc$),$d_1$) $=$ obtener(dameUno($cc$), $d_2$)) $\land$ \\ coincidenTodos($d_1$, sinUno($cc$), $d_2$))}


% \tadAxioma{coincidenTodosAux($d_1$, $cc$, $d_2$)}{$\emptyset$?($cc$) $\oluego$ def?(dameUno($cc$), $d_2$) $\yluego$ ((obtener(dameUno($cc$),$d_1$) $=$ obtener(dameUno($cc$), $d_2$)) $\land$ \\ coincidenTodos($d_1$, sinUno($cc$), $d_2$))}

% \tadAxioma{enTodos($c$, $cd$)}{$\emptyset$?($cd$) $\oluego$ ($c$ $\in$ campos(dameUno($cd$)) $\land$ enTodos($c$, sinUno($cd$)))}

% \tadAxioma{combinarTodos($c$, $d$, $cd$)}{
% $\textbf{if}$ ($\emptyset$?($cd$)) $\emph{then}$ \\
% 	$\hspace*{10px}$ $\emptyset$ \\
% $\textbf{else}$ \\
% 	$\hspace*{10px}$ combinarTodos($c$, $d$, sinUno($cd$)) $\cup$ $\textbf{if}$ (obtener($c$, dameUno($cd$)) $=$ \\
% 	$\hspace*{10px}$ obtener($c$, $d$)) $\textbf{\emph{then}}$ 	$\{$agregarCampos($d$, dameUno($cd$)) $\}$ $\textbf{else}$ $\emptyset$ $\textbf{fi}$ \\
% $\textbf{fi}$}
    
% $\vspace*{15px}$
  
%   \Encabezado{Funciones Extendidas}  
  
%   \InterfazFuncion{Campos}{\In{d}{dicc$(\kappa, \sigma)$}}{conj($\kappa$)}
%   [true] % Pre
%   {$res$ $\igobs$ campos($d$)} % Pos
%   [$O(\#Claves(d)*copy(\kappa))$] % Complejidad
%   [Retorna las claves del diccionario] % Descripción
%   [] % Aliasing
  
%   \InterfazFuncion{Borrar?}{\In{crit}{dicc$(\kappa, \sigma)$}, \In{d}{dicc$(\kappa, \sigma)$}}{bool}
%   [$\#$ campos($crit$) $\equiv$ 1] % Pre
%   {$res$ $\igobs$ borrar?($crit$,$d$)} % Pos
%   [$O(\sum_{k' \in K}equal(c,k') + equal(g_{crit},g_d))$] % Complejidad
%   [Dice sin un Diccionario es Borrable segun el criterio crit(que matchee el significado en $d$ y $crit$ para el campo(unico) de $crit$ ] % Descripción
%   [] % Aliasing
  
%    \InterfazFuncion{AgregarCampos}{\In{d_1}{dicc$(\kappa, \sigma)$}, \In{d_2}{dicc$(\kappa, \sigma)$}}{dicc$(\kappa, \sigma)$}
%   [true] % Pre
%   {$res$ $\igobs$ agregarCampos($d_1$,$d_2$)} % Pos
%   [$O(\#cc*\sum_{k' \in K}(equal(k,k') + copy(sc) + copy(\text{significado}(sc,d_1)))+(Copy(l) + Copy(g)))$, donde K $=$ Claves($d_1$), $sc$ $\in$ Claves($res$), $l$ $\in$ Claves($d_2$) y $g$ $=$ Significado($d_2$,$l$)  ] % Complejidad
%   [Retorna un Diccionario con las claves de $d_1$ y su respectivo significado, mas las claves que posee $d_2$ y no $d_1$, tambien con el significado que poseia en $d_2$] % Descripción
%   [] % Aliasing
  
%    \InterfazFuncion{CopiarCampos}{\In{cc}{conj($\kappa$)}, \In{d_1}{dicc$(\kappa, \sigma)$}, \In{d_2}{dicc$(\kappa, \sigma)$}}{dicc$(\kappa, \sigma)$}
%   [$cc$ $\subseteq$ campos($d_2$)] % Pre
%   {$res$ $\igobs$ copiarCampos($cc$,$dd1$,$d_2$)} % Pos
%   [$O(\#Claves(d_2)*\sum_{k' \in K}(equal(k,k') + copy(k) + copy(\text{significado}(k,d_1)))+(Copy(l) + Copy(g)))$, donde K $=$ Claves($d_1$), $l$ $\in$ Claves($d_2$) y $g$ $=$ Significado($d_2$,$l$)  ] % Complejidad
%   [Retorna un Diccionario con las claves de $d_1$ y su respectivo significado, mas las claves pertenecientes a $cc$ con el significado que tienen en $d_2$] % Descripción
%   [] % Aliasing
  
%    \InterfazFuncion{CoincideAlguno}{\In{d_1}{dicc$(\kappa, \sigma)$}, \In{cc}{conj($\kappa$)}, \In{d_2}{dicc$(\kappa, \sigma)$}}{bool}
%   [$cc$ $\subseteq$ (campos($d_1$) $\cap$ campos($d_2$))] % Pre
%   {$res$ $\igobs$ coincideAlguno($d_1$,$cc$,$d_2$)} % Pos
%   [$O(\#cc*(Sig_1 + Sig_2 + equal(g_{d_1},g_{d_2})))$, donde $Sig_i$ es $\sum_{k' \in K}(equal(k,k')$ K $=$ Claves($d_i$) y $g_i$ $=$ Significado($i$,$c$), $c \in cc$  ] % Complejidad
%   [Devuelve Verdadero si alguna clave perteneciente a $cc$ tiene el mismo significado en $d_1$ y en $d_2$ ] % Descripción
%   [] % Aliasing
  
%   \InterfazFuncion{CoincidenTodos}{\In{d_1}{dicc$(\kappa, \sigma)$}, \In{cc}{conj($\kappa$)}, \In{d_2}{dicc$(\kappa, \sigma)$}}{bool}
%   [$cc$ $\subseteq$ (campos($d_1$) $\cap$ campos($d_2$))] % Pre
%   {$res$ $\igobs$ coincidenTodos($d_1$,$cc$,$d_2$)} % Pos
%   [$O(\#cc*(Sig_1 + Sig_2 + equal(g_{d_1},g_{d_2})))$, donde $Sig_i$ es $\sum_{k' \in K}(equal(c,k')$ K $=$ Claves($d_i$) y $g_i$ $=$ Significado($i$,$c$), $c \in cc$ ] % Complejidad
%   [Devuelve Verdadero si para toda clave perteneciente a $cc$ tiene el mismo significado en $d_1$ y en $d_2$ ] % Descripción
%   [] % Aliasing

%  \InterfazFuncion{CoincidenTodosAux}{\In{d_1}{dicc$(\kappa, \sigma)$}, \In{cc}{conj($\kappa$)}, \In{d_2}{dicc$(\kappa, \sigma)$}}{bool}
%   [$cc$ $\subseteq$ campos($d_1$)] % Pre
%   {$res$ $\igobs$ coincidenTodosAux($d_1$,$cc$,$d_2$)} % Pos
%   [$O(\#cc*(Sig_1 + def_2 + Sig_2 + equal(g_{d_1},g_{d_2})))$, donde $Sig_i$ es $\sum_{k' \in K}(equal(c,k')$ K $=$ Claves($d_i$), $def_2$ $=$ $Sig_2$ y $g_i$ $=$ Significado($i$,$c$), $c \in cc$ ] % Complejidad
%   [Devuelve Verdadero si para toda clave perteneciente a $cc$ tiene el mismo significado en $d_1$ y en $d_2$ ] % Descripción
%   [] % Aliasing

%   \InterfazFuncion{enTodos}{\In{c}{$\kappa$}, \In{cd}{conj(dicc$(\kappa, \sigma)$)}}{bool}
%   [true] % Pre
%   {$res$ $\igobs$ enTodos($c$,$cd$)} % Pos
%   [$O(\#cr*\sum_{k' \in K}(equal(c,k')))$, K $=$ Claves($r$) $r$ $\in$ $cr$ ] % Complejidad
%   [Devuelve Verdadero si $c$ es una clave definida en cada diccionario perteneciente a $cd$ ] % Descripción
%   [] % Aliasing

% \newpage

%   \InterfazFuncion{CombinarTodos}{\In{c}{$\kappa$}, \In{d}{dicc$(\kappa, \sigma)$}, \In{cd}{conj(dicc$(\kappa, \sigma)$)}}{conj(dicc$(\kappa, \sigma)$)}
%   [$c$ $\in$ campos($r_1$) $\land$ enTodos($c$,$cr$)] % Pre
%   {$res$ $\igobs$ combinarTodos($c$,$d$,$cd$)} % Pos
%   [$O(Sig_d + \#cr*(Sig_r + equal(sc,g_{r}) + Ag))$, donde $Sig_i$ es $\sum_{k' \in K}(equal(c,k')$, $r$ $\in$ $cr$, K $=$ Claves($i$), $g_i$ $=$ Significado($i$,$c$), $d$ $=$ AgregarCampos($d$,$r$) y $Ag$ es $\sum_{d' \in res}(equal(d,d'))$] % Complejidad
%   [Devuelve un Conjunto que posee por elementos a diccionarios con las claves de $d$ y su respectivo significado, mas las claves que posee $d_i$ y no $d$, tambien con el significado que poseia en $d_i$, donde $d_i$ son los diccionarios pertenecientes a $cd$ tales que coicidan en su significado con $d$ para la clave $c$  ] % Descripción
%   [] % Aliasing

% \InterfazFuncion{Sub?}{\In{d_1}{dicc$(\kappa, \sigma)$}, \In{d_2}{dicc$(\kappa, \sigma)$}}{bool}
%   [true] % Pre
%   {$res$ $\igobs$ coincidenTodos($d_1$,claves($d_1$),$d_2$)} % Pos
%   [$O(\#claves(d_1)*(def_1 + Sig_2 + equal(g_{d_1},g_{d_2})))$, donde $def_1$ es $\sum_{t' \in T}(equal(c,t')$ T $=$ claves($d_1$),$Sig_2$ es $\sum_{k' \in K}(equal(c,k')$ K $=$ claves($d_2$) y $g_i$ $=$ Significado($i$,$c$), $c$ $\in$ claves($d_1$)] % Complejidad
%   [Devuelve Verdadero si para toda clave perteneciente a $claves(d_1)$,existe esta en $d_2$  y tiene el mismo significado que en $d_1$] % Descripción
%   [] % Aliasing

% \end{Interfaz}

% \begin{Algoritmos}

% \begin{algoritmo}{iCampos}{\In{d}{dic}}{conj($\kappa$)}
% 	$it \gets CrearIt(d.claves) $ \com*{$\Theta(1)$}
%     $res \gets Vacio() $ \com*{$\Theta(1)$}
%     \While(\com*[f]{$\Theta(1)$}){($HaySiguiente(it)$)}{
%      	$AgregarRapido(res, SiguienteClave(it))$ \com*{$\Theta(copy(sigCl(it)))$}
% 		$Avanzar(it)$ \com*{$\Theta(1)$}
% 	}
%     \medskip
% 	\underline{Complejidad:} $O(\#Claves(d)*copy(\kappa))$
%     \end{algoritmo}

% \begin{algoritmo}{iBorrar?}{\In{crit}{dic}, \Inout{d}{dic}}{bool}
% 	$itcrit \gets CrearIt(crit) $ \com*{$\Theta(1)$}
%     $c \gets SiguienteClave(itcrit) $ \com*{$\Theta(1)$}
%     $res \gets Significado(crit,c) = Significado(d,c)$ \com*{$\Theta(\sum_{k' \in K}equal(c,k') + equal(g_{crit},g_d))$}
% 	\medskip
% 	\underline{Complejidad:} $O(\sum_{k' \in K}equal(c,k') + equal(g_{crit},g_d))$, donde $K$ $=$ claves($d$), y $g_i$ $=$ Significado($i$,$c$)  
% \end{algoritmo}

% \begin{algoritmo}{iAgregarCampos}{\In{d_1}{dic}, \Inout{d_2}{dic}}{dic}
%     % lo hago asi porque no tengo la operacion resta de conjuntos, ni se su complejidad
%     $it2 \gets CrearIt(d_2) $ \com*{$\Theta(1)$}
%     $res \gets Copiar(d_1) 	$ \com*{$\Theta(\sum_{k \in K}(copy(k) + copy(\text{significado}(k,d_1)))$}
%     \While(\com*[f]{$\Theta(1)$}){($HaySiguiente(it2)$)}{
%     	\If(\com*[f]{$\Theta(\sum_{k' \in K}equal(SigCl(it2),k'))$}){$\lnot Definido?(res,SiguienteClave(it2))$}{
% 			$DefinirRapido(res, SiguienteClave(it2), SiguienteSignificado(it2))$ \com*{$\Theta(copy(k) + copy(s))$}
% 		}
%      	$Avanzar(it2)$ \com*{$\Theta(1)$}
% 	}
%     \medskip
% 	\underline{Complejidad:} $O(\#Claves(d_2)*\sum_{k' \in K}(equal(k,k') + copy(k) + copy(\text{significado}(k,d_1)))+(Copy(l) + Copy(g)))$, donde K $=$ Claves($d_1$), $l$ $\in$ Claves($d_2$) y $g$ $=$ Significado($d_2$,$l$) 
% \end{algoritmo}

% \begin{algoritmo}{iCopiarCampos}{\In{cc}{conj($\kappa$)}, \In{d_1}{dic}, \In{d_2}{dic}}{dic}
%     $itc \gets CrearIt(cc) $ \com*{$\Theta(1)$}
%     $res \gets Copiar(d_1) 	$ \com*{$\Theta(\sum_{k \in K}(copy(k) + copy(\text{significado}(k,d_1)))$}
%     \While(\com*[f]{$\Theta(1)$}){($HaySiguiente(itc)$)}{
%      	\If(\com*[f]{$\Theta(\sum_{k' \in K}equal(sigCl(itc),k'))$}){$\lnot Definido?(res,SiguienteClave(itc))$}{
% 			$DefinirRapido(res, SiguienteClave(itc), SiguienteSignificado(itc))$ \com*{$\Theta(copy(sc) + copy(s))$}
% 		}
%      	$Avanzar(it2)$ \com*{$\Theta(1)$}
% 	}
%     \medskip
% 	\underline{Complejidad:} $O(\#cc*\sum_{k' \in K}(equal(k,k') + copy(sc) + copy(\text{significado}(sc,d_1)))+(Copy(l) + Copy(g)))$, donde K $=$ Claves($d_1$), $sc$ $\in$ Claves($res$), $l$ $\in$ Claves($d_2$) y $g$ $=$ Significado($d_2$,$l$) 
% \end{algoritmo}

% \begin{algoritmo}{iCoincideAlguno}{\In{d_1}{dic}, \In{cc}{conj($\kappa$)}, \In{d_2}{dic}}{bool}
%     $itc \gets CrearIt(cc) $ \com*{$\Theta(1)$}
%     $res \gets false $ \com*{$\Theta(1)$}
%     \While(\com*[f]{$\Theta(1)$}){($HaySiguiente(itc) \land not res$)}{
%     	$res \gets Significado(d_1,Siguiente(itc)) = Significado(d_2,Siguiente(itc))$ \com*{$\Theta(Sig_1 + Sig_2 + equal(g_{d_1},g_{d_2}))$}
% 	$Avanzar(itc)$ \com*{$\Theta(1)$}
%     }
%     \medskip
% 	\underline{Complejidad:} $O(\#cc*(Sig_1 + Sig_2 + equal(g_{d_1},g_{d_2})))$, donde $Sig_i$ es $\sum_{k' \in K}(equal(k,k')$ K $=$ Claves($d_i$) y $g_i$ $=$ Significado($i$,$c$), $c \in cc$ 
% \end{algoritmo}

% \begin{algoritmo}{iCoincidenTodos}{\In{d_1}{dic}, \In{cc}{conj($\kappa$)}, \In{d_2}{dic}}{bool}
%     $itc \gets CrearIt(cc) $ \com*{$\Theta(1)$}
%     $res \gets true $ \com*{$\Theta(1)$}
%     \While(\com*[f]{$\Theta(1)$}){($HaySiguiente(itc) \wedge res$)}{
%     	$res \gets Significado(d_1,Siguiente(itc)) = Significado(d_2,Siguiente(itc))$ \com*{$\Theta(Sig_1 + Sig_2 + 		equal(g_{d_1},g_{d_2}))$}
% 		$Avanzar(itc)$ \com*{$\Theta(1)$}
% 	}
%     \medskip
% 	\underline{Complejidad:} $O(\#cc*(Sig_1 + Sig_2 + equal(g_{d_1},g_{d_2})))$, donde $Sig_i$ es $\sum_{k' \in K}(equal(c,k')$ K $=$ Claves($d_i$) y $g_i$ $=$ Significado($i$,$c$), $c \in cc$ 
% \end{algoritmo}

% \begin{algoritmo}{iCoincidenTodosAux}{\In{d_1}{dic}, \In{cc}{conj($\kappa$)}, \In{d_2}{dic}}{bool}
%     $itc \gets CrearIt(cc) $ \com*{$\Theta(1)$}
%     $res \gets true $ \com*{$\Theta(1)$}
%     \While(\com*[f]{$\Theta(1)$}){($HaySiguiente(itc) \wedge res$)}{
%     	\eIf(\com*[f]{$\Theta(\sum_{k' \in claves(d_2)}equal(sigCl(itc),k'))$}){$Definido?(d_2,SiguienteClave(itc))$}{
% 		$res \gets Significado(d_1,Siguiente(itc)) = Significado(d_2,Siguiente(itc))$ \com*{$\Theta(Sig_1 + Sig_2 + 		equal(g_{d_1},g_{d_2}))$}
% 		}{
%         $res \gets false $ \com*{$\Theta(1)$}
%         }
%     	$Avanzar(itc)$ \com*{$\Theta(1)$}
% 	}
%     \medskip
% 	\underline{Complejidad:} $O(\#cc*(Sig_1 + def_2 + Sig_2 + equal(g_{d_1},g_{d_2})))$, donde $Sig_i$ es $\sum_{k' \in K}(equal(c,k')$ K $=$ Claves($d_i$), $def_2$ $=$ $Sig_2$ y $g_i$ $=$ Significado($i$,$c$), $c \in cc$ 
% \end{algoritmo}

% \begin{algoritmo}{ienTodos}{\In{c}{campo}, \In{cr}{conj(dicc$(\kappa, \sigma)$)}}{bool}
%     $itcr \gets CrearIt(cr) $ \com*{$\Theta(1)$}
%     $res \gets true $ \com*{$\Theta(1)$}
%     \While(\com*[f]{$\Theta(1)$}){($HaySiguiente(itcr) \wedge res$)}{
%     	$res \gets Definido?(Siguiente(itcr), c)$\com*{$\Theta(\sum_{k' \in K}(equal(c,k')))$}
% 		$Avanzar(itc)$ \com*{$\Theta(1)$}
% 	}
%     \medskip
% 	\underline{Complejidad:} $O(\#cr*\sum_{k' \in K}(equal(c,k')))$, K $=$ Claves($r$) $r$ $\in$ $cr$
% \end{algoritmo}

% \begin{algoritmo}{iCombinarTodos}{\In{c}{$\kappa$}, \In{d}{dic}, \In{cr}{conj(dicc$(\kappa, \sigma)$)}}{conj(dicc$(\kappa, \sigma)$)}
%     $itcr \gets CrearIt(cr) $ \com*{$\Theta(1)$}
%     $res \gets Vacio() $ \com*{$\Theta(1)$}
%     $sc  \gets Significado(d,c)$ \com*{$O(Sig_d)$} 
%    \While(\com*[f]{$\Theta(1)$}){($HaySiguiente(itcr)$)}{
%     	\If(\com*[f]{$O(Sig_{Sig(itcr)} + equal(sc,g_{r}))$}){($sc = Significado(Siguiente(itcr),c)$)}{
% 			$Agregar(res, AgregarCampos(d, Siguiente(itcr)))$ \com*{$\Theta(\sum_{d' \in res}(equal(d,d')))$}
% 		}
%      	$Avanzar(itcr)$ \com*{$\Theta(1)$}	
%    }
%     \medskip
% 	\underline{Complejidad:} $O(Sig_d + \#cr*(Sig_r + equal(sc,g_{r}) + Ag))$, donde $Sig_i$ es $\sum_{k' \in K}(equal(c,k')$, $r$ $\in$ $cr$, K $=$ Claves($i$), $g_i$ $=$ Significado($i$,$c$), $d$ $=$ AgregarCampos($d$,$r$) y $Ag$ es $\sum_{d' \in res}(equal(d,d'))$
% \end{algoritmo}

% \begin{algoritmo}{iSub?}{\In{d_1}{dic}, \In{d_2}{dic}}{bool}
%     $itd1 \gets CrearIt(d_1) $ \com*{$\Theta(1)$}
%     $res \gets true $ \com*{$\Theta(1)$}
%     \While(\com*[f]{$\Theta(1)$}){($HaySiguiente(itd1) \land res$)}{
%     	\eIf(\com*[f]{$\Theta(\sum_{k \in K}equal(SigCl(itd1),k'))$}){$\lnot Definido?(d_2,SiguienteClave(itd1))$}{
% 			$res \gets SiguienteSignificado(itd1) = Significado(d_2,SiguienteClave(itd1))$ \com*{$\Theta(Sig_2 + 		equal(SigSdo(itd1),g_{d_2}))$}
% 		}{
%         	$res \gets false $ \com*{$\Theta(1)$}
%         }
% 		$Avanzar(itd1)$ \com*{$\Theta(1)$}
% 	}
%     \medskip
% 	\underline{Complejidad:} $O(\#claves(d_1)*(def_1 + Sig_2 + equal(g_{d_1},g_{d_2})))$, donde $def_1$ es $\sum_{t' \in T}(equal(c,t')$ T $=$ claves($d_1$),$Sig_2$ es $\sum_{k' \in K}(equal(c,k')$ K $=$ claves($d_2$) y $g_i$ $=$ Significado($i$,$c$), $c$ $\in$ claves($d_1$)
% \end{algoritmo}

% \end{Algoritmos}















\Encabezado{Conjunto Lineal($\alpha$)}


  
  \Encabezado{Funciones Extendidas}

%%%%%%%%%%%%%%%%%%%%%%%%%%%%%%%%%%%%%%%%%%%
%%%%%%%%%        FUNCIONES       %%%%%%%%%%
%%%%%%%%%%%%%%%%%%%%%%%%%%%%%%%%%%%%%%%%%%%

\tadAlinearFunciones{Minimo}{conj($\alpha$) \ c}

\tadOperacion{Minimo}{conj($\alpha$) \ $c$}{$\alpha$}{$\neg$ $\emptyset$?(c)}

\tadOperacion{Maximo}{conj($\alpha$) \ $c$}{$\alpha$}{$\neg$ $\emptyset$?(c)}


%%%%%%%%%%%%%%%%%%%%%%%%%%%%%%%%%%%%%%%%%%%
%%%%%%%%%         AXIOMAS        %%%%%%%%%%
%%%%%%%%%%%%%%%%%%%%%%%%%%%%%%%%%%%%%%%%%%%

\tadAxiomas[\paratodo{conj($\alpha$)}{c}, \paratodo{$\alpha$}{d}]

\tadAlinearAxiomas{Minimo(conj($\alpha$))}

\tadAxioma{Minimo($c$)}{
$\textbf{if}$ ($\emptyset$?(SinUno($c$))) $\emph{then}$ \\
	$\hspace*{10px}$ DameUno($c$) \\
$\textbf{else}$ \\
	$\hspace*{10px}$ $\textbf{if}$ dameUno($c$)) $\leq$ Minimo(SinUno($c$))\\
		$\hspace*{20px}$ dameUno($c$) \\
    $\hspace*{10px}$ $\textbf{else}$ \\
    	$\hspace*{20px}$ Minimo(SinUno($c$)) \\
    $\hspace*{10px}$ $\textbf{fi}$ \\
$\textbf{fi}$}

\tadAxioma{Maximo($c$)}{
$\textbf{if}$ ($\emptyset$?(SinUno($c$))) $\emph{then}$ \\
	$\hspace*{10px}$ DameUno($c$) \\
$\textbf{else}$ \\
	$\hspace*{10px}$ $\textbf{if}$ dameUno($c$)) $\geq$ Maximo(SinUno($c$))\\
		$\hspace*{20px}$ dameUno($c$) \\
    $\hspace*{10px}$ $\textbf{else}$ \\
    	$\hspace*{20px}$ Maximo(SinUno($c$)) \\
    $\hspace*{10px}$ $\textbf{fi}$ \\
$\textbf{fi}$}
    
%%%%%%%%%%%%%%%%%%%%%%%%%%%%%%%%%%%%%%%%%%%
%%%%%%%%%        INTERFAZ        %%%%%%%%%%
%%%%%%%%%%%%%%%%%%%%%%%%%%%%%%%%%%%%%%%%%%%    
  \begin{Interfaz}
  %\Encabezado{Funciones Extendidas}  
  
  \InterfazFuncion{Minimo}{\In{c}{conj($\alpha$)}}{$\alpha$}
  [$\neg$ $\emptyset$?(c)] % Pre
  {$res$ $\igobs$ Minimo($c$)} % Pos
  [$O(\sum_{k \in c}equal(k,\alpha) + copy(\alpha)) $] % Complejidad
  [Retorna el menor elemento del conjunto] % Descripción
  [] % Aliasing
  
  \InterfazFuncion{Maximo}{\In{c}{conj($\alpha$)}}{$\alpha$}
  [$\neg$ $\emptyset$?(c)] % Pre
  {$res$ $\igobs$ Maximo($c$)} % Pos
  [$O(\sum_{k \in c}equal(k,\alpha) + copy(\alpha)) $] % Complejidad
  [Retorna el mayor elemento del conjunto] % Descripción
  [] % Aliasing

\end{Interfaz}

\begin{Algoritmos}

\begin{algoritmo}{iminimo}{\In {c}{conj($\alpha$)}}{$\alpha$}
	$itc \gets crearIt(c)$ \com*{$\Theta(1)$}
    $res \gets Siguiente(itc)$ \com*{$\Theta(1)$}
    \While(\com*[f]{$\Theta(1)$}){($HaySiguiente(itc)$)}{
    	\If(\com*[f]{$O(equal(res,Siguiente(itc)))$}){$\neg (res \leq Siguiente(itc))$}{
			$res \gets Siguiente(itc)$ \com*{$\Theta(Copy(Siguiente(itc)))$}
		}
     	$Avanzar(itc)$ \com*{$\Theta(1)$}
	}
    \medskip
	\underline{Complejidad:} $O(\#(c)*(ed + copy(d_i)))$, donde ed es el costo de equal($res$,$d_i$) $d_i$ $\in$ $c$
\end{algoritmo} 

\begin{algoritmo}{imaximo}{\In {c}{conj($\alpha$)}}{$\alpha$}
	$itc \gets crearIt(c)$ \com*{$\Theta(1)$}
    $res \gets Siguiente(itc)$ \com*{$\Theta(1)$}
    \While(\com*[f]{$\Theta(1)$}){($HaySiguiente(itc)$)}{
    	\If(\com*[f]{$O(equal(res,Siguiente(itc)))$}){$(res \leq Siguiente(itc))$}{
			$res \gets Siguiente(itc)$ \com*{$\Theta(Copy(Siguiente(itc)))$}
		}
     	$Avanzar(itc)$ \com*{$\Theta(1)$}
	}
    \medskip
	\underline{Complejidad:} $O(\#(cd)*(ed + copy(d_i)))$, donde ed es el costo de equal($res$,$d_i$) $d_i$ $\in$ $c$
\end{algoritmo} 



\end{Algoritmos}



\section{M\'odulos Simples}


\subsection{nombreTabla}
  \servUsados{String}
  \Encabezado{Representaci\'on}
    \begin{Estructura}{nombreTabla}[String]
    \end{Estructura} 

\subsection{campo}
  \servUsados{String}
  \Encabezado{Representaci\'on}
    \begin{Estructura}{campo}[String]
    \end{Estructura} 
    
\subsection{itTablas}
  \servUsados{itConj}
  \Encabezado{Representaci\'on}
    \begin{Estructura}{itTablas}[itConj]
    \end{Estructura} 

\subsection{minMax}
  \servUsados{Tupla, dato}
  \Encabezado{Representaci\'on}
    \begin{Estructura}{minMax}[tupla(Min:dato, Max:Dato)]
    \end{Estructura} 
    
\subsection{ContenedorReg}
  \servUsados{Tupla, DiccStr, DiccNat, String, Nat, Conj, ItConj, reg }
  \Encabezado{Representaci\'on}
    \begin{Estructura}{ContenedorReg}[estrCont]
	  \begin{Tupla}[estrCont]
        \tupItem{IndiceString}{DiccStr<DatoString: String, ConjRegistros: Conj<Tupla(Reg: ItConj, OtroIndice: ItConj)> $>$}
        
        \tupItem{IndiceNat}{DiccNat<DatoNat: Nat, ConjRegistros: Conj<Tupla(Reg: ItConj, OtroIndice: ItConj)> $>$}
        
        \tupItem{Registros}{Conj<reg>}
      \end{Tupla}
    \end{Estructura} 

      
\subsection{Join}
  \servUsados{Tupla, campo, ContenedorReg, Conj, Modificacion}
  \Encabezado{Representaci\'on}
    \begin{Estructura}{Join}[estrJoin]
       \begin{Tupla}[estrJoin]
        \tupItem{Campo}{campo}
        \tupItem{Modificaciones}{Conj<Modificacion>}
        \tupItem{ConjJoin}{ContenedorReg}
      \end{Tupla}
    \end{Estructura}
    
\subsection{datosTabla}
  \servUsados{Tupla, Tabla, Join, DiccStr, nombreTablaf}
  \Encabezado{Representaci\'on}
    \begin{Estructura}{datosTabla}[estrDT]
         \begin{Tupla}[estrDT]
        \tupItem{Tabla}{Tabla}
        \tupItem{Joins}{DiccStr<tablaJoin: nombreTabla, Join: Join>}
      \end{Tupla}
    \end{Estructura}

\subsection{Modificacion}
  \servUsados{Tupla, Bool, Registro}
  \Encabezado{Representaci\'on}
    \begin{Estructura}{Modificacion}[estrMod]
         \begin{Tupla}[estrMod]
        \tupItem{Inserto?}{Bool}
        \tupItem{Reg}{reg}
      \end{Tupla}
    \end{Estructura} 
    

    

\section{M\'odulo Dato}


\Encabezado{Notas preliminares}
  En todos los casos, al indicar las complejidades de los algoritmos, las variables que se utilizan corresponden a:
  \vspace{-0.5em}\begin{itemize}
    \item $L$: M\'axima longitud de un valor STRING de un registro en la tabla pasada por par\'ametro.
  \end{itemize}

\servUsados{nat, string, bool}

\begin{Interfaz}
  

  \seExplicaCon{Dato}

  \generos{\tipo{dat}}
  
  \Encabezado{Operaciones basicas de Dato}

  \InterfazFuncion{datoStr}{\In {s} {string}}{dat}
  [true] % Pre
  {res $\igobs$ datoString(s)} % Pos
  [$\Theta(L)$] % Complejidad
  [Genera un dato string] % Descripción
  [...] % Aliasing

  ~

  \InterfazFuncion{datoN}{\In {n}{nat}}{dat}
  [true] % Pre
  {res $\igobs$ datoNat(n)} % Pos
  [$\Theta(1)$] % Complejidad
  [Genera un dato nat] % Descripción
  [...] % Aliasing

  ~
  
\InterfazFuncion{tipo?}{\In {d}{dat}}{bool}
  [true] % Pre
  {res $\igobs$ tipo?(d)} % Pos
  [$\Theta(1)$] % Complejidad
  [Devuelve true si el dato es nat, si es string devuelve false] % Descripción
  [...] % Aliasing

  ~

\InterfazFuncion{valorN}{\In {d}{dat}}{nat}
  [Nat?(d)] % Pre
  {res $\igobs$ valorNat(d)} % Pos
  [$\Theta(1)$] % Complejidad
  [Devuelve el valor de un dato nat] % Descripción
  [res es una referencia] % Aliasing

  ~
  
\InterfazFuncion{valorS}{\In {d}{dat}}{string}
  [String?(d)] % Pre
  {res $\igobs$ valorStr(d)} % Pos
  [$\Theta(1)$] % Complejidad
  [Devuelve el valor de un dato string] % Descripción
  [res es una referencia] % Aliasing
  
  ~
  
\newpage  

\InterfazFuncion{mismoTipo}{\In {d1}{dat}, \In {d2}{dat}}{bool}
  [true] % Pre
  {res $\igobs$ (tipo?(d1) == tipo?(d2))} % Pos
  [$\Theta(1)$] % Complejidad
  [Devuelve verdadero si coincide el tipo de los datos $d1$ y $d2$] % Descripción
  [...] % Aliasing
  
  ~
\InterfazFuncion{esString?}{\In {d}{dat}}{bool}
  [true] % Pre
  {res $\igobs$ String?(d)} % Pos
  [$\Theta(1)$] % Complejidad
  [Devuelve verdadero si $d$ es de tipo string] % Descripción
  [...] % Aliasing
  
  ~

\InterfazFuncion{esNat?}{\In {d}{dat}}{bool}
  [true] % Pre
  {res $\igobs$ Nat?(d)} % Pos
  [$\Theta(1)$] % Complejidad
  [Devuelve verdadero si $d$ es de tipo nat] % Descripción
  [...] % Aliasing
  ~

% \InterfazFuncion{minimo}{\In {cd}{conj(dat)}}{dat}
%   [true] % Pre
%   {res $\igobs$ min(cd)} % Pos
%   [$\Theta(1)$] % Complejidad
%   [Devuelve el minimo de un conjunto de datos] % Descripción
%   [...] % Aliasing
%   ~

% \InterfazFuncion{maximo}{\In {cd}{conj(dat)}}{dat}
%   [true] % Pre
%   {res $\igobs$ max(cd)} % Pos
%   [$\Theta(1)$] % Complejidad
%   [Devuelve el maximo de un conjunto de datos] % Descripción
%   [...] % Aliasing
%   ~

\InterfazFuncion{\argumento $\leq$ \argumento}{\In {d1}{dat}, \In {d2}{dat}}{bool}
[mismoTipo?(d1,d2)] % Pre
{res $\igobs$ (d1 $\leq$ d2)} % Pos
[$O(1 + comp(d1,d2))$] % Complejidad
[Devuelve verdadero si $d1$ es menor o igual a $d2$] % Descripción
[...] % Aliasing
  ~
  
  \InterfazFuncion{\argumento = \argumento}{\In {d1}{dat}, \In {d2}{dat}}{bool}
[true] % Pre
{res $\igobs$ (d1 = d2)} % Pos
[$O(1 + equal(d1,d2))$] % Complejidad
[Devuelve true si dos datos son iguales] % Descripción
[...] % Aliasing

%   \InterfazFuncion{copiar}{\In {d}{dat}}{dat}
% [true] % Pre
% {res $\igobs$ d} % Pos
% [$\Theta(1)$] % Complejidad
% [Retorna una copia de d ] % Descripción
% [...] % Aliasing

\end{Interfaz} 

\begin{Representacion}

  \begin{Estructura}{dat}[estrDato]

    \begin{Tupla}[estrDato]
      \tupItem{DatoString}{String}
      \tupItem{DatoNat}{Nat}
      \tupItem{EsNat?}{Bool}
    \end{Tupla}

  \end{Estructura}
  
%\textbf{Invariante de representaci\'on en castellano:}

 %   \begin{enumerate} 
  %    \item 
   % \end{enumerate}
  \RepFc[estrDato][e]{
    $\textbf{if}(e.esNat?)$ \emph{then} $e.DatoString \equiv$ $""$ $\textbf{else}$ $e.DatoNat \equiv 0$  
    }
    

  \AbsFc[estrDato]{Dato}[e]{d: Dato $/$ \\
     \begin{enumerate}
		\item $tipo?(d) \equiv e.esNat? \yluego$
        \item $tipo?(d) \impluego valorNat(d) \equiv e.DatoNat \land$ 
      	\item $\neg tipo?(d) \impluego valorStr(d) \equiv e.DatoString $ 
      \end{enumerate} 
  }
    
\end{Representacion}


\begin{Algoritmos}

\begin{algoritmo}{idatoStr}{\In {s}{string}}{estrDato}
	$res \gets \langle false,s,0 \rangle$ \com*{$O(3 + copy(s))$}
    \medskip
	\underline{Complejidad:} $O(copy(s))$
\end{algoritmo}    

\begin{algoritmo}{idatoN}{\In {n}{nat}}{estrDato}
	$res \gets \langle true,"",n \rangle$ \com*{$O(3 + copy(n))$}
    \medskip
	\underline{Complejidad:} $O(1)$
\end{algoritmo}    

\begin{algoritmo}{itipo?}{\In {d}{estrDato}}{bool}
	$res \gets \langle d.esNat? \rangle$ \com*{$\Theta(1)$}
    \medskip
	\underline{Complejidad:} $\Theta(1)$
\end{algoritmo}    

\begin{algoritmo}{ivalorN}{\In {d}{estrDato}}{nat}
	$res \gets d.DatoNat$ \com*{$\Theta(1)$}
    \medskip
	\underline{Complejidad:} $\Theta(1)$
\end{algoritmo}    

\begin{algoritmo}{ivalorS}{\In {d}{estrDato}}{string}
	$res \gets d.DatoString$ \com*{$\Theta(1)$}
    \medskip
	\underline{Complejidad:} $\Theta(1)$
\end{algoritmo}    

\begin{algoritmo}{imismoTipo}{\In {d1}{estrDato}, \In {d2}{estrDato}}{bool}
	$res \gets d1.esNat? = d2.esNat?$ \com*{$\Theta(1)$}
    \medskip
	\underline{Complejidad:} $\Theta(1)$
\end{algoritmo}    

\begin{algoritmo}{iesString?}{\In {d}{estrDato}}{bool}
	$res \gets not tipo?(d)$ \com*{$\Theta(1)$}
    \medskip
	\underline{Complejidad:} $\Theta(1)$
\end{algoritmo} 

\begin{algoritmo}{iesNat?}{\In {d}{estrDato}}{bool}
	$res \gets tipo?(d)$ \com*{$\Theta(1)$}
    \medskip
	\underline{Complejidad:} $\Theta(1)$
\end{algoritmo} 

% \begin{algoritmo}{iminimo}{\In {cd}{conj(dat)}}{dat}
% 	$itc \gets crearIt(cd)$ \com*{$\Theta(1)$}
%     $res \gets Siguiente(itc)$ \com*{$\Theta(1)$}
%     \While(\com*[f]{$\Theta(1)$}){($HaySiguiente(itc)$)}{
%     	\If(\com*[f]{$O(equal(res,Siguiente(itc)))$}){$\neg (res \leq Siguiente(itc))$}{
% 			$res \gets Siguiente(itc)$ \com*{$\Theta(Copy(Siguiente(itc)))$}
% 		}
%      	$Avanzar(itc)$ \com*{$\Theta(1)$}
% 	}
%     \medskip
% 	\underline{Complejidad:} $O(\#(cd)*(ed + copy(d_i)))$, donde ed es el costo de equal($res$,$d_i$) $d_i$ $\in$ $cd$
% \end{algoritmo} 

% \begin{algoritmo}{imaximo}{\In {cd}{conj(dat)}}{dat}
% 	$itc \gets crearIt(cd)$ \com*{$\Theta(1)$}
%     $res \gets Siguiente(itc)$ \com*{$\Theta(1)$}
%     \While(\com*[f]{$\Theta(1)$}){($HaySiguiente(itc)$)}{
%     	\If(\com*[f]{$O(equal(res,Siguiente(itc)))$}){$(res \leq Siguiente(itc))$}{
% 			$res \gets Siguiente(itc)$ \com*{$\Theta(Copy(Siguiente(itc)))$}
% 		}
%      	$Avanzar(itc)$ \com*{$\Theta(1)$}
% 	}
%     \medskip
% 	\underline{Complejidad:} $O(\#(cd)*(ed + copy(d_i)))$, donde ed es el costo de equal($res$,$d_i$) $d_i$ $\in$ $cd$
% \end{algoritmo} 

\begin{algoritmo}{i$\argumento = \argumento$}{\In {d1}{estrDato}, \In {d2}{estrDato}}{bool}
	$res \gets false $ \com*{$\Theta(1)$}
    \If(\com*[f]{$\theta(1)$}){$(esNat?(d1) \wedge esNat?(d2))$}{
			$res \gets valorN(d1) = valorN(d2) $ \com*{$\Theta(1)$}
		}
    \If(\com*[f]{$\theta(1)$}){$(esString?(d1) \wedge esString?(d2))$}{
			$res \gets valorS(d1) = valorS(d2) $ \com*{$O(equal(s_1,s_2))$}
		} 
    \medskip
	\underline{Complejidad:} $O(1 + equal(s_1,s_2))$, donde $s_i$ es el valorS(di)
\end{algoritmo}    

\begin{algoritmo}{i$\argumento \leq \argumento$}{\In {d1}{estrDato}, \In {d2}{estrDato}}{bool}
	$res \gets false $ \com*{$\Theta(1)$}
    \If(\com*[f]{$\theta(1)$}){$(esNat?(d1) \wedge esNat?(d2))$}{
			$res \gets valorN(d1) \leq valorN(d2) $ \com*{$\Theta(1)$}
		}
    \If(\com*[f]{$\theta(1)$}){$(esString?(d1) \wedge esString?(d2))$}{
			$res \gets valorS(d1) \leq valorS(d2) $ \com*{$O(comp(s_1,s_2))$}
		} 
    \medskip
	\underline{Complejidad:} $O(1 + comp(s_1,s_2))$, donde $s_i$ es el valorS(di)
\end{algoritmo}    

\end{Algoritmos}

\section{Modulo Registro}

\begin{Interfaz}
  
	\seExplicaCon{Registro}

	\generos{\tipo{reg}}
  
	\servUsados{string, dato, campo, Dicc, itDicc, conj($\alpha$)}	

	\tadExtiende{{Dicc(campo, dato)}}
    
    \tadTitulo{otras operaciones (exportadas)}
    
    \tadAlinearFunciones{coincidenTodosAux}{campo \ $c$, {{reg}} \ $d$, conj({{reg}}) \ $cd$}

    \tadOperacion{campos}{registro}{conj(campo)}{}

    \tadOperacion{borrar?}{registro /$crit$, registro}{bool}{$\#$campos($crit$) $\equiv$ 1}

    \tadOperacion{agregarCampos}{registro \ $r_1$, registro \ $r_2$}{registro}{}

    \tadOperacion{copiarCampos}{conj(campo) \ $cc$, registro \ $r_1$ , registro \ $r_2$}{registro}{$cc$ $\in$ campos($r_2$)}

    \tadOperacion{coincideAlguno}{registro \ $r_1$, conj(campo) \ $cc$, registro \ $r_2$}{bool}{$cc$ $\subseteq$ campos($r_1$) $\cap$ campos($r_2$)}

    \tadOperacion{coincidenTodos}{registro \ $r_1$, conj(campo) \ $cc$, registro \ $r_2$}{bool}{$cc$ $\subseteq$ campos($r_1$) $\cap$ campos($r_2$)}

    \tadOperacion{coincidenTodosAux}{registro \ $r_1$, conj(campo) \ $cc$, registro \ $r_2$}{bool}{$cc$ $\subseteq$ campos($r_1$)}

    \tadOperacion{enTodos}{campo \ $c$, conj(registro) \ $cr$}{bool}{}

    \tadOperacion{combinarTodos}{campo \ $c$, registro \ $r$, conj(registro) \ $cr$}{conj(registro)}{$c$ $\in$ campos($r_1$) $\land$ enTodos($c$,$cd$)}

    \tadAxiomas[\paratodo{registro}{r, r_1, r_2, crit}, \paratodo{conj(reg)}{cr}, \paratodo{conj(campo)}{cc}, \paratodo{campo}{c} ]

    \tadAlinearAxiomas{coicidenTodosAux($r_1$, $cc$, $r_2$)cccc}

    \tadAxioma{campos($r$)}{claves($r$)}

    \tadAxioma{borrar?($crit$, $r$)}{coincidenTodos($crit$, campos($crit$), $r$)}

    \tadAxioma{agregarCampos($r_1$, $r_2$)}{copiarCampos(campos($r_2$) - campos($r_1$), $r_1$, $r_2$)}

    \tadAxioma{copiarCampos($cc$, $r_1$, $r_2$)}{
    $\textbf{if}$ ($\emptyset$?($cc$)) $\emph{then}$ \\
        $\hspace*{10px}$ $r_1$ \\
    $\textbf{else}$ \\
        $\hspace*{10px}$ copiarCampos(sinUno($cc$), definir(dameUno($cc$), \\
        $\hspace*{10px}$ obtener(dameUno($cc$), $r_2$),$r_1$), $r_2$ ) \\
    $\textbf{fi}$}

    \tadAxioma{coincideAlguno($r_1$, $cc$, $r_2$)}{$\neg \emptyset$?($cc$) $\yluego$ ((obtener(dameUno($cc$),$r_1$) $=$ obtener(dameUno($cc$), $r_2$)) $\lor$ \\ coincideAlguno($r_1$, sinUno($cc$), $r_2$))} 

    \tadAxioma{coincidenTodos($r_1$, $cc$, $r_2$)}{$\emptyset$?($cc$) $\oluego$ ((obtener(dameUno($cc$),$r_1$) $=$ obtener(dameUno($cc$), $r_2$)) $\land$ \\ coincidenTodos($r_1$, sinUno($cc$), $r_2$))}


    \tadAxioma{coincidenTodosAux($r_1$, $cc$, $r_2$)}{$\emptyset$?($cc$) $\oluego$ def?(dameUno($cc$),$r_2$) $\yluego$ ((obtener(dameUno($cc$),$r_1$) $=$ obtener(dameUno($cc$), $r_2$)) $\land$ \\ coincidenTodos($r_1$, sinUno($cc$), $r_2$))}

\newpage

    \tadAxioma{enTodos($c$, $cr$)}{$\emptyset$?($cr$) $\oluego$ ($c$ $\in$ campos(dameUno($cr$)) $\land$ enTodos($c$, sinUno($cr$)))}

    \tadAxioma{combinarTodos($c$, $r$, $cr$)}{
    $\textbf{if}$ ($\emptyset$?($cr$)) $\emph{then}$ \\
        $\hspace*{10px}$ $\emptyset$ \\
    $\textbf{else}$ \\
        $\hspace*{10px}$ combinarTodos($c$, $r$, sinUno($cr$)) $\cup$ $\textbf{if}$ (obtener($c$, dameUno($cr$)) $=$ \\
        $\hspace*{10px}$ obtener($c$, $r$)) $\textbf{\emph{then}}$ 	$\{$agregarCampos($r$, dameUno($cr$)) $\}$ $\textbf{else}$ $\emptyset$ $\textbf{fi}$ \\
    $\textbf{fi}$}
    
\Encabezado{Funciones Extendidas}   
  
  \InterfazFuncion{Campos}{\In{r}{reg}}{conj(campo)}
  [true] % Pre
  {$res$ $\igobs$ campos($r$)} % Pos
  [$O(\#Claves(r)*copy(campo))$] % Complejidad
  [Devuelve el conjunto de campos del registro] % Descripción
  [] % Aliasing
  
  \InterfazFuncion{Borrar?}{\In{crit}{reg}, \In{r}{reg}}{bool}
  [$\#$ campos($crit$) $\equiv$ 1] % Pre
  {$res$ $\igobs$ borrar?($crit$,$r$)} % Pos
  [$O(\sum_{k' \in K}equal(c,k') + equal(g_{crit},g_r))$] % Complejidad
  [Devuelve verdadero si $r$ es Borrable segun el criterio crit(que matchee el significado en $r$ y $crit$ para el campo(unico) de $crit$ ] % Descripción
  [] % Aliasing
  
   \InterfazFuncion{AgregarCampos}{\In{r_1}{reg}, \In{r_2}{reg}}{reg}
  [true] % Pre
  {$res$ $\igobs$ agregarCampos($r_1$,$r_2$)} % Pos
  [$O(\#cc*\sum_{k' \in K}(equal(k,k') + copy(sc) + copy(\text{significado}(sc,r_1)))+(Copy(l) + Copy(g)))$, donde K $=$ Claves($r_1$), $sc$ $\in$ Claves($res$), $l$ $\in$ Claves($r_2$) y $g$ $=$ Significado($r_2$,$l$)  ] % Complejidad
  [devuelve un registro con los campos de $r_1$ y su respectivo dato, mas los campos que posee $r_2$ y no $r_1$, con su respectivo significado] % Descripción
  [] % Aliasing
  
   \InterfazFuncion{CopiarCampos}{\In{cc}{conj(campo)}, \In{r_1}{reg}, \In{r_2}{reg}}{reg}
  [$cc$ $\subseteq$ campos($r_2$)] % Pre
  {$res$ $\igobs$ copiarCampos($cc$,$r_1$,$r_2$)} % Pos
  [$O(\#Claves(r_2)*\sum_{k' \in K}(equal(k,k') + copy(k) + copy(\text{significado}(k,r_1)))+(Copy(l) + Copy(g)))$, donde K $=$ Claves($r_1$), $l$ $\in$ Claves($r_2$) y $g$ $=$ Significado($r_2$,$l$)  ] % Complejidad
  [Devuelve un regsitro con los campos de $r_1$ y su respectivo significado, mas los campos pertenecientes a $cc$ con el significado que tienen en $r_2$] % Descripción
  [] % Aliasing
  
   \InterfazFuncion{CoincideAlguno}{\In{r_1}{reg}, \In{cc}{conj(campo)}, \In{r_2}{reg}}{bool}
  [$cc$ $\subseteq$ (campos($r_1$) $\cap$ campos($r_2$))] % Pre
  {$res$ $\igobs$ coincideAlguno($r_1$,$cc$,$r_2$)} % Pos
  [$O(\#cc*(Sig_1 + Sig_2 + equal(g_{r_1},g_{r_2})))$, donde $Sig_i$ es $\sum_{k' \in K}(equal(k,k')$ K $=$ Claves($r_i$) y $g_i$ $=$ Significado($i$,$c$), $c \in cc$  ] % Complejidad
  [Devuelve Verdadero si algun campo perteneciente a $cc$ tiene el mismo significado en $r_1$ y en $r_2$ ] % Descripción
  [] % Aliasing
  
  \InterfazFuncion{CoincidenTodos}{\In{r_1}{reg}, \In{cc}{conj(campo)}, \In{r_2}{reg}}{bool}
  [$cc$ $\subseteq$ (campos($r_1$) $\cap$ campos($r_2$))] % Pre
  {$res$ $\igobs$ coincidenTodos($r_1$,$cc$,$r_2$)} % Pos
  [$O(\#cc*(Sig_1 + Sig_2 + equal(g_{r_1},g_{r_2})))$, donde $Sig_i$ es $\sum_{k' \in K}(equal(c,k')$ K $=$ Claves($r_i$) y $g_i$ $=$ Significado($i$,$c$), $c \in cc$ ] % Complejidad
  [Devuelve Verdadero si para todo campo perteneciente a $cc$ tiene el mismo significado en $r_1$ y en $r_2$ ] % Descripción
  [] % Aliasing

 \InterfazFuncion{CoincidenTodosAux}{\In{r_1}{reg}, \In{cc}{conj(campo)}, \In{r_2}{reg}}{bool}
  [$cc$ $\subseteq$ campos($r_1$)] % Pre
  {$res$ $\igobs$ coincidenTodosAux($r_1$,$cc$,$r_2$)} % Pos
  [$O(\#cc*(Sig_1 + def_2 + Sig_2 + equal(g_{r_1},g_{r_2})))$, donde $Sig_i$ es $\sum_{k' \in K}(equal(c,k')$ K $=$ Claves($r_i$), $def_2$ $=$ $Sig_2$ y $g_i$ $=$ Significado($i$,$c$), $c \in cc$ ] % Complejidad
  [Devuelve Verdadero si para todo campo perteneciente a $cc$ tiene el mismo significado en $r_1$ y en $r_2$ ] % Descripción
  [] % Aliasing

  \InterfazFuncion{enTodos}{\In{c}{$campo$}, \In{cr}{conj(reg)}}{bool}
  [true] % Pre
  {$res$ $\igobs$ enTodos($c$,$cr$)} % Pos
  [$O(\#cr*\sum_{k' \in K}(equal(c,k')))$, K $=$ Claves($r$) $r$ $\in$ $cr$ ] % Complejidad
  [Devuelve Verdadero si $c$ es campo de cada registro de $cr$ ] % Descripción
  [] % Aliasing

  \InterfazFuncion{CombinarTodos}{\In{c}{campo}, \In{r}{reg}, \In{cr}{conj(reg)}}{conj(reg)}
  [$c$ $\in$ campos($r_1$) $\land$ enTodos($c$,$cr$)] % Pre
  {$res$ $\igobs$ combinarTodos($c$,$r$,$cr$)} % Pos
  [$O(Sig_d + \#cr*(Sig_r + equal(sc,g_{r}) + Ag))$] % Complejidad
  [Devuelve un Conjunto que posee por elementos a registros con los campos $r$ y su respectivo significado, mas los campos que posee $r_i$ y no $r$, tambien con el significado que poseia en $r_i$, donde $r_i$ son los diccionarios pertenecientes a $cr$ tales que coicidan en su significado con $r$ para el campo $c$] % Descripción
  [] % Aliasing

\InterfazFuncion{Sub?}{\In{r_1}{reg}, \In{r_2}{reg}}{bool}
  [true] % Pre
  {$res$ $\igobs$ coincidenTodos($r_1$,campos($r_1$),$r_2$)} % Pos
  [$O(\#claves(r_1)*(def_1 + Sig_2 + equal(g_{r_1},g_{r_2})))$, donde $def_1$ es $\sum_{t' \in T}(equal(c,t')$ T $=$ claves($r_1$),$Sig_2$ es $\sum_{k' \in K}(equal(c,k')$ K $=$ claves($r_2$) y $g_i$ $=$ Significado($i$,$c$), $c$ $\in$ claves($r_1$)] % Complejidad
  [Devuelve Verdadero si para todo campo perteneciente a $claves(r_1)$,existe esta en $r_2$  y tiene el mismo significado que en $d_1$] % Descripción
  [] % Aliasing

\end{Interfaz}

\begin{Algoritmos}

$\textbf{NOTA:}$ $\emph{Terminamos poniendo reg en cada algoritmo, porque no necesitabamos acceder a la representacion interna,}$
$\hspace*{30px}$
$\emph{si bien teniamos la duda que te expresamos en el mail, ya que la extension de modulos no esta documentada,}$
$\hspace*{30px}$
$\emph{creimos que era mas correcto esto que quizas acceder a una estructura que la catedra concidera que es ajena.}$
$\\$
\begin{algoritmo}{iCampos}{\In{d}{reg}}{conj(campo)}
	$it \gets CrearIt(d) $ \com*{$\Theta(1)$}
    $res \gets Vacio() $ \com*{$\Theta(1)$}
    \While(\com*[f]{$\Theta(1)$}){($HaySiguiente(it)$)}{
     	$AgregarRapido(res, SiguienteClave(it))$ \com*{$\Theta(copy(sigCl(it)))$}
		$Avanzar(it)$ \com*{$\Theta(1)$}
	}
    \medskip
	\underline{Complejidad:} $O(\#Claves(d)*copy(campo))$
    \end{algoritmo}

\begin{algoritmo}{iBorrar?}{\In{crit}{reg}, \Inout{d}{reg}}{bool}
	$itcrit \gets CrearIt(crit) $ \com*{$\Theta(1)$}
    $c \gets SiguienteClave(itcrit) $ \com*{$\Theta(1)$}
    $res \gets Significado(crit,c) = Significado(d,c)$ \com*{$\Theta(\sum_{k' \in K}equal(c,k') + equal(g_{crit},g_d))$}
	\medskip
	\underline{Complejidad:} $O(\sum_{k' \in K}equal(c,k') + equal(g_{crit},g_d))$, donde $K$ $=$ claves($d$), y $g_i$ $=$ Significado($i$,$c$)  
\end{algoritmo}

\begin{algoritmo}{iAgregarCampos}{\In{d_1}{reg}, \Inout{d_2}{reg}}{reg}
    % lo hago asi porque no tengo la operacion resta de conjuntos, ni se su complejidad
    $it2 \gets CrearIt(d_2) $ \com*{$\Theta(1)$}
    $res \gets Copiar(d_1) 	$ \com*{$\Theta(\sum_{k \in K}(copy(k) + copy(\text{significado}(k,d_1)))$}
    \While(\com*[f]{$\Theta(1)$}){($HaySiguiente(it2)$)}{
    	\If(\com*[f]{$\Theta(\sum_{k' \in K}equal(SigCl(it2),k'))$}){$\lnot Definido?(res,SiguienteClave(it2))$}{
			$DefinirRapido(res, SiguienteClave(it2), SiguienteSignificado(it2))$ \com*{$\Theta(copy(k) + copy(s))$}
		}
     	$Avanzar(it2)$ \com*{$\Theta(1)$}
	}
    \medskip
	\underline{Complejidad:} $O(\#Claves(d_2)*\sum_{k' \in K}(equal(k,k') + copy(k) + copy(\text{significado}(k,d_1)))+(Copy(l) + Copy(g)))$, donde K $=$ Claves($d_1$), $l$ $\in$ Claves($d_2$) y $g$ $=$ Significado($d_2$,$l$) 
\end{algoritmo}

\begin{algoritmo}{iCopiarCampos}{\In{cc}{conj(campo)}, \In{d_1}{reg}, \In{d_2}{reg}}{reg}
    $itc \gets CrearIt(cc) $ \com*{$\Theta(1)$}
    $res \gets Copiar(d_1) 	$ \com*{$\Theta(\sum_{k \in K}(copy(k) + copy(\text{significado}(k,d_1)))$}
    \While(\com*[f]{$\Theta(1)$}){($HaySiguiente(itc)$)}{
     	\If(\com*[f]{$\Theta(\sum_{k' \in K}equal(sigCl(itc),k'))$}){$\lnot Definido?(res,SiguienteClave(itc))$}{
			$DefinirRapido(res, SiguienteClave(itc), SiguienteSignificado(itc))$ \com*{$\Theta(copy(sc) + copy(s))$}
		}
     	$Avanzar(it2)$ \com*{$\Theta(1)$}
	}
    \medskip
	\underline{Complejidad:} $O(\#cc*\sum_{k' \in K}(equal(k,k') + copy(sc) + copy(\text{significado}(sc,d_1)))+(Copy(l) + Copy(g)))$, donde K $=$ Claves($d_1$), $sc$ $\in$ Claves($res$), $l$ $\in$ Claves($d_2$) y $g$ $=$ Significado($d_2$,$l$) 
\end{algoritmo}

\begin{algoritmo}{iCoincideAlguno}{\In{d_1}{reg}, \In{cc}{conj(campo)}, \In{d_2}{reg}}{bool}
    $itc \gets CrearIt(cc) $ \com*{$\Theta(1)$}
    $res \gets false $ \com*{$\Theta(1)$}
    \While(\com*[f]{$\Theta(1)$}){($HaySiguiente(itc) \land \neg res$)}{
    	$res \gets Significado(d_1,Siguiente(itc)) = Significado(d_2,Siguiente(itc))$ \com*{$\Theta(Sig_1 + Sig_2 + equal(g_{d_1},g_{d_2}))$}
	$Avanzar(itc)$ \com*{$\Theta(1)$}
    }
    \medskip
	\underline{Complejidad:} $O(\#cc*(Sig_1 + Sig_2 + equal(g_{d_1},g_{d_2})))$, donde $Sig_i$ es $\sum_{k' \in K}(equal(k,k')$ K $=$ Claves($d_i$) y $g_i$ $=$ Significado($i$,$c$), $c \in cc$ 
\end{algoritmo}

\begin{algoritmo}{iCoincidenTodos}{\In{d_1}{reg}, \In{cc}{conj(campo)}, \In{d_2}{reg}}{bool}
    $itc \gets CrearIt(cc) $ \com*{$\Theta(1)$}
    $res \gets true $ \com*{$\Theta(1)$}
    \While(\com*[f]{$\Theta(1)$}){($HaySiguiente(itc) \wedge res$)}{
    	$res \gets Significado(d_1,Siguiente(itc)) = Significado(d_2,Siguiente(itc))$ \com*{$\Theta(Sig_1 + Sig_2 + 		equal(g_{d_1},g_{d_2}))$}
		$Avanzar(itc)$ \com*{$\Theta(1)$}
	}
    \medskip
	\underline{Complejidad:} $O(\#cc*(Sig_1 + Sig_2 + equal(g_{d_1},g_{d_2})))$, donde $Sig_i$ es $\sum_{k' \in K}(equal(c,k')$ K $=$ Claves($d_i$) y $g_i$ $=$ Significado($i$,$c$), $c \in cc$ 
\end{algoritmo}

\begin{algoritmo}{iCoincidenTodosAux}{\In{d_1}{reg}, \In{cc}{conj(campo)}, \In{d_2}{reg}}{bool}
    $itc \gets CrearIt(cc) $ \com*{$\Theta(1)$}
    $res \gets true $ \com*{$\Theta(1)$}
    \While(\com*[f]{$\Theta(1)$}){($HaySiguiente(itc) \wedge res$)}{
    	\eIf(\com*[f]{$\Theta(\sum_{k' \in claves(d_2)}equal(sigCl(itc),k'))$}){$Definido?(d_2,SiguienteClave(itc))$}{
		$res \gets Significado(d_1,Siguiente(itc)) = Significado(d_2,Siguiente(itc))$ \com*{$\Theta(Sig_1 + Sig_2 + 		equal(g_{d_1},g_{d_2}))$}
		}{
        $res \gets false $ \com*{$\Theta(1)$}
        }
    	$Avanzar(itc)$ \com*{$\Theta(1)$}
	}
    \medskip
	\underline{Complejidad:} $O(\#cc*(Sig_1 + def_2 + Sig_2 + equal(g_{d_1},g_{d_2})))$, donde $Sig_i$ es $\sum_{k' \in K}(equal(c,k')$ K $=$ Claves($d_i$), $def_2$ $=$ $Sig_2$ y $g_i$ $=$ Significado($i$,$c$), $c \in cc$ 
\end{algoritmo}

\begin{algoritmo}{ienTodos}{\In{c}{$campo$}, \In{cr}{conj(reg)}}{bool}
    $itcr \gets CrearIt(cr) $ \com*{$\Theta(1)$}
    $res \gets true $ \com*{$\Theta(1)$}
    \While(\com*[f]{$\Theta(1)$}){($HaySiguiente(itcr) \wedge res$)}{
    	$res \gets Definido?(Siguiente(itcr), c)$\com*{$\Theta(\sum_{k' \in K}(equal(c,k')))$}
		$Avanzar(itc)$ \com*{$\Theta(1)$}
	}
    \medskip
	\underline{Complejidad:} $O(\#cr*\sum_{k' \in K}(equal(c,k')))$, K $=$ Claves($r$) $r$ $\in$ $cr$
\end{algoritmo}

\begin{algoritmo}{iCombinarTodos}{\In{c}{$campo$}, \In{d}{reg}, \In{cr}{conj(reg)}}{conj(reg)}
    $itcr \gets CrearIt(cr) $ \com*{$\Theta(1)$}
    $res \gets Vacio() $ \com*{$\Theta(1)$}
    $sc  \gets Significado(d,c)$ \com*{$O(Sig_d)$} 
   \While(\com*[f]{$\Theta(\#cr)$}){($HaySiguiente(itcr)$)}{
    	\If(\com*[f]{$O(Sig_{Sig(itcr)} + equal(sc,g_{r}))$}){($sc = Significado(Siguiente(itcr),c)$)}{
			$Agregar(res, AgregarCampos(d, Siguiente(itcr)))$ \com*{$\Theta(\sum_{d' \in res}(equal(d,d')))$}
		}
     	$Avanzar(itcr)$ \com*{$\Theta(1)$}	
   }
    \medskip
	\underline{Complejidad:} $O(Sig_d + \#cr*(Sig_r + equal(sc,g_{r}) + Ag))$, donde $Sig_i$ es $\sum_{k' \in K}(equal(c,k')$, $r$ $\in$ $cr$, K $=$ Claves($i$), $g_i$ $=$ Significado($i$,$c$), $d$ $=$ AgregarCampos($d$,$r$) y $Ag$ es $\sum_{d' \in res}(equal(d,d'))$
\end{algoritmo}

\begin{algoritmo}{iSub?}{\In{d_1}{reg}, \In{d_2}{reg}}{bool}
    $itd1 \gets CrearIt(d_1) $ \com*{$\Theta(1)$}
    $res \gets true $ \com*{$\Theta(1)$}
    \While(\com*[f]{$\Theta(1)$}){($HaySiguiente(itd1) \land res$)}{
    	\eIf(\com*[f]{$\Theta(\sum_{k \in K}equal(SigCl(itd1),k'))$}){$\lnot Definido?(d_2,SiguienteClave(itd1))$}{
			$res \gets SiguienteSignificado(itd1) = Significado(d_2,SiguienteClave(itd1))$ \com*{$\Theta(Sig_2 + 		equal(SigSdo(itd1),g_{d_2}))$}
		}{
        	$res \gets false $ \com*{$\Theta(1)$}
        }
		$Avanzar(itd1)$ \com*{$\Theta(1)$}
	}
    \medskip
	\underline{Complejidad:} $O(\#claves(d_1)*(def_1 + Sig_2 + equal(g_{d_1},g_{d_2})))$, donde $def_1$ es $\sum_{t' \in T}(equal(c,t')$ T $=$ claves($d_1$),$Sig_2$ es $\sum_{k' \in K}(equal(c,k')$ K $=$ claves($d_2$) y $g_i$ $=$ Significado($i$,$c$), $c$ $\in$ claves($d_1$)
\end{algoritmo}

\end{Algoritmos}


\section{M\'odulo Tabla}

\Encabezado{Notas preliminares}
  En todos los casos, al indicar las complejidades de los algoritmos, las variables que se utilizan corresponden a:
  \vspace{-0.5em}\begin{itemize}
    \item $n$: N\'umero de registros en la tabla pasada por par\'ametro.
    \item $m$: N\'umero de registros en la tabla pasada por par\'ametro.
    \item $L$: M\'axima longitud de un valor STRING de un registro en la tabla pasada por par\'ametro.
  \end{itemize}

\begin{Interfaz}
  

  \seExplicaCon{Tabla}

  \generos{\tipo{tbl}}
  
    \servUsados{nombreTabla, campo, Bool, Conj, itConj, String, Nat, reg, Tupla, dat, ContenedorReg, itTablas, minMax}
  
  \Encabezado{Operaciones de Tabla}

   \InterfazFuncion{nuevaTabla}{\In{nombre}{string}, \In{claves}{conj(campo)}, \In{columnas}{reg}}{tbl}
  [$claves \neq \emptyset \land claves  \subseteq campos(columnas)$] % Pre
  {$res$ $\igobs$ nuevaTabla()} % Pos
  [$\Theta(1)$ ] % Complejidad
  [Crea una tabla Vacia] % Descripción
  [] % Aliasing
  
  \InterfazFuncion{agregarRegistro}{\In{r}{reg}, \Inout{t}{tbl}}{}
  [$t \igobs t_0 \land campos(r) \igobs campos(t) \wedge puedoInsertar?(r,t))$] % Pre
  {$t \igobs agregarRegistro(r,t_0)$} % Pos
  [$O(L + log(n))$ ] % Complejidad
  [Agrega a la tabla el Registro $r$] % Descripción
  [] % Aliasing
  
  \InterfazFuncion{borrarRegistro}{\In{crit}{reg}, \Inout{t}{tbl}}{}
  [$t \igobs t_0 \land \#campos(crit) = 1 campos(t) \yluego dameUno(campos(crit)) \in claves(t)$] % Pre
  {$t \igobs borrarRegistro(r,t_0)$} % Pos
  [$O(n*(n+L))$ ] % Complejidad
  [Borra un Registro de la tabla, segun el Criterio $crit$] % Descripción
  [] % Aliasing
  
  \InterfazFuncion{Indexar}{\In{c}{campo}, \Inout{t}{tbl}}{}
  [$t \igobs t_0 \land puedeIndexar?(c,t))$] % Pre
  {$t \igobs indexar(c,t_0)$} % Pos
  [$O(n * (n * lg(n) + L))$ ] % Complejidad
  [Indexa la tabla por el campo $c$] % Descripción
  [] % Aliasing
  
  \InterfazFuncion{Nombre}{\In{t}{tbl}}{string}
  [true] % Pre
  {$res \igobs nombre(t)$} % Pos
  [$\Theta(1)$ ] % Complejidad
  [Retorna el nombre de la tabla] % Descripción
  [] % Aliasing
  
  \InterfazFuncion{Claves}{\In{t}{tbl}}{conj(campo)}
  [true] % Pre
  {$res \igobs claves(t)$} % Pos
  [$\Theta(1)$ ] % Complejidad
  [Retorna un conjunto con las claves de la tabla] % Descripción
  [] % Aliasing
  
\newpage
  
  \InterfazFuncion{Indices}{\In{t}{tbl}}{conj(campo)}
  [true] % Pre
  {$res \igobs indices(t)$} % Pos
  [$\Theta(1)$ ] % Complejidad
  [Retorna un conjunto con los indices de la tabla] % Descripción
  [] % Aliasing
  
  \InterfazFuncion{Campos}{\In{t}{tbl}}{conj(campo)}
  [true] % Pre
  {$res \igobs campos(t)$} % Pos
  [$\Theta(1)$ ] % Complejidad
  [Retorna un conjunto con los campos de la tabla] % Descripción
  [] % Aliasing

  \InterfazFuncion{esNat?}{\In{c}{campo}, \In{t}{tbl}}{bool}
  [$c$ $\in$ campos($t$)] % Pre
  {$res$ $\igobs$ tipoCampo($c$,$t$)} % Pos
  [$\Theta(1)$ ] % Complejidad
  [Devuelve verdadero si el campo $c$ es de tipo nat] % Descripción
  [] % Aliasing
  
  \InterfazFuncion{Registros}{\In{t}{tbl}}{conj(reg)}
  [true] % Pre
  {$res \igobs registros(t)$} % Pos
  [$\Theta(1)$ ] % Complejidad
  [Retorna un conjunto con los registros de la tabla. ] % Descripción
  [Se devuelve una referencia no modificable] % Aliasing
  
   \InterfazFuncion{CantidadDeAccesos}{\In{t}{tbl}}{nat}
  [true] % Pre
  {$res \igobs cantidadDeAccesos(t)$} % Pos
  [$\Theta(1)$ ] % Complejidad
  [Retorna los accesos de la tabla] % Descripción
  [] % Aliasing
  
   \InterfazFuncion{PuedoInsertar?}{\In{r}{reg}, \In{t}{tbl}}{bool}
   [true] % Pre
   {$res \igobs puedoInsertar?(r,t)$} % Pos
   [$O(n*L)$ ] % Complejidad
   [Devuelve verdadero si el registro $r$ es insertable] % Descripción
   [] % Aliasing
  
   \InterfazFuncion{Compatible}{\In{r}{reg}, \In{t}{tbl}}{bool}
   [true] % Pre
   {$res \igobs compatible(r,t)$} % Pos
   [$\Theta(1)$ ] % Complejidad
   [Devuelve verdadero si el registro $r$ tiene los mismo campos y tipos que la tabla] % Descripción
   [] % Aliasing
   
   \InterfazFuncion{hayCoincidencia}{\In{r}{reg}, \In{cc}{conj(campo)}, \In{cr}{conj(reg)}}{bool}
   [true] % Pre
   {$res \igobs hayCoincidencia(r,cc,cr)$} % Pos
   [$O(Cardinal(cr)*Cardinal(cc)*L)$ ] % Complejidad
   [Devuelve verdadero si en $r$ y un registro de $cr$ coincide en campo y dato para algun campo de $cc$ ] % Descripción
   [] % Aliasing

   \InterfazFuncion{MismosTipos}{\In{r}{reg}, \In{t}{tbl}}{bool}
   [$ campos(r) \subseteq campos(t) $] % Pre
   {$res \igobs mismosTipos(r,t)$} % Pos
   [$O((\#Campos(r) + \#Campos(t)) * L$ ] % Complejidad
   [Devuelve verdadero si el registro tienen los mismos campos y tipo para cada uno de ellos] % Descripción
   [] % Aliasing
   
   \InterfazFuncion{minimo}{\In{c}{campo}, \In{t}{tbl}}{dat}
   [$\neg \emptyset ?(registros(t)) \land c \in indices(t)$] % Pre
   {$res \igobs minimo(r,t)$} % Pos
   [$O(1)$ ] % Complejidad
   [Devuelve el minimo  de los datos correspondientes al campo $c$] % Descripción
   [res es una referencia] % Aliasing
  
   \InterfazFuncion{maximo}{\In{c}{campo}, \In{t}{tbl}}{dat}
   [$\neg \emptyset ?(registros(t)) \land c \in indices(t)$] % Pre
   {$res \igobs maximo(r,t)$} % Pos
   [$O(1)$ ] % Complejidad
   [Devuelve el maximo  de los datos correspondientes al campo $c$] % Descripción
   [res es una referencia] % Aliasing
  
   \InterfazFuncion{puedoIndexar}{\In{c}{campo}, \In{t}{tbl}}{bool}
   [true] % Pre
   {$res \igobs puedoIndexar(c,t)$} % Pos
   [$\Theta(L)$ ] % Complejidad
   [Devuelve verdadero si se puede indexar la tabla por el campo $c$] % Descripción
   [] % Aliasing
  
   \InterfazFuncion{coincidencias}{\In{r}{reg}, \In{cr}{conj(reg)}}{conj(reg)}
   [true] % Pre
   {$res \igobs coincidencias(r,cr)$} % Pos
   [$O(\#cr * L)$ ] % Complejidad
   [Devuelve un conjunto con los registros de $cr$ que coincidan en campo y dato para cada campo de $r$] % Descripción
   [] % Aliasing
  
   \InterfazFuncion{coincidenciasRap}{\In{r}{reg}, \In{t}{tbl}}{conj(reg)}
   [true] % Pre
   {$res \igobs coincidencias(r,registros(t))$} % Pos
   [$O(\#cr * L)$ en peor caso, si algun campo del registro es un campo indice y clave en la tabla la complejidad es $O(log(n) + L)$ ] % Complejidad
   [Devuelve un conjunto con los registros de $t$ que coincidan en campo y dato para cada campo de $r$ utilizando los campos indexados de t (en caso de existir) para poder encontrar las coincidencias en menor tiempo] % Descripción
   [] % Aliasing
  
   \InterfazFuncion{combinarRegistros}{\In{c}{campo}, \In{cr1}{conj(reg)}, \In{cr2}{conj(reg)}}{conj(reg)}
   [true] % Pre
   {$res \igobs combinarRegistros(c,cr1,cr2)$} % Pos
   [$O(n * m * L * min\{n,m\})$ ] % Complejidad
   [Devuelve un conjunto el cual es el resultado de la UNION de aplicar combinarTodos($c$,$cr1_i$,$cr2$) para todo $cr1_i$ registro de $cr1$ ] % Descripción
   [] % Aliasing
   
   \InterfazFuncion{combinarRegistrosRap}{\In{c}{campo}, \In{cr1}{conj(reg)}, \In{t}{tbl}}{conj(reg)}
   [true] % Pre
   {$res \igobs combinarRegistros(c,cr1,registros(t))$} % Pos
   [S\'i c es \'indice y clave, entonces $O(n*((log(m) + L)))$. Sino, en peor caso, $O(n*m*L*min\lbrace m, n \rbrace)$] % Complejidad
   [Este metodo esta pensado para utilizar cuando los conjuntos de registros que se reciben son los registros de una tabla. Y, estos, en el caso de que algun campo sea indice, permiten buscar algun registro, dentro del conjunto, mas rapido. Por eso se recibe un conjunto arbitrario y una tabla. Esto nos permite buscar, para cada registro del conjunto $cr1$, en los registros de la tabla en un tiempo igual o menor a que si fuera un conjunto arbitrario, ya que usamos los indices de la tabla, si existen. Y, si ademas, el campo es clave de la tabla, entonces podemos asegurar que la complejidad del algoritmo es todavia mejor, ya que sabemos que para cada registro de $cr1$ solo va a existir un \'unico registro en los registros de la tabla. ]% Descripción
   [] % Aliasing
  
   \InterfazFuncion{dameColumna}{\In{c}{campo}, \In{cr}{conj(reg)}}{conj(dat)}
   [true] % Pre
   {$res \igobs dameColumna(c,cr)$} % Pos
   [$O((\#cr)²*L)$ ] % Complejidad
   [Devuelve un conjunto con los datos del campo $c$ para cada registro de la tabla] % Descripción
   [] % Aliasing
  
  
\end{Interfaz}

\begin{Representacion}

  \begin{Estructura}{tabla}[estrTabla]

    \begin{Tupla}[estrTabla]
	\tupItem{Nombre}{nombreTabla}
	\tupItem{CantAccesos}{Nat}
	\tupItem{HayIndiceNat?}{Bool}

	\tupItem{CampoIndiceNat}{campo}
	\tupItem{HayIndiceString?}{Bool}

	\tupItem{CampoIndiceString}{campo}
    \tupItem{Columnas}{reg}
    
	\tupItem{Claves}{Conj<campo>}
	\tupItem{Registros}{ContenedorReg}
    
	\tupItem{Propiedades}{Dicc<campo, minMax>}
	\end{Tupla}

  \end{Estructura}
  
\textbf{Invariante de representaci\'on en castellano:}

    \begin{enumerate} 
		\item Si HayIndiceNat? es falso, CampoIndiceNat es $""$. Sino, pertenece a los Campos de Columnas.
		\item Si HayIndiceString? es falso, CampoIndiceString es $""$. Sino, pertenece a los  Campos de columnas.
        \item Claves esta incluido en los Campos de Columnas.
        \item CantAccesos es mayor o igual a la cantidad de registros.
        \item Claves no es vacio.
        \item Las claves de propiedades son los campos indexados.
        \item Todos los registros tienen los mismos campos que el registro Columnas.
        \item Todos los registros tienen el mismo tipo en cada campo que el registro Columnas.
        \item Para cada campo indexado, los minimos y maximos en propiedades, son el mismo tipo de dato que en los registros.
		\item Para todo campo indexado, los valores de ese campo en todos los registros estan acotados ente el min y max, y estos pertenecen.
        \item En cada campo que es clave, todos los registros tienen distintos valores.
        \item Si hay indice NAT, en Registros, las claves del diccionario IndiceNat son los datos de los registros en el CampoIndiceNat, y su significado es un conjunto de tuplas cuyo primer elemento es un iterador que apuntan a un registro del conjunto Registros, con ese dato en el campo indexado, y, si hay indice STRING, el segundo elemento es un iterador que apunta a una tupla en indiceString, cuyo primer elemento es un iterador que apunta al mismo registro. Si no hay indice STRING, este iterador apunta a NULL. 
        \item Si no hay indice nat, IndiceNat es un diccionario vacio.
        \item Si hay indice STRING, en Registros, las claves del diccionario IndiceString son los datos de los registros en el CampoIndiceString, y su significado es un conjunto de iteradores que apuntan a un registro del conjunto Registros, con ese dato en el campo indexado, y, si hay indice NAT, el segundo elemento es un iterador que apunta a una tupla en IndiceNat cuyo primer elemento es un iterador que apunta al mismo registro. Si no hay indice NAT, este iterador apunta a NULL. 
        \item Si no hay indice string, IndiceString es un diccionario vacio.
        
    \end{enumerate}
  \RepFc[Tabla][e]{ 
  \begin{enumerate}  
      \item $\neg$ e.HayIndiceNat? $\Rightarrow$ e.CampoIndiceNat $=$ $""$ $\land$ e.HayIndiceNat? $\Rightarrow$ e.CampoIndiceNat $\in$ campos(e.Columnas) $\land$
      \item $\neg$ e.HayIndiceString? $\Rightarrow$ e.CampoIndiceString $= "" \land$ e.HayIndiceString? $\Rightarrow$ e.CampoIndiceString $\in$ campos(e.Columnas) $\land$
      \item e.Claves $\subseteq$ campos(e.Columnas) $\yluego$
      \item cardinal(e.Registros.Registros) $\leq$ e.CantAccesos $\land$
      \item e.Claves $\neq \emptyset \ \yluego$
      \item Claves(e.Propiedades) $=$ Indices(t)  $\yluego$
      \item ($\forall$ r $\in$ e.Registros.Registros) campos(r) $=$ campos(e.Columnas) $\yluego$
      \item ($\forall$ r $\in$ e.Registros.Registros) ($\forall$ c $\in$ e.Campos) mismoTipo(Significado(r,c),Significado(e.Columnas,c)) $\yluego$
      \item ($\forall$ c $\in$ Indices(t)) mismoTipo(Significado(e.Propiedades, c).Min, Significado(e.Columnas, c)) $\land$ mismoTipo(Significado(e.Propiedades, c).Max, Significado(e.Columnas, c)) $\yluego$
      \item ($\forall$ r $\in$ e.Registros.Registros)($\forall$ c $\in$ Indices(t)) Significado(e.Propiedades, c).Min $\leq$ Significado(r, c) $\leq$ Significado(e.Propiedades, c).Max $\land$
      \item ($\forall$ c $\in$ indices(t)) Significado(e.Propiedades, c).Min $\in$ dameColumna(c, e.Registros.Registros) $\land$ Significado(e.Propiedades, c).Max $\in$ dameColumna(c, e.Registros.Registros) $\land$
      \item ($\forall$ r1, r2 $\in$ e.Registros.Registros) r1 $\neq$ r2 $\Rightarrow$ ($\forall$ c $\in$ e.Claves) Significado(r1, c) $\neq$ Significado(r2,c) $\land$
      \item e.HayIndiceNat? $\Rightarrow$ (Claves(e.Registros.IndiceNat) $=$ dameColumna(e.CampoIndiceNat, e.Registros.Registros) $\yluego$ ($\forall$ n $\in$ Claves(e.Registros.IndiceNat))(($\forall$ tupIt $\in$ Obtener(e.Registros.IndiceNat, n)) Obtener(Siguiente(tupIt.Reg), e.CampoIndiceNat) $=$ n $\land$ (e.HayIndiceString? $\Rightarrow$ Obtener(Siguiente(Siguiente(tupIt.OtroIndice).Reg), e.CampoIndiceNat) $=$ n) $\land$ ($\neg$ e.HayIndiceString? $\Rightarrow$ Siguiente(tupIt.OtroIndice) $=$ NULL) $\land$
      \item $\neg$ e.HayIndiceNat? $\Rightarrow$ Claves(e.Registros.IndiceNat) $=$ $\emptyset $ $\land$
      \item e.HayIndiceString? $\Rightarrow$ (Claves(e.Registros.IndiceString) $=$ dameColumna(e.CampoIndiceString, e.Registros.Registros) $\yluego$ ($\forall$ s $\in$ Claves(e.Registros.IndiceNat))($\forall$ tupIt $\in$ Obtener(e.Registros.IndiceString, s)) Obtener(Siguiente(tupIt.Reg), e.CampoIndiceString) $=$ s) $\land$ (e.HayIndiceNat? $\Rightarrow$ Obtener(Siguiente(Siguiente(tupIt.OtroIndice).Reg), e.CampoIndiceString) $=$ s) $\land$ ($\neg$ e.HayIndiceNat? $\Rightarrow$ Siguiente(tupIt.OtroIndice) $=$ NULL)$\land$
      \item $\neg$ e.HayIndiceString? $\Rightarrow$ Claves(e.Registros.IndiceString) $=$ $\emptyset$ $\land$
      \end{enumerate}
    }
    

  \AbsFc[estrDato]{Tabla}[e]{t: Tabla | 
	\begin{enumerate}
		\item e.Nombre $=$ nombre(t) $\land$
        \item e.CantAccesos $=$ cantidadDeAccesos(t) $\land$
        \item e.HayIndiceNat? $\Leftrightarrow$ ($\exists$ c $\in$ indices(t)) tipoCampo(c,t) $=$ true $\land$
        \item e.HayIndiceString? $\Leftrightarrow$ ($\exists$ c $\in$ indices(t)) tipoCampo(c,t) $=$ false $\land$
        \item e.CampoIndiceNat $\in$ indices(t)$\land$
        \item e.CampoIndiceString $\in$ indices(t)$\land$
        \item campos(e.Columnas) $=$ campos(t) $\land$
        \item e.Claves $=$ claves(t) $\land$
        \item e.Registros.Registros $=$ registros(t)
        \item esNat?(e.columnas,c) $=$ tipoCampo(c,t)
	\end{enumerate} }
  
  ~

 % \textbf{Funciones Auxiliares:}

  ~
    
 % \tadOperacion{...}{...}{...}{}
 % \tadAxioma{...}{...}
  
\end{Representacion}


\begin{Algoritmos}

  % \nuevoAlgo
  \begin{algoritmo}{inuevaTabla}{\In{nombre}{string}, \In{claves}{conj(campo)}, \In{columnas}{reg}}{estrTabla}
	registros $\leftarrow$ <Vacio(), Vacio(), Vacio()> \com*{$O(3)$}
    res $\leftarrow$ $\langle$ nombre, 0, false, $""$, false, $""$, columnas, claves, registros, Vacio() $\rangle$ \com*{$O(10 + |L|)$}
  
  \end{algoritmo}
  \datosAlgoritmo{} % Descripción
  {} % Pre
  {} % Post
  {$O(1)$} % Complejidad
  {Como los nombres de los campos son acotados $O(9 + |L|)$ = $O(10)$ donde |L| es la longitud del nombre de campo mas largo. Por algebra de ordenes $O(3)$ + $O(10)$ = $O(1)$} % Justificación  
  
  \begin{algoritmo}{agregarRegistro}{\In{r}{reg}, \Inout{t}{estrTabla}}{}
  	itReg $\leftarrow$ AgregarRapido(t.Registros.registros, r) \com*{$O(L)$} 
	\If(\com*[f]{$O(1)$}){t.HayIndiceNat?}{
    	\eIf(\com*[f]{$O(lg(n))$}){Definido?(valorN(Significado(r, t.CampoIndiceNat)),t.Registros.IndiceNat)}{
			itTupNat $\gets$ AgregarRapido(Obtener(valorN(Significado(r, t.CampoIndiceNat)), t.Registros.IndiceNat),$<$itReg, NULL$>$) \com*{$O(lg(n))$}
        }{
        	nuevoConjN $\leftarrow$ Vacio() \com*{$O(1)$} 
            itTupNat $\gets$ AgregarRapido(nuevoConjN, $<$itReg, NULL$>$)) \com*{$O(1)$} 
        	Definir(valorN(Significado(r, t.CampoIndiceNat)), nuevoConjN) \com*{$O(lg(n))$}
        }
    }
	\If(\com*[f]{$O(1)$}){t.HayIndiceString?}{
    	\eIf(\com*[f]{$O(|L|)$}){Definido?(valorS(Significado(r, t.CampoIndiceString)),t.Registros.IndiceString)}{
        	itTupStr $\gets$ AgregarRapido(Obtener(valorS(Significado(r, t.CampoIndiceString)), t.Registros.IndiceString),$<$itReg, NULL$>$)) \com*{$O(|L|)$}
        }{
        	nuevoConjS $\leftarrow$ Vacio() \com*{$O(1)$}
            itTupStr $\gets$ AgregarRapido(nuevoConjS, $<$itReg, NULL$>$)) \com*{$O(1)$}
        	Definir(valorS(Significado(r, t.CampoIndiceString)), nuevoConjS) \com*{$O(|L|)$}
        }
    }
    \If(\com*[f]{$O(1)$}){HayIndiceNat? $\land$ HayIndiceString?}{
    	Siguiente(itTupNat).OtroIndice $\gets$ itTupString \com*{$O(1)$}
        Siguiente(itTupStr).OtroIndice $\gets$ itTupNat \com*{$O(1)$}
    }
\If(\com*[f]{$O(1)$}){HayIndiceNat?}{
	\If(\com*[f]{$O(1)$}){Significado(r, campoIndiceNat) $<$ Significado(t.Propiedades, campoIndiceNat).Min}{
    	actualMax $\gets$ Significado(t.Propiedades, campoIndiceNat).Max \com*{$O(1)$}
        nuevoMin $\gets$ Significado(r, campoIndiceNat) \com*{$O(1)$}
		Definir(t.Propiedades, campoIndiceNat, $<$nuevoMin, actualMax$>$) \com*{$O(1)$}
	}
	\If(\com*[f]{$O(1)$}){Significado(r, campoIndiceNat) $>$ Significado(t.Propiedades, campoIndiceNat).Max}{
    	actualMin $\gets$ Significado(t.Propiedades, campoIndiceNat).Min \com*{$O(1)$}
        nuevoMax $\gets$ Significado(r, campoIndiceNat) \com*{$O(1)$}
		Definir(t.Propiedades, campoIndiceNat, $<$actualMin, nuevoMax$>$) \com*{$O(1)$}
	}
}
\If(\com*[f]{$O(1)$}){HayIndiceString?}{
	\If(\com*[f]{$O(1)$}){Significado(r, campoIndiceString) $<$ Significado(t.Propiedades, campoIndiceString).Min}{
    	actualMax $\gets$ Significado(t.Propiedades, campoIndiceString).Max \com*{$O(1)$}
        nuevoMin $\gets$ Significado(r, campoIndiceString) \com*{$O(1)$}
		Definir(t.Propiedades, campoIndiceString, $<$nuevoMin, actualMax$>$) \com*{$O(1)$}
	}
	\If(\com*[f]{$O(1)$}){Significado(r, campoIndiceString) $>$ Significado(t.Propiedades, campoIndiceString).Max}{
    	actualMin $\gets$ Significado(t.Propiedades, campoIndiceString).Min \com*{$O(1)$}
        nuevoMax $\gets$ Significado(r, campoIndiceString) \com*{$O(1)$}
		Definir(t.Propiedades, campoIndiceString, $<$actualMin, nuevoMax$>$) \com*{$O(1)$}
	}
}
    
    t.CantAccesos $\leftarrow$ t.CantAccesos + 1 \com*{$O(1)$}
    
  \end{algoritmo}
  \datosAlgoritmo{} % Descripción
  {} % Pre
  {} % Post
  {$O(L+lg(n))$} % Complejidad
 {Como la cantidad de campos de una tabla es acotada $O(Cardinal(campos(r)))$ $=$ $O(1)$  En el caso de no haber indices los if de las lineas 2 y 12 cuestan $O(1)$.Si hay indice String el if de la linea 12 cuesta $O(2|L|)$ $=$ $O(|L|)$. Si hay indice nat el if de la linea 2 cuesta $O(2lg(n))$ $=$ $O(lg(n))$.La unica operacion que no tiene complejidad $O(1)$ dentro del $while$  es la comparacion $\leq$ que cuesta $O(|L|)$ . Por lo tanto aplicando algebra de ordenes si no hay indices la complejidad es $O(|L|)$, si solo hay indice string la complejidad tambien es $O(|L|)$, si solo hay indice nat o ambos indices la complejidad es $O(|L| + lg(n))$} % Justificacion
  


  \begin{algoritmo}{iborrarRegistro}{\In{crit}{reg}, \Inout{t}{tbl}}{}
    itCamp $\leftarrow$ CrearIt(Campos(crit)) \com*{$O(1)$}
    campoBorr $\leftarrow$ Siguiente(itCamp) \com*{$O(1)$}
    valorBorr $\leftarrow$ Significado(crit, campoBorr) \com*{$O(1)$} %porque estan acotados los campos

    \If(\com*[f]{$O(1)$}){t.HayIndiceString? $\land$ campoBorr = t.CampoIndiceString}{
    	valorIndiceString $\gets$ valorBorr \com*{$O(1)$}
    	itRegBS $\leftarrow$ CrearIt(Obtener(t.Registros.IndiceString, valorBorr)) \com*{$O(|L|)$}
%        \While(\com*[f]{$O(Cardinal(t.Registros.Registros))$}){HaySiguiente(itRegBS)}{
            t.CantAccessos $\leftarrow$ t.CantAccesos + 1 \com*{$O(1)$}
            \If(\com*[f]{$O(1)$}){t.HayIndiceNat?}{
	            valorIndiceNat $\gets$ Significado(Siguiente(Siguiente(itRegBS).Reg), t.CampoIndiceNat)\com*{$O(1)$}
                EliminarSiguiente(Siguiente(itRegBs).OtroIndice)\com*{$O(1)$}
				\If(\com*[f]{$O(log(n)) + O(1)$}){Cardinal(Obtener(t.Registros.IndiceNat, valorIndiceNat)) $=$ 0}{
                	Borrar(t.Registros.IndiceNat, valorIndiceNat)\com*{$O(n) \ o \ O(log(n))$ en promedio}
                }
            }
            EliminarSiguiente(Siguiente(itRegBS).Reg) \com*{$O(1)$}
%            Avanzar(itRegBS) \com*{$O(1)$}
%        }
        Borrar(t.Registros.IndiceString, valorBorr) \com*{$O(|L|)$}
    }\Else{
        \If(\com*[f]{$O(1)$}){t.HayIndiceNat? $\land$ campoBorr $=$ t.CampoIndiceNat}{
    	valorIndiceNat $\gets$ valorBorr \com*{$O(1)$}
        itRegBN $\leftarrow$ CrearIt(Obtener(t.Registros.IndiceNat, valorBorr)) \com*{$O(n) \ o \ O(log(n))$ en promedio}
%        \While(\com*[f]{$O(Cardinal(t.Registros.IndiceNat))$}){HaySiguiente(itRegBN)}{
            \If(\com*[f]{$O(1)$}){t.HayIndiceString?}{
	            valorIndiceString $\gets$ Significado(Siguiente(Siguiente(itRegBS).Reg), t.CampoIndiceString)\com*{$O(1)$}
                EliminarSiguiente(Siguiente(itRegBs).OtroIndice)\com*{$O(1)$}
				\If(\com*[f]{$O(|L|) + O(1)$}){Cardinal(Obtener(t.Registros.IndiceString, valorIndiceString)) $=$ 0}{
                	Borrar(t.Registros.IndiceString, valorIndiceString)\com*{$O(L)$}
                }
            }
            EliminarSiguiente(Siguiente(itRegBN)) \com*{$O(2)$}
            t.CantAccessos $\leftarrow$ t.CantAccesos + 1 \com*{$O(1)$}
%            Avanzar(itRegBN) \com*{$O(1)$}
%        }
        Borrar(t.Registros.IndiceNat, valorBorr)  \com*{$O(n) \ o \ O(log(n))$ en promedio}
        }
    }
    
\end{algoritmo}
\begin{algoritmo}{iborrarRegistro}{...continua}{}
    
    \If(\com*[f]{$O(1)$}){($\neg$ (t.HayIndiceString? $\vee$ t.HayIndiceNat?)) $\vee$ $\neg$ ((t.HayIndiceString? $\yluego$ campoBorr = CampoIndiceString) $\vee$ (t.HayIndiceNat? $\yluego$ campoBorr = CampoIndiceNat))}{
    itReg $\leftarrow$ CrearIt(t.Registros.registros) \com*{$O(1)$}
    \While(\com*[f]{$O(Cardinal(t.Registros.registros))$}){HaySiguiente(itReg)}{
    	\If(\com*[f]{$O(|L|)$}){valorBorr = Significado(Siguiente(itReg), campoBorr)}{
        \If(\com*[f]{$O(1)$}){t.HayIndiceNat?}{
        valorIndiceNat $\gets$ Significado(Siguiente(itReg), CampoIndiceNat) \com*{$O(1)$}
        itBorr $\leftarrow$ CrearIt(Obtener(t.Registros.IndiceNat, valorIndiceNat )) \com*{$O(n) \ o \ O(log(n))$ en promedio}
        \While(\com*[f]{$O(1)$}){HaySiguiente(itBorr)}{
        \If(\com*[f]{$O(|L|)$}){Siguiente(Siguiente(itBorr).Reg) = Siguiente(itReg)}{
			\If(\com*[f]{$O(1)$}){t.HayIndiceString?}{
				valorIndiceString $\gets$ Significado(Siguiente(Siguiente(itBorr).Reg), t.CampoIndiceString)\com*{$O(1)$} 
				EliminarSiguiente(Siguiente(itBorr).OtroIndice) \com*{$O(1)$}
				\If(\com*[f]{$O(|L|) + O(1)$}){Cardinal(Obtener(t.Registros.IndiceString, valorIndiceString)) $=$ 0}{
                	Borrar(t.Registros.IndiceString, valorIndiceString)\com*{$O(|L|)$}
               	}
            }
            EliminarSiguiente(itBorr) \com*{$O(1)$}
        }
        Avanzar(itBorr) \com*{$O(1)$}
        }
        }
        \If(\com*[f]{$O(1)$}){t.HayIndiceString?}{
        valorIndiceString $\gets$ Significado(Siguiente(itReg), CampoIndiceString) \com*{$O(1)$}
        itBorr $\leftarrow$ CrearIt(Obtener(t.Registros.IndiceString, valorIndiceString )) \com*{$O(|L|)$}
        \While(\com*[f]{$O(1)$ en promedio}){HaySiguiente(itBorr)}{
        \If(\com*[f]{$O(|L|)$}){Siguiente(Siguiente(itBorr)) = Siguiente(itReg)}{
			\If(\com*[f]{$O(1)$}){t.HayIndiceNat?}{
				valorIndiceNat $\gets$ Significado(Siguiente(Siguiente(itBorr).Reg), t.CampoIndiceNat)\com*{$O(1)$} 
				EliminarSiguiente(Siguiente(itBorr).OtroIndice) \com*{$O(1)$}
				\If(\com*[f]{$O(|L|) + O(1)$}){Cardinal(Obtener(t.Registros.IndiceNat, valorIndiceNat)) $=$ 0}{
                	Borrar(t.Registros.IndiceNat, valorIndiceNat)\com*{$O(|L|)$}
               	}
            }
        	EliminarSiguiente(itBorr) \com*{$O(1)$}
        }
        Avanzar(itBorr) \com*{$O(1)$}
        }
        }
        EliminarSiguiente(itReg) \com*{$O(1)$}
        t.CantAccessos $\leftarrow$ t.CantAccesos + 1 \com*{$O(1)$}
        
        }
        Avanzar(itReg) \com*{$O(1)$}
    }    
}     
\end{algoritmo}

\begin{algoritmo}{iborrarRegistro}{...continua}{}
\If(\com*[f]{$O(1)$}){HayIndiceNat?}{
	\If(\com*[f]{$O(1)$}){valorIndiceNat $=$ Significado(t.Propiedades, campoIndiceNat).Min}{
    	actualMax $\gets$ Significado(t.Propiedades, campoIndiceNat).Max \com*{$O(1)$}
        nuevoMin $\gets$ minimaClave(t.IndiceNat) \com*{$O(log(n))$}
		Definir(t.Propiedades, campoIndiceNat, $<$nuevoMin, actualMax$>$) \com*{$O(1)$}
	}
	\If(\com*[f]{$O(1)$}){valorIndiceNat $=$ Significado(t.Propiedades, campoIndiceNat).Max}{
    	actualMin $\gets$ Significado(t.Propiedades, campoIndiceNat).Min \com*{$O(1)$}
        nuevoMax $\gets$ maximaClave(t.IndiceNat) \com*{$O(log(n))$}
		Definir(t.Propiedades, campoIndiceNat, $<$actualMin, nuevoMax$>$) \com*{$O(1)$}
	}
}
\If(\com*[f]{$O(1)$}){HayIndiceString?}{
	\If(\com*[f]{$O(1)$}){valorIndiceString $=$ Significado(t.Propiedades, campoIndiceString).Min}{
    	actualMax $\gets$ Significado(t.Propiedades, campoIndiceString).Max \com*{$O(1)$}
        nuevoMin $\gets$ minimaClave(t.IndiceString) \com*{$O(L)$}
		Definir(t.Propiedades, campoIndiceString, $<$nuevoMin, actualMax$>$) \com*{$O(1)$}
	}
	\If(\com*[f]{$O(1)$}){valorIndiceString $=$ Significado(t.Propiedades, campoIndiceString).Max}{
    	actualMin $\gets$ Significado(t.Propiedades, campoIndiceString).Min \com*{$O(1)$}
        nuevoMax $\gets$ maximaClave(t.IndiceString) \com*{$O(L)$}
		Definir(t.Propiedades, campoIndiceString, $<$actualMin, nuevoMax$>$) \com*{$O(1)$}
	}
}
\end{algoritmo}
  \datosAlgoritmo{} % Descripción
  {} % Pre
  {} % Post
  {$O(n*(n+L))$} % Complejidad
  {Obviando las operaciones constantes, tanto si entra en el primer if (linea 4 primer pagina) como el segundo (linea 19 primer pagina): $O(L)$ + $O(log(n))$ + $O(n)$
  
  + $O(n)$ * ($O(n)$ + 6*$O(L)$) Si entra en el tercer if (linea 1 segunda pagina)
  
  + $O(L) + O(log(n))$ (actualizar maximos y minimos)
  Por \'algebra de ordenes, en peor caso: = $O(n*(n+L))$.

Si el criterio de borrado es indice sobre campo tipo nat, no entra nunca al if de la segunda pagina, entonces en promedio:

$O(log(n))$ + $O(L)$ + $O(log(n))$ + $O(L)$ + $O(log(n))$ = $O(log(n) + L)$
} % Justificación
  
  
\begin{algoritmo}{Indexar}{\In{c}{campo}, \Inout{t}{estrTabla}}{}
     itReg $\leftarrow$ CrearIt(t.Registros.registros) \com*{$O(1)$}
     \If(\com*[f]{$O(|L|)$}){esNat?(Significado(t.Columnas, c))}{
     	t.HayIndiceNat? $\leftarrow$ true \com*{$O(1)$}
        t.CampoIndiceNat $\leftarrow$ c \com*{$O(1)$}
         \While(\com*[f]{$O(Cardinal(t.Registros.registros))$}){HaySiguiente(itReg)}{
         	itTupStr $\leftarrow$ NULL \com*{$O(1)$}
            \If(\com*[f]{$O(1)$}){t.HayIndiceString}{
            	itOtroInd $\leftarrow$ CrearIt(Obtener(t.Registros.IndiceString, valorS(Significado(Siguiente(itReg), t.CampoIndiceString)))) \com*{$O(2|L|)$}
                \While(\com*[f]{$O(C)$}){ (Siguiente(itOtroInd).Reg $\neq$ itReg) $\yluego$ HaySiguiente(itOtroInd)}{
                	itTupStr $\leftarrow$ Siguiente(itOtroInd) \com*{$O(1)$}
                	Avanzar(itOtroInd) \com*{$O(1)$}
                }
            }
         	\eIf(\com*[f]{$O(n + |L|)$}){Definido?(valorN(Significado(Siguiente(itReg), c)), t.Registros.IndiceNat)}{
             	itTupNat $\leftarrow$ AgregarRapido(Obtener(t.Registros.IndiceNat, valorN(Significado(Siguiente(itReg), c))),<itReg, itTupStr>) \com*{$O(n + |L|)$}
             }{
               conjAux $\leftarrow$ Vacio() \com*{$O(1)$}
               itTupNat $\leftarrow$ AgregarRapido(conjAux, <itReg, itTupStr>) \com*{$O(1)$}
               Definir(valorN(Significado(Siguiente(itReg), c))), conjAux, t.Registros.IndiceNat) \com*{$O(n + |L|)$}
             }
            \If(\com*[f]{$O(1)$}){Significado(Siguiente(itReg), c) $<$ Significado(t.Propiedades, campoIndiceNat).Min}{
                actualMax $\gets$ Significado(t.Propiedades, campoIndiceNat).Max \com*{$O(1)$}
                nuevoMin $\gets$ Significado(Siguiente(itReg), c) \com*{$O(1)$}
                Definir(t.Propiedades, campoIndiceNat, $<$nuevoMin, actualMax$>$) \com*{$O(1)$}
            }
            \If(\com*[f]{$O(1)$}){Significado(Siguiente(itReg), c) $>$ Significado(t.Propiedades, campoIndiceNat).Max}{
                actualMin $\gets$ Significado(t.Propiedades, campoIndiceNat).Min \com*{$O(1)$}
                nuevoMax $\gets$ Significado(Siguiente(itReg) \com*{$O(1)$}
                Definir(t.Propiedades, campoIndiceNat, $<$actualMin, nuevoMax$>$) \com*{$O(1)$}
            }
             Avanzar(itReg) \com*{$O(1)$}
         }
     }
\end{algoritmo}
\begin{algoritmo}{Indexar}{...continua}{}

     \Else{
     	t.HayIndiceString? $\leftarrow$ true \com*{$O(1)$}
        t.CampoIndiceString $\leftarrow$ c \com*{$O(1)$}
         \While(\com*[f]{$O(Cardinal(t.Registros.registros))$}){HaySiguiente(itReg)}{
         itTupNat $\leftarrow$ NULL \com*{$O(1)$}
         \If(\com*[f]{$O(1)$}){t.HayIndiceNat}{
            	itOtroInd $\leftarrow$ CrearIt(Obtener(t.Registros.IndiceNat, valorN(Significado(Siguiente(itReg), t.CampoIndiceNat)))) \com*{$O(lg(n) + |L|)$}
                \While(\com*[f]{$O(C)$}){ (Siguiente(itOtroInd) $\neq$ itReg) $\yluego$ HaySiguiente(itOtroInd)}{
                	itTupNat $\leftarrow$ Siguiente(itOtroInd).OtroIndice \com*{$O(1)$}
                	Avanzar(itOtroInd) \com*{$O(1)$}
                }
                
            }
         	\eIf(\com*[f]{$O(2|L|)$}){Definido?(valorS(Significado(Siguiente(itReg), c)), t.Registros.IndiceString)}{
             	itTupStr $\leftarrow$ AgregarRapido(Obtener(t.Registros.IndiceString, valorS(Significado(Siguiente(itReg), c))),<itReg, NULL>) \com*{$O(|L|)$}
             }{
               conjAux $\leftarrow$ Vacio() \com*{$O(1)$}
               itTupStr $\leftarrow$ AgregarRapido(conjAux, <itReg, NULL>) \com*{$O(1)$}
               Definir(valorS(Significado(Siguiente(itReg), c))), conjAux, t.Registros.IndiceString) \com*{$O(2|L|)$}
             }
            \If(\com*[f]{$O(1)$}){Significado(Siguiente(itReg), c) $<$ Significado(t.Propiedades, campoIndiceString).Min}{
                actualMax $\gets$ Significado(t.Propiedades, campoIndiceString).Max \com*{$O(1)$}
                nuevoMin $\gets$ Significado(Siguiente(itReg), c) \com*{$O(1)$}
                Definir(t.Propiedades, campoIndiceString, $<$nuevoMin, actualMax$>$) \com*{$O(1)$}
            }
            \If(\com*[f]{$O(1)$}){Significado(Siguiente(itReg), c) $>$ Significado(t.Propiedades, campoIndiceString).Max}{
                actualMin $\gets$ Significado(t.Propiedades, campoIndiceString).Min \com*{$O(1)$}
                nuevoMax $\gets$ Significado(Siguiente(itReg) \com*{$O(1)$}
                Definir(t.Propiedades, campoIndiceString, $<$actualMin, nuevoMax$>$) \com*{$O(1)$}
            }
             Avanzar(itReg)   \com*{$O(1)$}
     }
     }
   \end{algoritmo}
   \datosAlgoritmo{} % Descripción
   {} % Pre
   {} % Post
   {$O(n * (n * lg(n) + L))$} % Complejidad
   {C es el cardinal del conjunto de iteradores que estoy recorriendo. n es la cantidad de registros que tiene mi tabla, los cuales voy a recorrer para agregar un iterador a cada registro en el indice que estoy creando. C esta acotado por n ya que mi peor caso sucederia si tengo que recorrer en la linea 8 o 27 un conjunto de iteradores que tiene a todos los iteradores a los registros por lo que su cardinal seria n.  \\ En el caso de que no habia otro indice creado previamente (no entraria en los $IF$ de las lineas 6 y 26 por lo que me costarian $O(1)$) mis complejidades son las siguientes: Si el indice es nat, como las datos nat se insertan uniformemente las complejidades de definir y definido? son $O(lg(n))$ por lo que sumando $O(|L|)$ de la operacion Significado, si el indice es nat mi complejidad final sera $O(n*(lg(n) + L))$. Si el indice es string, definir y definido? me cuestan $O(L)$ por lo que mi complejidad final sera $O(n*(4*L))$ = $O(n*(L))$ \\ En el caso de que ya tenia un indice creado mis complejidades son las siguientes: Si el indice es nat, como las datos nat se insertan uniformemente las complejidades de definir y definido? son $O(lg(n))$ por lo que sumando $O(L)$ de la operacion Significado y una C por la busqueda que hago en la linea 8 o 28 en el otro indice (acoto cada C por n), si el indice a crear es nat mi complejidad final sera $O(n*(n + lg(n) + L))$. Si el indice es string, definir y definido? me cuestan $O(L)$ y por cada registro que indexo le sumo una C por la busqueda que hago en la linea 8 o 28 en el otro indice (acoto cada C por n) por lo que mi complejidad final sera $O(n*(n + 4*L))$ = $O(n*(n + L))$} % Justificación
  
  
\begin{algoritmo}{iNombre}{\In{t}{estrTabla}}{string}
    res $\leftarrow$ t.Nombre \com*{$O(1)$}
\end{algoritmo}
  \datosAlgoritmo{} % Descripción
  {} % Pre
  {} % Post
  {$O(1)$} % Complejidad
  {Se realiza una sola operacion que cuesta $O(1)$} % Justificación

  \begin{algoritmo}{iClaves}{\In{t}{estrTabla}}{conj(campo)}
    res $\leftarrow$ t.Claves \com*{$O(1)$}
  \end{algoritmo}
  \datosAlgoritmo{} % Descripción
  {} % Pre
  {} % Post
  {$O(1)$} % Complejidad
 {Se copia el conjunto de claves de la tabla, este conjunto esta acotado porque la cantidad de campos de una tabla esta acotado, por lo tanto el costo de esta operacion es $O(1)$} % Justificación
  
  \begin{algoritmo}{iIndices}{\In{t}{estrTabla}}{conj(campo)}
    ccampo $\leftarrow$ Vacio() \com*{$O(1)$}
    \If(\com*[f]{$O(1)$}){t.HayIndiceNat?}{
    	AgregarRapido(ccampo, t.CampoIndiceNat) \com*{$O(1)$}
    }
    \If(\com*[f]{$O(1)$}){t.HayIndiceString?}{
    	AgregarRapido(ccampo, t.CampoIndiceString) \com*{$O(1)$}
    }
    res $\leftarrow$ ccampo \com*{$O(1)$}
  \end{algoritmo}
  \datosAlgoritmo{} % Descripción
  {} % Pre
  {} % Post
  {$O(1)$} % Complejidad
  {Solo se ejecutan operaciones con costo $O(1)$ por lo que por algebra de ordenes la complejidad es $O(1)$ + $O(1)$ + $O(1)$ +$O(1)$ = $O(4)$ = $O(1)$ } % Justificación

  \begin{algoritmo}{iCampos}{\In{t}{estrTabla}}{conj(campo)}
    res $\leftarrow$ Campos(t.Columnas) \com*{$O(1)$}
  \end{algoritmo}
  \datosAlgoritmo{} % Descripción
  {} % Pre
  {} % Post
  {$O(1)$} % Complejidad
  {Solo se ejecuta una operacion con costo $O(1)$} % Justificación

  \begin{algoritmo}{iesNat?}{\In{c}{campo}, \In{t}{estrTabla}}{bool}
	$res \gets esNat?(Significado(t.Columnas, c))$ \com*{$O(1)$}
  \end{algoritmo}
  \datosAlgoritmo{} % Descripción
  {} % Pre
  {} % Post
  {$O(L)$} % Complejidad
  {Ya que es una busqueda sobre diccionario lineal, pero la cantidad de claves son acotadas y campo es un string acotado} % Justificación

  \begin{algoritmo}{iRegistros}{\In{t}{estrTabla}}{conj(reg)}
    res $\leftarrow$ t.Registros.registros \com*{$O(1)$}
  \end{algoritmo}
  \datosAlgoritmo{} % Descripción
  {} % Pre
  {} % Post
  {$O(1)$} % Complejidad
  {Solo se ejecuta una operacion con costo $O(1)$ } % Justificación

  \begin{algoritmo}{iCantidadDeAccesos}{\In{t}{estrTabla}}{nat}
    res $\leftarrow$ t.CantAccesos \com*{$O(1)$}
  \end{algoritmo}
  \datosAlgoritmo{} % Descripción
  {} % Pre
  {} % Post
  {$O(1)$} % Complejidad
  {Se ejecuta una sola operacion con costo $O(1)$} % Justificación

 \begin{algoritmo}{iPuedoInsertar?}{\In{r}{reg}, \In{t}{estrTabla}}{bool}
	res $\leftarrow$ (Compatible(r,t) $\wedge$ $\neg$ HayCoincidencia(r, Claves(t), Registros(t))) \com*{$O(1) + O(Cardinal(cr) * (Cardinal(cc) * |L|))$}
  \end{algoritmo}
  \datosAlgoritmo{} % Descripción
  {} % Pre
  {} % Post
  {$O(n*L)$} % Complejidad
  {$O(Cardinal(cr) * (Cardinal(cc) * |L|))$, siendo cr los registros de t, y cc los campos de t. Como t tiene cantidad acotada de campos:
  $O(Cardinal(cr) * (Cardinal(cc) * |L|))$} % Justificación

  \begin{algoritmo}{iCompatible}{\In{r}{reg}, \In{t}{estrTabla}}{bool}
    \eIf(\com*[f]{$O(|L|)$}){Campos(r) = Campos(t)}{
    	res $\leftarrow$ MismosTipos(r,t) \com*{$O(mtip)$}
    }{
    	res $\leftarrow$ false \com*{$O(1)$}
    }
  \end{algoritmo}
  \datosAlgoritmo{} % Descripción
  {} % Pre
  {} % Post
  {$\Theta(1)$} % Complejidad
  {} % Justificación

  \begin{algoritmo}{hayCoincidencia}{\In{r}{reg}, \In{cc}{conj(campo)}, \In{cr}{conj(reg)}}{bool}
  itReg $\leftarrow$ CrearIt(cr) \com*{$O(1)$}
  encCoinc $\leftarrow$ false \com*{$O(1)$}
  \While(\com*[f]{$O(Cardinal(cr))$}){HaySiguiente(itReg) $\wedge$ $\neg$ encCoinc}{
  	encCoinc $\leftarrow$ CoincideAlguno(r, cc, Siguiente(itReg)) \com*{$O(Cardinal(cc) * |L|)$}
    Avanzar(itReg) \com*{$O(1)$}
  }
    res $\leftarrow$ encCoinc
  \end{algoritmo}
  \datosAlgoritmo{} % Descripción
  {} % Pre
  {} % Post
  {$O(Cardinal(cr) * (Cardinal(cc) * |L|))$} % Complejidad
  {} % Justificación

  \begin{algoritmo}{imismosTipos}{\In{r}{reg}, \In{t}{estrTabla}}{bool}
  $itRc \gets CrearIt(campos(r))$ \com*{$O(\#Claves($r$)*copy(d))$}
  $res \gets true $ \com*{$O(1)$}
  \While(\com*[f]{$O(1)$}){HaySiguiente($itRc$) $\land$ $res$}{
  	$c \gets Siguiente(itRc) $ \com*{$O(1)$, lo asigna por referencia}
  	$f \gets Significado(r,Siguiente(itRc)) $ \com*{$O(\sum_{k' \in K}equal(Sig(itRc),k'))$}
  	$res \gets esNat?(f) = esNat?(c,t)$   \com*{$O(\sum_{t' \in T}equal(c,t'))$}
  	$Avanzar(itRc) $ \com*{$O(1)$}
  }
  \end{algoritmo}
  \datosAlgoritmo{} % Descripción
  {} % Pre
  {} % Post
  {$O((K + T) * L)$} % Complejidad
  {donde $K$ $=$ campos($r$), $T$ $=$ campos($t$), $d$ es un dato, copy(d) $=$ 1 si $d$ es nat o largo de string sino, llamemoslo $L$, equal entre Campos tambien tiene complejidad largo de string, por lo que abusando de la notacion llamemos $L$ tambien a este mismo} % Justificación

  \begin{algoritmo}{iminimo}{\In{c}{campo}, \In{t}{estrTabla}}{dat}
  	$res \gets Significado(t.Propiedades,c).Min$ \com*{$O(\#t.Propiedades*equal(c,k) k \in claves(t.Propiedades))$}
  \end{algoritmo}
  \datosAlgoritmo{} % Descripción
  {} % Pre
  {} % Post
  {$ O(1)$} % Complejidad
  {ya que los campos son string acotados y las $\#$claves(t.Propiedades) $\leq$ 2, ademas $res$ se asigna por Referencia } % Justificación

  \begin{algoritmo}{imaximo}{\In{c}{campo}, \In{t}{estrTabla}}{dat}
    $res \gets Significado(t.Propiedades,c).Max$ \com*{$O(\#t.Propiedades*equal(c,k) k \in claves(t.Propiedades))$}
  \end{algoritmo}
  \datosAlgoritmo{} % Descripción
  {} % Pre
  {} % Post
  {$ O(1)$} % Complejidad
  {ya que los campos son string acotados y las $\#$claves(t.Propiedades) $\leq$ 2, ademas $res$ se asigna por Referencia} % Justificación


  \begin{algoritmo}{ipuedoIndexar}{\In{c}{campo}, \In{t}{estrTabla}}{bool}
  $r \gets Pertenece?(campos(t), c) $ \com*{$O(\sum_{k' \in K}equal(c,k'))$}
  $e \gets \neg Pertenece?(indices(t), c) $ \com*{$O(\sum_{i' \in I}equal(c,i'))$}
  \eIf(\com*[f]{$O(\sum_{k' \in K}equal(c,k'))$}){esNat?(c,t)}{
    	$s \gets \neg t.HayIndiceNat?$ \com*{$O(1)$}
    }{
    	$s \gets \neg t.HayIndiceString?$ \com*{$O(1)$}
    }
    $res \gets r \land e \land s $ \com*{$O(1)$}
  \end{algoritmo}
  \datosAlgoritmo{} % Descripción
  {} % Pre
  {} % Post
  {$O(K* L)$} % Complejidad
  {donde K $=$ campos(t), I $=$ indices(t), ademas \#I $\leq$ 2 entoces K*equal(..) es igual a (K+2)*equal(..), equal entre Campos tiene complejidad largo de string, llamemos L a este mismo} % Justificación

  \begin{algoritmo}{icoincidencias}{\In{r}{reg}, \In{cr}{conj(reg)}}{conj(reg)}
  itR $\gets$ CrearIt(cr) \com*{$O(1)$}
  res $\gets$ Vacio()  \com*{$O(1)$}
  \While(\com*[f]{$O(1)$}){HaySiguiente($itR$)}{
  	\If(\com*[f]{$O(\#campos(r)*(\#campos(r)+\#campos(Sig(itR))+L))$}){coincidenTodosAux($r$,campos($r$),Siguiente($itR$))}{
    	$AgregarRapido(res, Siguiente(itR))$ \com*{$O(copy(Sig(itR)))$}
    }
  	$Avanzar(itR) $ \com*{$O(1)$}
  }
  \end{algoritmo}
  \datosAlgoritmo{} % Descripción
  {} % Pre
  {} % Post
  {$O(\#cr * L)$} % Complejidad
  {coincidenTodosAux tiene dicha complejidad , ya que equal entre Datos tiene complejidad del orden 1 si es nat o largo de string, llamemos $L$ a este mismo y ademas el conjunto que se usa para llamarlo son los campos de r. $MR$ es MAX($\#$campos($reg$)), $reg$ $\in$ cr

$O(\#cr*\#campos(r)*(\#campos(r)+MR+L)) = O(\#cr * L)$
Ya que en el contexto de las tablas los registros tienen cantidad acotada de campos.} % Justificación
  
% POR LO QUE BUSCAR SERIA LITERALMENTE<coicidenciasRap(dameTabla(t,b),criterio)>

  \begin{algoritmo}{icoincidenciasRap}{\In{t}{estrTabla}, \In{r}{reg}}{conj(reg)}
  \eIf(\com*[f]{$O(1)$}){$\neg$(t.HayIndiceNat? $\lor$ t.HayIndiceString?)}{
  	$res \gets coincidencias(r, registros(t))$ \com*{$O(\#regristros(t)*\#campos(r)*(\#campos(r)+MR+L))$}
  }{
  	$itR \gets CrearIt(r)$ \com*{$O(1)$}
  	$res \gets Vacio()$ \com*{$O(1)$}
  	$continue \gets true$ \com*{$O(1)$}
    $matchS \gets false$ \com*{$O(1)$}
    $matchN \gets false$ \com*{$O(1)$}
    $keyN \gets false$ \com*{$O(1)$}
    $keyS \gets false$ \com*{$O(1)$}
  	\While(\com*[f]{$O(1)$}){HaySiguiente($itR$) $\land$ continue}{
    	\If(\com*[f]{$O(L)$}){SiguienteClave(itR) $=$ t.CampoIndiceNat $\land$ esNat?(SiguienteSignificado($itR$))}{
  			\eIf(\com*[f]{$O(\#claves(t)*L)$}){Pertenece?(claves(t),SiguienteClave(itR)}{
            		$continue \gets false$ \com*{$O(1)$}
            		$matchN \gets true$ \com*{$O(1)$}
					$keyN \gets true$ \com*{$O(1)$}
            }{
            		$matchN \gets true$ \com*{$O(1)$}
            		$v \gets ValorN(SiguienteSignificado(itR))$ \com*{$O(1)$} 
            }
  		}
  		\If(\com*[f]{$O(L)$}){SiguienteClave(itR) $=$ t.CampoIndiceString $\land$ esString?(SiguienteSignificado($itR$))}{
  			\eIf(\com*[f]{$O(\#claves(t)*L)$}){Pertenece?(claves(t),SiguienteClave(itR)}{
            	$continue \gets false$ \com*{$O(1)$}
            	$matchS \gets true$ \com*{$O(1)$}
				$keyS \gets true$ \com*{$O(1)$}
            }{
            	$matchS \gets true$ \com*{$O(1)$}
            	$s \gets ValorS(SiguienteSignificado(itR))$ \com*{$O(L)$} 
            }
		}   
  		\If(\com*[f]{$O(1)$}){continue}{
  			$Avanzar(itR) $ \com*{$O(1)$}
  		}
  	}
    \If(\com*[f]{$O(1)$}){keyN}{
    	$v \gets ValorN(SiguienteSignificado(itR))$ \com*{$O(1)$}
  		$itC \gets CrearIt(Obtener(v,t.Registros.IndiceNat))$ \com*{$O(n)$}			 %obtener de dicNat
  		\If(\com*[f]{$O(1)$}){HaySiguiente($itC$)}{							 	 %si es clave hay un solo elemento
  			\If(\com*[f]{$O(sub$)}){Sub?(r,Siguiente(Siguiente(itC).Reg))}{
  				$AgregarRapido(res,Siguiente(Siguiente(itC).Reg))$ \com*{$O(1)$}
  			}
  		}
	}        
  	\If(\com*[f]{$O(1)$}){keyS}{ 
    	$s \gets ValorS(SiguienteSignificado(itR))$ \com*{$O(L)$}
        $itC \gets CrearIt(Obtener(s,t.Registros.IndiceString))$ \com*{$O(|s|)$} 	  %obtener de dicStr
        \If(\com*[f]{$O(1)$}){HaySiguiente($itC$)}{								  %si es clave hay un solo elemento
        	\If(\com*[f]{$O(sub$)}){Sub?(r,Siguiente(Siguiente(itC).Reg))}{
        		$AgregarRapido(res,Siguiente(Siguiente(itC).Reg))$ \com*{$O(1)$}
        	}
        }
	}
  $\textbf{NOTA: el Algoritmo no termina aca, sigue abajo, como si estuviera en esta parte, osea}$
  $\textbf{todavia adentro del ELSE}$
  }
  \end{algoritmo}
  \begin{algoritmo}{Continuacion de coincidenciasRap}{}{}
  $\textbf{Aca pongo este ELSE para que se note que sigo adento del anterior}$
  \Else{
  	\If(\com*[f]{$O(1)$}){matchN $\land$ $\neg$(keyS $\lor$ keyN)}{
  		$itC \gets CrearIt(Obtener(v,t.Registros.IndiceNat))$ \com*{$O(n)$}			 %obtener de dicNat
  		\While(\com*[f]{$O(1)$}){HaySiguiente($itC$)}{							 	 
  			\If(\com*[f]{$O(sub$)}){Sub?(r,Siguiente(Siguiente(itC).Reg))}{
  				$AgregarRapido(res,Siguiente(Siguiente(itC).Reg))$ \com*{$O(1)$}
  			}
  			$Avanzar(itC) $ \com*{$O(1)$}
  		}
	}        
  	\If(\com*[f]{$O(1)$}){matchS $\land$ $\neg$(keyS $\lor$ keyN)}{ 
        $itC \gets CrearIt(Obtener(s,t.Registros.IndiceString))$ \com*{$O(|s|)$} 	  %obtener de dicStr
        \While(\com*[f]{$O(1)$}){HaySiguiente($itC$)}{								 
        	\If(\com*[f]{$O(sub$)}){Sub?(r,Siguiente(Siguiente(itC).Reg))}{
        		$AgregarRapido(res,Siguiente(Siguiente(itC).Reg))$ \com*{$O(1)$}
        	}
        	$Avanzar(itC)$ \com*{$O(1)$}
        }
	}
   } 
  \end{algoritmo}
  \datosAlgoritmo{} % Descripción
  {}  %Pre 
  {}  %Post
  {$O(\#campos(t)*\#campos(r)*(\#campos(r)+MR+L)+\#campos(r)*(n+(CN*sub)+L+(CS*sub) + \#indices(t)*L))$} % Complejidad
  { 
  	\begin{enumerate}
		\item donde sub $=$ $\#$campos(r)*($\#$campos($r$)*$L$ + $\#$campos(t)*$L$ + $L$), $MR$ es $\#$campos($t$), $n$ $=$ $\#$registros(t), $CN$ es la cantidad de registros que tienen el mismo valor que $r$ para el campo Ind. Nat de $t$, $CS$ es la cantidad de registros que tienen el mismo valor que $r$ para el campo Ind. String de $t$, $|s|$ $\leq$ $L$, esa es la complejidad dada en peor caso, la primera parte la hereda de coincidencias de t y la segunda $n$ y $L$ es la complejidad de buscar en un DicNat y en un DicStr respectivamente, mas la complejidad de iterar el conjunto($CN$ o $CS$) y la heredada de Sub. 
    	\item en el caso de que el criterio de busqueda sea un campo $\textbf{indice}$ y $\textbf{clave}$, la primera parte queda descartada ya que hay indice, tambien como los campos son acotados pasan a ser una constante, por lo que sub pasa a tener complejidad $L$ y $CN$, $CS$ tambien pasan a estar acotadas, mas precisamente son igual a 1, por ser clave el indice, entonces la complejidad queda de la forma $\textbf{O(n + L)}$, pero nosotros sabemos que los datos nat estan bien distribuidos por lo que queda $\textbf{O(log(n) + L)}$, ya que buscar en un DicNat sobre datos bien distribuidos es logaritmico sobre cantidad de elementos.   
	\end{enumerate}
% Sub? tiene dicha complejidad, ya que la cantidad de campos de registros(t) y $r$ es acotada, por lo tanto pasa a ser una constante y equal entre Datos tiene complejidad del orden 1 si es nat o largo de string, llamemos $L$ a este mismo
	}
    
\begin{algoritmo}{icombinarRegistros}{\In{c}{campo}, \In{cr1}{conj(reg)}, \In{cr2}{conj(reg)}}{conj(reg)}
	$itRegs \gets CrearIt(cr1)$ \com*{$O(1)$}  	
	$regRes \gets Vacio()$ \com*{$O(1)$}   
	\While(\com*[f]{$O(\#cr1)$}){HaySiguiente(itRegs)}{
        $reg1 \gets Siguiente(itRegs) $ \com*{$O(1)$}   
    	$regsComb \gets CombinarTodos(c,reg1, cr2) $ \com*{$O(combinarTodos(c,r,cr2))$}   
        $itRegsComb \gets CrearIt(regsComb)$ \com*{$O(1)$}  	
        \While(\com*[f]{$O(\#regsComb)$}){HaySiguiente(itRegsComb)}{
        	$regComb \gets Siguiente(itRegsComb)$ \com*{$O(1)$}   
        	$Agregar(regRes, regComb) $ \com*{$O(\#regRes * L)$}   
			$Avanzar(itRegsComb) $ \com*{$O(1)$}
        }
		$Avanzar(itRegs) $ \com*{$O(1)$}
    }
    $res \gets regRes $ \com*{$O(1)$}
\end{algoritmo}
\datosAlgoritmo{} % Descripción
  {} % Pre 
  {} % Post
  {$O(n*m*L*min\lbrace m, n \rbrace)$} % Complejidad
  {\begin{enumerate}
  	\item $\#$regsComb puede ser como maximo \#cr2. Luego lo podemos acotar por dicho valor y lo renombramos $m$
    \item $\#$cr1 lo renombramos $n$
    \item $\#$regRes esta acotado por el minimo entre $m$ y $n$
    \item $L$ es la longitud m\'as larga de alg\'un string en alg\'un registro de cualquiera de los dos conjuntos.
  	\item Para cada registro en el conjunto cr1 hacemos:
    \item Llamamos a combinarTodos con el campo, el registro y el otro conjunto de registros. La complejidad en peor caso es O(m*L)
    \item Luego para cada registro Combinado hacemos:
    \item Lo agregamos al conjunto resultado. Esto necesita recorrerlo y determinar si pertenece o no. Para no agregar Repetidos.
    \item Por lo tanto la complejidad del loop anterior es $O(m*(min\lbrace m,n \rbrace*L))$
    \item Entonces. La complejidad del algoritmo, en el peor caso, es  $O(n*(m*L + (m*min\lbrace m,n \rbrace *L))))$
    \item Por algebra de ordenes, se deduce que esa complejidad $ \in O(n*m*L*(1+min\lbrace m, n \rbrace)) $
    \item Siguiendo, se llega a $ O(n*m*L*min\lbrace m, n \rbrace) $ 
	\end{enumerate} 
    } % Justificación

\begin{algoritmo}{icombinarRegistrosRap}{\In{c}{campo}, \In{cr1}{conj(reg)},\In{t}{estrTabla}}{conj(reg)}
	\eIf(\com*[f]{$O(1)$}){Pertenece?(Indices(t), c)}{
    	$itRegs \gets CrearIt(cr1)$ \com*{$O(1)$}  	
		$regsComb \gets Vacio()$ \com*{$O(1)$}   
		\While(\com*[f]{$O(\#cr1)$}){HaySiguiente(itRegs)}{
    		$reg1 \gets Siguiente(itRegs) $ \com*{$O(1)$}   
	        \eIf(\com*[f]{$O(1)$}){esNat?(c,t)}{
		       	\If(\com*[f]{$O(log(registros(t)))$}){Definido?(Significado(reg1, c), t.Registros.IndiceNat)}{
					$regs2 \gets Obtener(Significado(reg1, c), t.Registros.IndiceNat) $ \com*{$O(log(registros(t))$}
                    $itRegs2 \gets CrearIt(regs2) $ \com*{$O(1)$}
					\While(\com*[f]{$O(\#regs2)$}){HaySiguiente(itRegs2)}{
                    	$reg2 \gets Siguiente(itRegs2).Reg $ \com*{$O(1)$}
                        $regComb \gets AgregarCampos(reg1, reg2) $ \com*{$O(L)$}
						$Agregar(regsComb, regComb) $ \com*{$O(\#regsComb*L)$}
						$Avanzar(itRegs2) $ \com*{$O(1)$}
                    }						        	
       			}
        	}{
        		\If(\com*[f]{$O(L)$}){Definido?(Significado(reg2, c), t.Registros.IndiceString)}{
					$regs2 \gets Obtener(Significado(reg1, c), t.Registros.IndiceString) $ \com*{$O(L)$}
                    $itRegs2 \gets CrearIt(regs2) $ \com*{$O(1)$}
					\While(\com*[f]{$O(\#regs2)$}){HaySiguiente(itRegs2)}{
                    	$reg2 \gets Siguiente(itRegs2).Reg $ \com*{$O(1)$}
                        $regComb \gets AgregarCampos(reg1, reg2) $ \com*{$O(L)$}
						$Agregar(regsComb, regComb) $ \com*{$O(\#regsComb*L)$}
						$Avanzar(itRegs2) $ \com*{$O(1)$}
                    }
       			}
        	}
			$Avanzar(itRegs) $ \com*{$O(1)$}
    	}
        $res \gets regsComb $ \com*{$O(1)$}   
    }{
    	$res \gets combinarRegistros(c,cr1, registros(t)) $ \com*{$O(combinarRegistros(c,cr1, cr2))$}   
    }
	
\end{algoritmo}
\datosAlgoritmo{} % Descripción
  {} % Pre 
  {} % Post
  {$O(n*m*L*min\lbrace m, n \rbrace)$ Y, s\'i c es \'Indice, $O(n*((log(m) + L)*m*min\lbrace m,n\rbrace*L))$. Y, s\'i adem\'as c es clave, $O(n*((log(m) + L)))$} % Complejidad
  {\begin{enumerate}
	\item $\#registros(t)$ lo renombramos $m$
    \item $\#$cr1 lo renombramos $n$
    \item $L$ es la longitud m\'as larga de alg\'un string en alg\'un registro de cualquiera de los dos conjuntos.
  	\item En el peor caso, este algoritmo se comporta como CombinarRegistros(c,cr1,cr2)
    \item Pero si el campo por el cual se pide combinar fuese indice de la tabla. Entonces se puede mejorar la complejidad. A continuaci\'on se detalla como:
  	\item Para cada registro en el conjunto cr1 hacemos:
    \item Lo buscamos en el \'indice de la tabla. Esta operaci\'on est\'a acotada por la siguiente complejidad O(log(m) + L). Porque siempre realiza alguna de las dos. O sea, s\'i lo busca en el DiccNat la complejidad ser\'ia O(log m) y s\'i lo busca en el DiccStr la complejidad ser\'ia O(L)
    \item Luego para cada registro Combinado hacemos:
    \item Llamamos a la operacion AgregarCampos(r1,r2) cuya complejidad es O(L)
    \item Lo agregamos al conjunto resultado. Esta Operacion tiene una complejidad O(min(m,n)*L) 
    \item Por lo tanto la complejidad del loop anterior es $O(m*(min\lbrace m,n \rbrace*L))$
    \item Entonces. La complejidad del algoritmo, en el peor caso, es  $O(n*((log(m)+L) + (m*min\lbrace m,n \rbrace *L)))$
    \item Si el campo adem\'as fuera clave. Entonces la complejidad en el peor caso ser\'ia $O(n*((log(m)+L)))$. Porque el conjunto de resultados de buscar en los indices ser\'ia de un elemento.
    \end{enumerate} 
    } % Justificación

\begin{algoritmo}{idameColumna}{\In{c}{campo}, \In{cr}{conj(reg)}}{conj(dat)}
    $itRegs \gets CrearIt(cr)$ \com*{$O(1)$}  	
    $columna \gets Vacio()$ \com*{$O(1)$}
    \While(\com*[f]{$O(\#cr)$}){HaySiguiente(itRegs)}{
		$dato \gets Significado(Siguiente(itRegs), c)$ \com*{$O(1)$}
    	\If(\com*[f]{$O(\#cr*L)$}){$\neg$ Pertenece?(columna,dato)}{
	        $AgregarRapido(columna, dato) $ \com*{$O(copy(L))$}
    	    
        }
        $Avanzar(itRegs) $ \com*{$O(1)$}
    }
	$res \gets columna $ \com*{$O(1)$}
  \end{algoritmo}
  \datosAlgoritmo{} % Descripción
  {} % Pre
  {} % Post
  {$ (O(\#cr)*O(\#cr*L)) \in O(((\#cr)^2)*L)$} % Complejidad
  {
	  \begin{enumerate}
		\item Recorre todos los elementos del conjunto de registros recibido por parametro. Esto tiene una complejidad $O(\#cr)$
		\item Para cada elemento del conjunto se pregunta si ya pertenece al conjunto que esta construyendo. Esto tiene una complejidad $O(\#cr*L)$. Porque si el tipo del campo fuera String. El costo de comparar dos string depende del largo de la string. Y entonces, el peor caso queda acotado superiormente por esa complejidad. Donde L representa la longitud del mayor Dato String en el conjunto de registros.
        \item Luego agrega el dato al conjunto. El peor caso ser\'ia si el dato fuera String. En ese caso, la complejidad de agregar queda acotada superiormente por la complejidad de copiar el string. 
        \item Por los items anteriores y usando algebra de ordenes se obtiene la complejidad $O(((\#cr)^2)*L)$
    \end{enumerate}
  } % Justificación
  
\end{Algoritmos}

\section{M\'odulo Base de Datos}

\Encabezado{Notas Preliminares}
  En todos los casos, al indicar las complejidades de los algoritmos, las variables que se utilizan corresponden a:
  \vspace{-0.5em}\begin{itemize}
    \item $n$: N\'umero de registros en la tabla pasada por par\'ametro.
    \item $m$: N\'umero de registros en la tabla pasada por par\'ametro.
    \item $L$: M\'axima longitud de un valor STRING de un registro en la tabla pasada por par\'ametro.
    \item $T$: Cantidad de tablas en la base de datos
    \item $R$: Cantidad de elementos en la lista de modificaciones de un join.
  \end{itemize}


\begin{Interfaz}
	\seExplicaCon{BaseDeDatos}
    
    \generos{\tipo{bd}} 
    
 		\servUsados{tbl, nombreTabla, itTablas, reg, campo, bool, conj, tupla, diccStr, datosTabla}
    
    \Encabezado{Operaciones de Base De Datos}
    
    \InterfazFuncion{nuevaDB}{}{bd}%
    [true]%pre
    {res $\igobs$ nuevaDB()}%pos
    [$O(1)$]%complejidad
    [Genera una nueva Base De Datos. ]%descripcion
    []%aliasing
    
    \InterfazFuncion{agregarTabla}{\In{t}{tbl}, \Inout{b}{bd}}{}%
    [b $\igobs$ $b_{0}$ $\land$ $\emptyset$?(registros(t))]%pre
    {b $\igobs$ agregarTabla(t, $b_{0}$)}%pos
    [$O(1)$]%complejidad
    [Agrega una nueva tabla sin registros a la base de datos]%descripcion
    []%aliasing
    
    \InterfazFuncion{insertarEntrada}{\In{reg}{reg}, \In{t}{nombreTabla}, \Inout{b}{bd}}{}%
    [b $\igobs$ $b_{0}$ $\land$ t $\in$ tablas(b) $\yluego$ puedoInsertar?(reg, t)]%pre
    {b $\igobs$ insertarEntrada(reg, t, $b_{0}$)}%pos
    [$O(T*L + lg(n))$]%complejidad
    [Inserta un nuevo registro en la tabla cuyo nombre es pasado por par\'ametro]%descripcion
    []%aliasing
    
    \InterfazFuncion{borrar}{\In{cr}{reg}, \In{t}{nombreTabla}, \Inout{b}{bd}}{}%
    [b $\igobs$ $b_{0}$ $\land$ t $\in$ tablas(b) $\land$ \#(campos(cr))=1]%pre
    {b $\igobs$ borrar(reg, t, $b_{0}$)}%pos
    [$O(T*L + n *(n + L)$]%complejidad
    [Borra todos los registros de la tabla, cuyo nombre es pasado por par\'ametro, cuyos datos del campo del registro pasado por par\'ametro coincidan con el dato del mismo registro]%descripcion
	[]%aliasing
    
    \InterfazFuncion{generarVistaJoin}{\In{t1}{nombreTabla}, \In{t2}{nombreTabla}, \In{c}{campo} \Inout{b}{bd}}{}%
    [b $\igobs$ $b_{0} \land$ t1 $\neq$ t2 $\land$ $\lbrace$t1, t2$\rbrace$ $\subseteq$ tablas(b) $\yluego$ c $\in$ claves(dameTabla(t1, b)) $\land$ c $\in$ claves(dameTabla(t2,b)) $\land$ $\lnot$hayJoin?(t1,t2,b)]%pre
    {b $\igobs$ generarVistaJoin(t1, t2, c, $b_{0}$)}%pos
    [$O((n+m)*(L + log(n + m)))$]%complejidad
    [Genera un join entre las tablas cuyos nombres son pasados por par\'ametro]%descripcion
    []%aliasing 
  
    \InterfazFuncion{borrarJoin}{\In{t1}{nombreTabla} \In{t2}{nombreTabla} \Inout{b}{bd}}{}%
    [b $\igobs$ $b_{0} \land$ hayJoin?(t1,t2,b)]%pre
    {b $\igobs$ borrarJoin(t1,t2,$b_{0}$}%pos
    [$O(1)$]%complejidad
    [Borra el join entre las tablas cuyos nombres son pasados por par\'ametro]%descripcion
    []%aliasing 
  
    \InterfazFuncion{tablas}{\In{b}{bd}}{itTablas}%
    [true]%pre
    {esPermutacion?(SecuSuby(res), tablas(b)) }%pos
    [$O(1)$]%complejidad
    [Devuelve un iterador de Tablas con siguiente ]%descripcion
    []%aliasing     

    \InterfazFuncion{dameTabla}{\In{t}{nombreTabla} \In{b}{bd}}{tbl}%
    [t $\in$ tablas(b)]%pre
    {res $\igobs$ dameTabla(t,b) }%pos
    [$O(1)$]%complejidad
    [Devuelve la tabla de la base de datos cuyo nombre es $t$]%descripcion
    [tbl se devuelve por referencia]%aliasing     

    \InterfazFuncion{hayJoin?}{\In{t1}{nombreTabla} \In{t2}{nombreTabla} \In{b}{bd}}{bool}%
    [true]%pre
    {res $\igobs$ hayJoin(t1,t2,b) }%pos
    [$O(1)$]%complejidad
    [Devuelve si hay join entre las dos tablas cuyos nombres son los pasados como parametro]%descripcion
    []%aliasing   

    \InterfazFuncion{campoJoin}{\In{t1}{nombreTabla} \In{t2}{nombreTabla} \In{b}{bd}}{campo}%
    [hayJoin?(t1,t2,b)]%pre
    {res $\igobs$ campoJoin(t1,t2,b)}%pos
    [$O(1)$]%complejidad
    [Devuelve el campo por el cual hay join en las tablas cuyos nombres son pasados como parametro]%descripcion
    []%aliasing   

    \InterfazFuncion{registros}{\In{t}{nombreTabla} \In{b}{bd}}{conj(reg)}%
    [t $\in$ tablas(b)]%pre
    {res $\igobs$ registros(t,b)}%pos
    [$O(n)$]%complejidad
    [Devuelve el conjunto de registros de la tabla cuyo nombre es pasado como parametro]%descripcion
    [El conjunto de registros de pasa por referencia no modificable]%aliasing   

    \InterfazFuncion{vistaJoin}{\In{t1}{nombreTabla} \In{t2}{nombreTabla} \In{b}{bd}}{conj(reg)}%
    [hayJoin?(t1,t2,b)]%pre
    {res $\igobs$ vistaJoin(t1,t2,b)}%pos
    [$O(R*(L+log(n+m))$]%complejidad
    [Actualiza y devuelve el join entre las dos tablas cuyos nombres son pasados como parametro.]%descripcion
    [El conjunto de registros de devuelve por referencia no modificable]%aliasing   

    \InterfazFuncion{cantidadDeAccesos}{\In{t}{nombreTabla} \In{b}{bd}}{conj(reg)}%
    [t $\in$ tablas(b)]%pre
    {res $\igobs$ cantidadDeAccesos(t,b)}%pos
    [$O(1)$]%complejidad
    [Devuelve la cantidad de accesos de la tabla cuyo nombre es pasado como parametro]%descripcion
    []%aliasing   
    
    \InterfazFuncion{tablaMaxima}{\In{b}{bd}}{nombreTabla}%
    [tablas(b) $\neq \emptyset$]%pre
    {res $\igobs$ tablaMaxima(t1,t2,b)}%pos
    [$O(1)$]%complejidad
    [Devuelve el nombre de la tabla m\'axima de la base de datos]%descripcion
    []%aliasing   

    \InterfazFuncion{encontrarMaximo}{\In{t}{nombreTabla} \In{ct}{conj(nombreTabla)} \In{b}{bd}}{nombreTabla}%
    [\{t\} $\cup$ ct $\subseteq$ tablas(b)]%pre
    {res $\igobs$ encontrarMaximo(t,ct,b)}%pos
    [$O(T)$]%complejidad
    [Devuelve el nombre de la tabla m\'axima de entre una tabla y conjunto de tablas cuyos nombres son pasados por parametro]%descripcion
    []%aliasing   
    
    \InterfazFuncion{buscar}{\In{criterio}{reg} \In{t}{nombreTabla} \In{b}{bd}}{conj(reg)}%
    [t $\in$ tablas(b)]%pre
    {res $\igobs$ buscar(criterio,t,b)}%pos
    [$O(n*L)$ en peor caso, si alguno de los campos de criterio es un campo indexado y clave en la tabla de nombre t la complejidad es $O(log(n) + L)$]%complejidad
    [Devuelve un conjunto con los registros de la tabla con nombre $t$ que coincidan en campo y dato para cada campo de $r$ utilizando los campos indexados de t (si existen y si son campos de criterio) para realizar la busqueda en menor tiempo]%descripcion
    [El conjunto de registros se pasa por referencia no modificable]%aliasing   
\end{Interfaz}    

\begin{Representacion}
	\begin{Estructura}{bd}[estrBD]
    	\begin{Tupla}[estrBD]
		\tupItem{NombresTablas}{Conj<nombreTabla>)}
        \tupItem{Tablas}{DiccStr<nombreTabla, datosTabla>}
        \tupItem{tablaMaxima}{nombreTabla}
        \end{Tupla}
	\end{Estructura}
    
    
\textbf{Invariante de Representaci\'on en castellano}

	\begin{enumerate}
		\item TablaMaxima es el nombre de una tabla en las tablas de la base de datos
        \item Para cada tabla en Tablas, la cantidad de accesos es menor o igual a la cantidad de accesos de la tabla cuyo nombre es el de tablaMaxima
        \item Los nombres en Tablas son los mismos que el conjunto NombresTablas, y para cada nombre en Tablas, la Tabla en su significado tiene el mismo nombre.
        \item En cada Tabla, las tablas en Joins pertenecen a las tablas de la base de datos.
		\item En cada Tabla, el nombre de la tabla, no aparece en sus Joins en datosTabla.
        \item En cada Tabla, las claves de todas las tablas con las que tiene Join, estan incluidas en las claves de Tabla.
        \item En cada Tabla, y cada Join, el campo del Join es un campo de las dos tablas.
        \item En cada Tabla, los registros de Joins de datosTabla, son el resultado de combinar las dos claves por el campo del Join.
        \item En cada Tabla, si el campo del Join es de tipo String, entonces en ConjJoin en Joins de datosTabla, cada DatoString de IndiceString es equivalente a obtener el CampoIndiceString de Registros, y su significado es una tupla cuyo primer elemento es un iterador que apunta al registro que tiene ese dato en el campo CampoIndiceString, y el segundo es NULL.
        \item En cada Tabla, si el campo del Join es de tipo Nat, entonces en ConjJoin en Joins de datosTabla, cada DatoNat de IndiceNat es equivalente a obtener el CampoIndiceNat de Registros, y su significado es una tupla cuyo primer elemento es un iterador que apunta al registro que tiene ese dato en el campo CampoIndiceString, y el segundo es NULL.
	\end{enumerate}
    
 \textbf{Invariante de Representaci\'on}
     \Rep[bd][e]{
     \begin{enumerate}
		\item ($\exists$ n $\in$ e.NombresTablas) n = e.tablaMaxima $\yluego$ 
		\item ($\forall$ n $\in$ e.NombresTablas) cantidadDeAccesos(Obtener(n , e.Tablas).Tabla) $\leq$ cantidadDeAccesos(Obtener( e.TablaMaxima, e.Tablas).Tabla)  $\land$ 
		\item e.NombresTablas = Claves(e.Tablas) $\yluego$ ($\forall$ n $\in$ e.NombresTablas)) n $=$ nombre(Obtener(n, e.Tablas).Tabla) $\land$ 
		\item ($\forall$ n1 $\in$ e.NombresTablas) ($\forall$ n2 $\in$ Claves(Obtener(n1, e.Tablas).Joins)) n2 $\in$ e.NombresTablas $\yluego$ 
        \item ($\forall$ n1 $\in$ e.NombresTablas) ($\forall$ n2 $\in$ Claves(Obtener(n1,	 e.Tablas).Joins)) n1 $\neq$ n2 $\land$ 
        \item ($\forall$ n1 $\in$ e.NombresTablas) ($\forall$ n2 $\in$ Claves(Obtener(n1, e.Tablas).Joins)) claves(Obtener(n2, e.Tablas).Tabla) $\in$ claves(Obtener(n1, e.Tablas).Tabla) $\land$ 
        \item ($\forall$ n1 $\in$ e.NombresTablas) ($\forall$ n2 $\in$ Claves(Obtener(n1, e.Tablas).Joins)) Obtener(n2, Obtener(n1, e.Tablas).Joins).Campo $\in$ campos(Obtener(n1, e.Tablas).Tabla) $\land$ Obtener(n2, Obtener(n1, e.Tablas).Joins).Campo $\in$ campos(Obtener(n2, e.Tablas).Tabla) $\land$ 
		\item ($\forall$ n1 $\in$ e.NombresTablas) ($\forall$ n2 $\in$ Claves(Obtener(n1, e.Tablas).Joins)) Obtener(n2, Obtener(n1, e.Tablas).Joins).ConjJoin.Registros $=$ CombinarRegistros(Obtener(n2, Obtener(n1, e.Tablas).Joins).ConjJoin.Campo , registros(Obtener(n1, e.Tablas).Tabla) , registros(Obtener(n1, e.Tablas).Tabla) ) $\land$
		\item ($\forall$ n1 $\in$ e.NombresTablas) ($\forall$ n2 $\in$ Claves(Obtener(n1, e.Tablas).Joins)) String?(Obtener(n2, Obtener(n1, e.Tablas).Joins).Campo) $\impluego$ 
        ($\forall$ s$\in$ Claves(Obtener(n2, Obtener(n1, e.Tablas).Joins).ConjJoin.IndiceString)) Obtener(Obtener(n2, Obtener(n1, e.Tablas).Join).Campo , Siguiente(Obtener(s , Obtener(n2, Obtener(n1, e.Tablas).Join).ConjJoin.IndiceString))) $=$ s
		\item ($\forall$ n1 $\in$ e.NombresTablas) ($\forall$ n2 $\in$ Claves(Obtener(n1, e.Tablas).Joins)) Nat?(Obtener(n2, Obtener(n1, e.Tablas).Joins).Campo) $\impluego$ 
        ($\forall$ s$\in$ Claves(Obtener(n2, Obtener(n1, e.Tablas).Joins).ConjJoin.IndiceNat)) Obtener(Obtener(n2, Obtener(n1, e.Tablas).Joins).Campo , Siguiente(Obtener(s ,Obtener(n2, Obtener(n1, e.Tablas).Joins).ConjJoin.IndiceNat))) $=$ s.
      \end{enumerate}  
      }


\textbf{Funcion de Abstracci\'on}

\Abs[estrBD]{BaseDeDatos}[e]{bd}{
\begin{enumerate}
	\item tablas(bd) $=$ e.NombresTablas $\yluego$ 
    \item ($\forall$ s $\in$ tablas(bd)) s $\in$ tablas(bd) $\impluego$ dameTabla(s, bd) $=$ Obtener(s, e.Tablas).Tabla $\yluego$ 
    \item ($\forall$ t1 $\in$ tablas(bd)) ($\forall$ t2 $\in$ tablas(b)) hayJoin?(t1,t2,bd) $=$ Def?(t2, Obtener(t1, e.Tablas).Joins) $\land$ 
    \item ($\forall$ t1 $\in$ tablas(bd)) ($\forall$ t2 $\in$ tablas(b)) hayJoin?(t1,t2,bd) $\impluego$ campoJoin(t1,t2,bd) $=$ Obtener(t2, Obtener(t1, e.Tablas).Join).Campo   
    \end{enumerate}
}
    
\end{Representacion}  

\begin{Algoritmos}

\begin{algoritmo}{iNuevaBD}{}{estrBD}
	$res.NombreTablas \gets Vacio()$ \com*{$O(1)$}
  	$res.Tablas \gets Vacio()$ \com*{$O(1)$}
    $res.tablaMaxima \gets "" $ \com*{$O(1)$}
    
\end{algoritmo}
\datosAlgoritmo{} % Descripción
{} % Pre
{} % Post
{$O(1)$} % Complejidad
{$O(1) + O(1) + O(1) = O(1) $} % Justificación

\begin{algoritmo}{iAgregarTabla}{\Inout{b}{estrBD}, \In{t}{tbl}}{bd}
	\If(\com*[f]{$O(|t.Nombre|)$}){$\neg$ Definido?(b.Tablas, t.Nombre)}{
    	$AgregarRapido(b.NombreTablas, t.Nombre)$ \com*{$O(copy(t.Nombre))$}
        $datos \gets < t , Vacio() > $ \com*{$O(1) + O(1)$}
        $Definir(b.Tablas, t.Nombre, datos) $ \com*{$O(copy(t.Nombre))$}
        \If(\com*[f]{$O(equals(b.tablaMaxima, t.Nombre)$}){b.tablaMaxima = $""$}{
            $b.tablaMaxima \gets t.Nombre$ \com*{$O(copy(t.Nombre))$}
        }
    }	 
\end{algoritmo}
\datosAlgoritmo{} % Descripción
{} % Pre
{} % Post
{$O(copy(t.Nombre))$} % Complejidad
{El algoritmo tiene llamada a funciones con costo $O(copy(t.Nombre))$, $O(|t.Nombre|)$ , $O(equals(b.tablaMaxima, t.Nombre)$ , $O(1)$ y $O(1)$. Aplicando \'algebra de \'ordenes: \\ $O(|t.Nombre|)$ + $O(copy(t.Nombre))$ + $O(1)$ + $O(1)$ + $O(copy(t.Nombre))$ + $O(equals(b.tablaMaxima, t.Nombre)$ + $O(copy(t.Nombre))$ = $O(copy(t.Nombre))$

Como el nombre de las tablas esta acotado, el costo de copiado es $O(1)$} % Justificación

\begin{algoritmo}{iInsertarEntrada}{\In{reg}{reg}, \In{t}{string}, \Inout{b}{estrBD}}{}
	$tablaMaxima \gets DameTabla(b.tablaMaxima, b)$ \com*{$O(|b.tablaMaxima|)$}
   	$tabla \gets DameTabla(t, b)$ \com*{$O(|t|)$}
   	$AgregarRegistro(reg, tabla)$ \com*{$O(L) + O(in)$}
    \If(\com*[f]{$O(1)$}){$CantidadAccesos(tabla) > CantidadAccesos(tablaMaxima)$}{		    	    			$b.tablaMaxima \gets t$ \com*{$O(1)$}
    }
	$ActualizarJoin(true, reg,t,b)$ \com*{$O(T*L)$}
\end{algoritmo}
\datosAlgoritmo{} % Descripción
{} % Pre
{} % Post
{$O(T*L + lg(n))$} % Complejidad
{Por algebra de ordenes
$O(|b.tablaMaxima|) + O(|t|) + O(L) + O(in) + O(1) + O(T*L) =$

$O(1) + O(1) + O(L) + O(in) + O(1) + O(T*L) =$

$O(T*L) + O(in)$ Ya que los nombres de las tablas son acotados.

Donde in es $O(|L|)$, si no hay indices o si solo hay indice string, y $O(lg(n))$ si hay indice NAT} % Justificación
		
\begin{algoritmo}{iBorrar}{\In{cr}{reg}, \In{t}{string}, \Inout{b}{estrBD}}{}
	$tablaMaxima \gets DameTabla(b.tablaMaxima, b)$ \com*{$O(|b.tablaMaxima|)$}
   	$tabla \gets DameTabla(t, b)$ \com*{$O(|t|)$}
   	$BorrarRegistro(reg, tabla)$ \com*{$O(L) + O(in)$}
    \If(\com*[f]{$O(1)$}){$CantidadAccesos(tabla) > CantidadAccesos(tablaMaxima)$}{		    	    			$b.tablaMaxima \gets t$ \com*{$O(1)$}
    }
	$ActualizarJoin(false, cr,t,b)$ \com*{$O(T*L)$}
    
\end{algoritmo}
\datosAlgoritmo{} % Descripción
{} % Pre
{} % Post
{$O(T*L + n*(n+L)) $} % Complejidad
{$O(T) + O(L) + O(in) + O(T*L)$ = $O(T*L + in)$ Por algebra de ordenes.

Donde in es $O(n *(n + L))$ en peor caso, y $O(log(n) + L)$ si el criterio de borrado es indice sobre campo tipo nat. } % Justificación 

\begin{algoritmo}{iGenerarVistaJoin}{\In{t1}{string}, \In{t2}{string}, \In{c}{campo}, \Inout{b}{estrBD}}{}
    $datosT \gets Obtener(b.Tablas, t1)$ \com*{$O(|t1.Nombre|)$}
   	$contenedor \gets <Vacio(), Vacio(), Vacio() > $ \com*{$O(3)$}
	$join \gets <c, Vacio(), contenedor > $ \com*{$O(3)$}
	$Definir(datosT.Joins, t2, join) $ \com*{$O(copy(t1.Nombre))$}
    $regsT1 \gets registros(t1, bd)$ \com*{$O(1)$}
    $regsComb \gets combinarRegistrosRap(c, regsT1, dameTabla(t2))$ \com*{$O(A)$}
    $itRegsComb \gets CrearIt(regsComb) $ \com*{$O(1)$}
    \While(\com*[f]{$O((n+m)*(3*L+(3*log(n+m)))$}){($HaySiguiente(itRegsComb)$)}{
    
		$regComb \gets Siguiente(itRegsComb) $ \com*{$O(1)$} 
        $itRegComb \gets AgregarRapido(contenedor.Registros, regComb) $ \com*{$O(1)$} 
    	\eIf(\com*[f]{$O(|campos(regComb)|)$}){esNat?(Significado(regComb,c))}{
        	$DatoNat \gets ValorN(Significado(regComb, c))$ \com*{$O(|campos(regComb)|)$} 
            \eIf(\com*[f]{$O(log(n+m))$}){Definido?(contenedor.IndiceNat, DatoNat)}{
            	$ConjIts \gets Significado(contenedor.IndiceNat, DatoNat) $ \com*{$O(log(n+m))$}
                $AgregarRapido(ConjIts, $<$itRegComb,NULL$>$) $ \com*{$O(1)$}
            }{
            	$ConjIts \gets Vacio() $ \com*{$O(1)$}
				$AgregarRapido(ConjIts, $<$itRegComb,NULL$>$) $ \com*{$O(1)$}
				$DefinirRapido(contenedor.IndiceNat, DatoNat, ConjIts) $ \com*{$O(log(n+m))$}             	
            }
        }
        {
        	$DatoString \gets ValorS(Significado(regComb, c))$ \com*{$O(|campos(regComb)|)$} 
            \eIf(\com*[f]{$O(L)$}){Definido?(contenedor.IndiceString, DatoString)}{
            	$ConjIts \gets Significado(contenedor.IndiceString, DatoString) $ \com*{$O(L)$}
                $AgregarRapido(ConjIts, $<$itRegComb,NULL$>$) $ \com*{$O(1)$}
            }{
            	$ConjIts \gets Vacio() $ \com*{$O(1)$}
				$AgregarRapido(ConjIts, $<$itRegComb,NULL$>$) $ \com*{$O(1)$}
				$DefinirRapido(contenedor.IndiceString, DatoString, ConjIts) $ \com*{$O(L)$}             	
            }
		}
        $Avanzar(itRegsComb)$ \com*{$O(1)$}
    }
\end{algoritmo}

\datosAlgoritmo{} % Descripción
{} % Pre
{} % Post
{$O((n+m)*(L+log(n+m))$} % Complejidad
{\begin{enumerate}
\item $n$ es la cantidad de registros de la tabla 1
\item $m$ es la cantidad de registros de la tabla 2
\item $L$ es la maxima longitud de algun campo string de algun registro que este en alguno de los dos conjuntos de registros de las tablas.
\item $A$ es la complejidad de combinarRegistrosRap. Y como el campo es Clave e Indice. Entonces esta complejidad es $O(n*(log(m) + L))$
\item El nombre de las tablas es acotado por lo tanto $O(t1.Nombre) \in O(1)$
\item La cantidad de campos de un registro es acotado por lo tanto $O(|campos(regComb)|) \in O(1)$
\item Por los items anteriores y las justificaciones del codigo se deduce que la complejidad del while es $O((n+m)*(3*L+(3*log(n+m)))$
\item Por lo tanto la complejidad del algoritmo es $O((n+m)*(3*L+(3*log(n+m)))) + O(n*(log m + L))$
\item Aplicando algebra de ordenes se obtiene $O(n*L + m*L + n*log(n+m) + m*log(n+m) + n*log(m) + n*L)$
\item Siguiendo $O(n*L +n*L + m*L + n*log(n+m) + m*log(n+m) + n*log(m))$
\item Siguiendo $O(2*n*L + m*L + n*log(n+m) + m*log(n+m) + n*log(m))$
\item Siguiendo $O(2*n*L + m*L + n*log(n+m) + m*log(n+m) + n*log(m+n))$
\item Siguiendo $O(2*n*L + m*L + 2*n*log(n+m) + m*log(n+m))$
\item Siguiendo $O(n*L + m*L + n*log(n+m) + m*log(n+m))$
\item Finalmente queda que la complejidad es $O((n+m)*(L + log(n+m)))$
\end{enumerate}} % Justificación

\begin{algoritmo}{iborrarJoin}{\In{t1}{nombreTabla} \In{t2}{nombreTabla} \Inout{b}{estrBD}}{}

    $datosT \gets Obtener(b.Tablas, t1)$ \com*{$O(|t1|)$}
    $borrar(datosT.Joins, t2)$\com*{$O(1)$}
    
\end{algoritmo}

\datosAlgoritmo{} % Descripción
{} % Pre
{} % Post
{$O(1)$} % Complejidad
{$O(|t1|) + O(1) = O(1) $porque los nombres de las tablas estan acotados.} % Justificación

\begin{algoritmo}{itablas}{\In{b}{estrBD}}{itTablas}
    $itTbl \gets CrearIt(b.NombresTablas)$ \com*{$O(1)$}
	$res \gets itTbl$ \com*{$O(1)$}  	    
\end{algoritmo}

\datosAlgoritmo{} % Descripción
{} % Pre
{} % Post
{$O(1)$} % Complejidad
{$O(1) + O(1) = O(1)$} % Justificación

\begin{algoritmo}{idameTabla}{\In{t}{nombreTabla} \In{b}{estrBD}}{tbl}

	$datosT \gets Obtener(b.Tablas,t)$ \com*{$O(|t|)$}
	$res \gets datosT.Tabla$ \com*{$O(1)$}
    
\end{algoritmo}

\datosAlgoritmo{} % Descripción
{} % Pre
{} % Post
{$O(1)$} % Complejidad
{$O(|t|) + O(1) = O(1)$ porque los nombres de las tablas estan acotados} % Justificación

\begin{algoritmo}{ihayJoin?}{\In{t1}{nombreTabla} \In{t2}{nombreTabla} \In{b}{estrBD}}{bool}

	$datosT \gets Obtener(b.Tablas, t1)$ \com*{$O(|t1|)$}
	$res \gets Def?(datosT.Joins, t2)$ \com*{$O(|t2|)$}

\end{algoritmo}

\datosAlgoritmo{} % Descripción
{} % Pre
{} % Post
{$O(1)$} % Complejidad
{$O(|t1|) + O(|t2|) = O(1)$ porque los nombres de las tablas estan acotados} % Justificación

\begin{algoritmo}{icampoJoin}{\In{t1}{nombreTabla} \In{t2}{nombreTabla} \In{b}{estrBD}}{campo}%

	$datosT \gets Obtener(b.Tablas,t1)$ \com*{$O(|t1|)$}
    $res \gets Obtener(datosT.Joins, t2).Campo$ \com*{$O(|t2|)$}
    
\end{algoritmo}

\datosAlgoritmo{} % Descripción
{} % Pre
{} % Post
{$O(1)$} % Complejidad
{$O(|t1|) + O(|t2|) = O(1)$ porque los nombres de las tablas estan acotados} % Justificación

\begin{algoritmo}{iregistros}{\In{t}{nombreTabla} \In{b}{estrBD}}{conj(reg)}%

	$datosT \gets Obtener(b.Tablas,t1)$ \com*{$O(|t|)$}
    $res \gets registros(datosT.Tabla)$ \com*{$O(registros(t))$}
    
\end{algoritmo}

\datosAlgoritmo{} % Descripción
{} % Pre
{} % Post
{$O(registros(t))$} % Complejidad
{$O(|t|) + O(registros(t)) = O(registros(t))$ porque los nombres de las tablas estan acotados} % Justificación


\begin{algoritmo}{iVistaJoin}{\In{t1}{string}, \In{t2}{string}, \In{b}{estrBD}}{conj(reg)}
    $datosT \gets Obtener(b.Tablas, t1)$ \com*{$O(|t1|)$}
    $join \gets Obtener(datosT.Joins, t2) $ \com*{$O(|t2|)$}
    $itMod \gets CrearIt(join.Modifiaciones) $ \com*{$O(1)$}
	\While(\com*[f]{$O(R*(L+log(n*m)))$}){HaySiguiente(itMod)}{
    	$ mod \gets Siguiente(itMod) $ \com*{$O(1)$}
        $conjReg2 \gets buscar(mod.Reg, t2, b)$ \com*{$O(buscar(r,t,b))$}
        $conjReg1 \gets Vacio() $ \com*{$O(1)$}
        $AgregarRapido(conjReg1, mod.Reg) $ \com*{$O(1)$}
        $regsComb \gets combinarRegistros(join.Campo, conjReg1, conjReg2)$ \com*{$O(A)$}
        $itRegsComb \gets CrearIt(regsComb) $ \com*{$O(1)$}
        \While(\com*[f]{$O(\#regsComb)$}){($HaySiguiente(itRegsComb)$)}{
            $regComb \gets Siguiente(itRegsComb) $ \com*{$O(1)$} 
        	\eIf(\com*[f]{$O(|campos(regComb)|)$}){Nat?(Significado(regComb,join.Campo))}{
				$AuxActContenedor(join.ConjJoin, regComb, mod, join.Campo, true)$ \com*{$O(log(n+m) + L)$}
        	}
        	{
				$AuxActContenedor(join.ConjJoin, regComb, mod, join.Campo, false)$ \com*{$O(log(n+m) + L)$}
			}
		}
    }
    $ res \gets join.Reg $ \com*{$O(1)$}
\end{algoritmo}
\datosAlgoritmo{} % Descripción
{} % Pre
{} % Post
{$O(R*(L+log(n+m))$} % Complejidad
{\begin{enumerate}
\item $R$ es la cantidad de elementos en la lista de modificaciones del Join
\item $n$ es la cantidad de elementos de la tabla 1
\item $m$ es la cantidad de elementos de la tabla 2
\item El nombre de las tablas es acotado por lo tanto $O(t1.Nombre) \in O(1)$
\item La cantidad de campos de un registro es acotado por lo tanto $O(|campos(regComb)|) \in O(1)$
\item $O(buscar(r,t,b)$ es $O(log(m) + L)$ pues el campo del join es un campo clave con indice. Y todos los registros que se agreguen al la lista de modificaciones tiene al dicho campo.
\item $A$ es la complejidad de combinarRegistros(c, cr1, cr2) la cual es $O(\#cr1*\#cr2*L*min\lbrace \#cr1, \#cr2\rbrace)$. Pero como $\#cr1 = \#cr2 = 1$. Entonces la complejidad queda O(1*1*L*1), entonces por algebra de ordenes queda O(L)
\item La complejidad de la funcion AuxActContenedor es O((log(m+n) + L) + (A*L)) pero como el campo del join es indice y clave. Entonces queda O(log(m+n) + 2L) pues (A) es la cantidad de elementos que puede haber en un algun Indice. Y, como el campo es clave, esa cantidad es 1.
\item La complejidad del algoritmo es : O(R*((log(m) + L) + log(m+n) + 2L))
\item Aplicando algebra de Ordenes queda : O(R*(log(m+n)*L))
\end{enumerate}} % Justificación


\begin{algoritmo}{iAuxActContenedor}{\Inout{cont}{estrCont}, \In{reg}{reg}, \In{mod}{estrMod}, \In{c}{string},\In{nat?}{bool}}{}
	\eIf(\com*[f]{$O(1)$}){nat?}{
 		$DatoNat \gets ValorN(Significado(reg, c))$ \com*{$O(|campos(regComb)|)$} 
        \eIf(\com*[f]{$O(log(n+m))$}){Definido?(cont.IndiceNat, DatoNat)}{
        	$ ConjIts \gets Significado(cont.IndiceNat, DatoNat) $ \com*{$O(log(n+m))$}
        	\eIf(\com*[f]{$O(1)$}){mod.Inserto?}{
	    	    $itRegComb \gets AgregarRapido(cont.Registros, reg) $ \com*{$O(1)$}
    	    	$AgregarRapido(ConjIts, $<$itRegComb,NULL$>$) $ \com*{$O(1)$}
			}{
				$AuxBorrarDelJoin(ConjIts,c,DatoNat,nat?)$ \com*{$O(1)$}
	        }
        }{
        	$ConjIts \gets Vacio() $ \com*{$O(1)$}
	        $DefinirRapido(cont.IndiceNat, DatoNat, ConjIts) $ \com*{$O(log(n+m))$}
    	    \If(\com*[f]{$O(1)$}){mod.Inserto?}{
    	    	$itRegComb \gets AgregarRapido(cont.Registros, reg) $ \com*{$O(1)$}
	    	    $AgregarRapido(ConjIts, $<$itRegComb,NULL$>$) $ \com*{$O(1)$}
	        }

		}
	}{
          $DatoString \gets ValorS(Significado(regComb, c))$ \com*{$O(|campos(regComb)|)$}
	      \eIf(\com*[f]{$O(log(n+m))$}){Definido?(contenedor.IndiceString, DatoString)}{
          $ ConjIts \gets Significado(contenedor.IndiceString, DatoString) $ \com*{$O(log(n+m))$}
          \eIf(\com*[f]{$O(1)$}){mod.Inserto?}{
              $itRegComb \gets AgregarRapido(contenedor.Registros, regComb) $ \com*{$O(1)$}
              $AgregarRapido(ConjIts, $<$itRegComb,NULL$>$) $ \com*{$O(1)$}
          }{
				$AuxBorrarDelJoin()$ \com*{$O(1)$}
          }
          }{
              $ConjIts \gets Vacio() $ \com*{$O(1)$}
              $DefinirRapido(contenedor.IndiceString, DatoString, ConjIts) $ \com*{$O(log(n+m))$}
              \If(\com*[f]{$O(1)$}){mod.Inserto?}{
                  $itRegComb \gets AgregarRapido(contenedor.Registros, regComb) $ \com*{$O(1)$}
                  $AgregarRapido(ConjIts, $<$itRegComb,NULL$>$) $ \com*{$O(1)$}
              }  		             	
          }
   
    }
\end{algoritmo}
\datosAlgoritmo{Use este algoritmo auxiliar porque actualizar vista join no entraba en la pagina}
{}
{}
{$O((log(n+m) + L) + (A)*L) $}
{\begin{enumerate}
\item Busca el registro en el indice que corresponda. O(log(m+n) + L)
\item Si la modificacion fue de Borrar. Borrar en los indices segun corresponda. O(A*L)
\item A es la cantidad de elementos en el significado de la clave del indice a borrar.
\item Finalmente la complejidad del algoritmo es O((log(m+n) + L) + (A)*L)
\end{enumerate}}

\begin{algoritmo}{iAuxBorrarDelJoin}{\Inout{its}{conj}, \In{c}{string}, \In{dato}{Dato}, \In{nat?}{bool}}{}
    $itConjIts \gets CrearIt(its) $ \com*{$O(1)$}
    \While(\com*[f]{$O(|ConjIts|)$}){HaySiguiente(itConjIts)}{
	    $ itReg \gets Siguiente(itConjIts).Reg $\com*{$O(1)$}
    	$ regB \gets Siguiente(itReg) $ \com*{$O(1)$}
	    $ datoR \gets Significado(regB, c) $ \com*{$O(|campos(regB)|$}
		\If(\com*[f]{$O(L)$}){(nat? $\wedge$ ValorN(datoR) = dato)) $\vee$ (ValorS(datoR) = dato)}{
	        $EliminarSiguiente(itReg) $ \com*{$O(1)$}
            $EliminarSiguiente(itConjIts, it) $ \com*{$O(1)$}
        }
		$ Avanzar(itConjIts) $ \com*{$O(1)$}
    }
\end{algoritmo}
\datosAlgoritmo{Use este algoritmo auxiliar porque actualizar vista join no entraba en la pagina}
{}
{}
{O(A*L)}
{\begin{enumerate}
\item A es la cantidad de elementos en el significado de la clave del indice a borrar, o sea $\#ConjIts$.
\end{enumerate}}
\begin{algoritmo}{iActualizarJoin}{\In{inserta}{bool},\In{r}{reg}, \In{t}{string}, \Inout{b}{estrBD}}{}
    $itTbl \gets CrearIt(b.NombresTablas)$ \com*{$O(1)$}
    $datosT \gets Obtener(b.Tablas, t)$ \com*{$O(|t|)$}
    \While(\com*[f]{$O(|NombreTablas|*(2*copy(reg)))$}){($HaySiguiente(itTbl)$)}{
    	$nombreT \gets Siguiente(itTbl)$ \com*{$O(1)$}
        \If(\com*[f]{$O(|nombreT|)$}){HayJoin?(t, nombreT)}{
        	$join \gets Obtener(datosT.Joins, nombreT) $ \com*{$O(|nombreT|)$}
            $tup \gets <$inserta$, $copiar(reg)$> $ \com*{$O(copy(reg))$}
	        $AgregarAdelante(join.modificaciones, tup) $ \com*{$O(copy(reg))$}
        }
        $Avanzar(itTbl) $ \com*{$O(1)$}        
    }
    $itTbl \gets CrearIt(b.NombresTablas)$ \com*{$O(1)$}
    \While(\com*[f]{$O(|NombreTablas|*(2*copy(reg)))$}){($HaySiguiente(itTbl)$)}{
    	$nombreT \gets Siguiente(itTbl)$ \com*{$O(1)$}
        \If(\com*[f]{$O(1)$}){HayJoin?(nombreT, t)}{
            $datosT \gets Obtener(b.Tablas, nombreT)$ \com*{$O(|nombreT|)$}
        	$join \gets Obtener(datosT.Joins, t) $ \com*{$O(|t|) $}
            $tup \gets <$inserta$, $copiar(reg)$> $ \com*{$O(copy(reg))$}
	        $AgregarAdelante(join.modificaciones, tup) $ \com*{$O(copy(reg))$}
        }
        $Avanzar(itTbl) $ \com*{$O(1)$}    
    }
\end{algoritmo}

\datosAlgoritmo{} % Descripción
{$t \in tablas(b) $} % Pre
{$ (\forall t_{1}, t_{2} \in tablas(b)) \ hayJoin?(t1,t2,b) \impluego vistaJoin(t1,t2,b) \equiv combinarRegistros(campoJoin(t1,t2,b), registros(t1), registros(t2)) $} % Post
{$O(n*L)$} % Complejidad
{\begin{enumerate}
	\item $n$ es la cantidad de Tablas de la base de datos.
    \item $L$ es la longitud maxima de algun dato string del registro a copiar
	\item Recorre los joins de la tabla. O(n)
    \item Para cada join de la tabla. Agrega una copia del registro. O(L)
    \item Por los 2 items anteriores el costo total del while es O(n*L)
    \item Luego, recorre todas las tablas de la base de datos. O(n)
    \item Luego para cada tabla se pregunta si existe un join con la tabla $t$. Como los nombre de las tablas estan acotados, esta operaci\'on es O(1), por su definici\'on
    \item Si hay un join, lo recupera, y como el nombre de las tablas es acotado y las tablas estan implementadas sobre un DiccString, el Obtener es O(1)
    \item Luego, genera la modificacion para eso copia el registro. Copiar un registro cuesta O(L)
    \item Por ultimo, lo agrega a la lista de modificaciones del join. Esta operacion tambien copia el registro por lo tanto cuesta O(L)
    \item Finalmente, por lo items anteriores, el segundo while cuesta $O(n*(1+1+L+L) \in O(n*L)$
    \item Sino hay join, no se hace nada.
    \item Luego, la complejidad total del algoritmo es $O(n*L + n*L) \in O(n*L)$
\end{enumerate}} % Justificación



\begin{algoritmo}{icantidadDeAccesos}{\In{t}{nombreTabla} \In{b}{estrBD}}{conj(reg)}

	$datosT \gets Obtener(b.Tablas,t)$ \com*{$O(|t|)$}
	$res \gets cantidadDeAccesos(datosT.Tabla)$ \com*{$O(1)$}
    
\end{algoritmo}

\datosAlgoritmo{} % Descripción
{} % Pre
{} % Post
{$O(1)$} % Complejidad
{$O(|t|) + O(1) = O(1)$ porque los nombres de las tablas estan acotados} % Justificación

\begin{algoritmo}{itablaMaxima}{\In{b}{estrBD}}{nombreTabla}%

	$res \gets b.tablaMaxima$ \com*{$O(1)$}
   
\end{algoritmo}

\datosAlgoritmo{} % Descripción
{} % Pre
{} % Post
{$O(1)$} % Complejidad
{} % Justificación

\begin{algoritmo}{iencontrarMaximo}{\In{t}{nombreTabla} \In{ct}{conj(nombreTabla)} \In{b}{estrBD}}{nombreTabla}%
		
	$itTbl \gets CrearIt(ct)$ \com*{$O(1)$}
	$res \gets t$ \com*{$O(1)$}
	\While(\com*[f]{$O(1)$}){($HaySiguiente(itTbl)$)}{
		$nombreT \gets Siguiente(itTbl)$ \com*{$O(1)$}
		\If(\com*[f]{$O(1)$}){cantidadDeAccesos(dameTabla(res)) $\leq$ cantidadDeAccesos(dameTabla(nombreT)) }{
            $res \gets nombreT$ \com*{$O(1)$}
        }
		$Avanzar(itTbl)$ \com*{$O(1)$}
    }
\end{algoritmo}
\datosAlgoritmo{} % Descripción
{} % Pre
{} % Post
{$O(\#(ct))$} % Complejidad
{} % Justificación

\begin{algoritmo}{ibuscar}{\In{criterio}{registro} \In{t}{nombreTabla} \In{b}{estrBD}}{conj(reg)}%

	$res \gets coincidenciasRap(dameTabla(t,b),criterio)$ \com*{$O(coincidenciasRap)$}
    
\end{algoritmo}

\datosAlgoritmo{} % Descripción
{} % Pre
{} % Post
{$O(coincidenciasRap)$} % Complejidad
{Hereda la complejidad de coincidenciasRap: \\ En el caso de que la tabla de nombre $t$ no tenga indices o que ningun campo de $criterio$ sea indice en $t$, coincidenciasRap debe recorrer linealmente los registros de $t$ teniendo la complejidad de coincidencias: $O(n*L)$ \\ Si alguno de los campos de $criterio$ es indice solo debe buscar las coincidencias en el conjunto de registros que coincidan en el dato del campo indexado, los cuales obtengo realizando un $"Obtener"$ de ese dato en un diccStr o en un diccNat. Si alguno de los campos de $criterio$ ademas de ser indice es clave en t puede haber a lo sumo un registro en t que coincide con el valor de ese campo en $criterio$ por lo que al buscar los registros a traves del campo indexado voy a obtener un conjunto de 1 elemento por lo que la complejidad sera solamente la del $Obtener$ de los diccionarios: $O(log(n) + L)$} % Justificación

	
\end{Algoritmos}
    
    
    

\section{M\'odulo Diccionario Rapido Nat}

\Encabezado{Notas Preliminares}
  En todos los casos, al indicar las complejidades de los algoritmos, las variables que se utilizan corresponden a:
  \vspace{-0.5em}\begin{itemize}
    \item $n$: N\'umero de claves en el Diccionario Rapido Nat pasado por par\'ametro.

  \end{itemize}
  

%%%%%%%%%%%%%%%%%%%%%%%%%%

\subsection{Interfaz}
  
  \textbf{parametros formales} \hangindent=2 \parindent \\
  \parbox{1.7cm}{\textbf{generos}} $\alpha$
  
  \textbf{se explica con}: \tadNombre{dicc(nat, $\alpha$)}.
 
  \textbf{generos}: \TipoVariable{diccNat(nat, $\alpha$)}.
  
   \servUsados{nat}

  \subsubsection{Operaciones basicas}
  
  \InterfazFuncion{Vacio{}}{}{diccNat(nat, $\alpha)$}
  [true]
  {$res$ $\igobs$ vacio}%
  [$O(1)$]
  [Crea un Diccionario Vacio]
  []

  ~

  \InterfazFuncion{Definido?}{\In{n}{nat}, \In{d}{diccNat(nat, $\alpha$)})}{bool}
 [true]
{$res$ $\igobs$ def?($n$, $d$)}
[$O(n)$ en el peor caso, el caso promedio si las claves se insertan con distribucion uniforme es O(lg n)]
[Retorna si la Clave esta definida]
[]
  
  ~

  \InterfazFuncion{Definir}{\In{n}{nat} , \In{a}{$\alpha$}, \Inout{d}{diccNat(nat, $\alpha)$}}{}
  [$ d \igobs d_0 $]
  {$ d \igobs$ definir($n, a, d_0$)}
  [$O(n + copy(a))$ en el peor caso, el caso promedio si las claves se insertan con distribucion uniforme es O(lg n + copy(a))]
  [Define la Clave $n$ y el Significado $a$ en $d$]
  []
  
  ~
  \InterfazFuncion{Obtener}{\In{n}{nat}, \In{d}{diccNat(nat, $\alpha$)}}{$\alpha$}
  [def?($n$, $d$)]
  {alias($res$ $\igobs$ obtener($n$, $d$))}
  [$O(n)$ en el peor caso, el caso promedio si las claves se insertan con distribucion uniforme es O(lg n)]
  [Retorna el Significado de la Clave $n$]
  [$res$ es modificable si y solo si $d$ es modificable]

  ~
  
    \InterfazFuncion{Borrar}{\Inout{d}{diccNat(nat, $\alpha$)}, \In{n}{nat}}{}
  [($d$ = $d_0$) $\wedge$ def?($s$, $d$)]
  {d $\igobs$ borrar($d_0$ , s)}
  [$O(n)$ en el peor caso, el caso promedio si las claves se insertan con distribucion uniforme es O(lg n)]
  [Borra la clave n con su significado.]
  []
  
    
 ~
 
 \InterfazFuncion{minimaClave}{\In{d}{diccNat(nat, $\alpha$)}}{nat}
  [Cardinal(claves(d)) > 0]
  {res $\igobs$ minimo(claves(d))}
  [$O(n)$ en el peor caso, el caso promedio si las claves se insertan con distribucion uniforme es O(lg n)]
  [Retorna la minima clave de mi diccionario]
  [$res$ es modificable si y solo si $d$ es modificable]
  
       \InterfazFuncion{maximaClave}{\In{d}{diccNat(nat, $\alpha$)}}{nat}
  [Cardinal(claves(d))]
  {res $\igobs$ maximo(claves(d))}
  [$O(n)$ en el peor caso, el caso promedio si las claves se insertan con distribucion uniforme es O(lg n)]
  [Retorna la maxima clave de mi diccionario]
  [$res$ es modificable si y solo si $d$ es modificable]
 
%%%%%%%%%%%%%%%%%%%%%%%%%%

\subsection{Representacion}

	\begin{Estructura}{diccNat$(nat, \alpha)$}[estrDN donde estrDN es puntero(nodo)]
		\begin{Tupla}[nodo]
        	\tupItem{clave}{nat}
			\tupItem{significado}{$\alpha$}
			\tupItem{izq}{puntero(nodo)}
            \tupItem{der}{puntero(nodo)}
		\end{Tupla}
	\end{Estructura}

\subsubsection{Invariante de Representacion}
\begin{enumerate}
\item Para todo nodo del ABB, el valor de la clave de su hijo izquierdo es menor que la clave de ese nodo, y la clave del hijo derecho es mayor a la clave de ese nodo
%\item No hay ninguna rama del ABB que tenga Ciclos, osea que cada rama siempre tiene un nodo final que sus 2 hijos apuntan  a NULL.%
\end{enumerate}

\Rep[estrDN][e]{
	respetaMayMen(e)  
  }

\textbf{Funciones Auxiliares:}
\tadAlinearFunciones{respetaMayMen}{estrDN e}
\tadOperacion{respetaMayMen}{estrDN e}{bool}{}
 \tadAxioma{respetaMayMen(e)}{$
e \neq NULL \impluego (e.izq \neq NULL \impluego ((e.clave.izq \leq e.clave) \wedge respetaMayMen(e.izq)) \wedge \\
								(e.der \neq NULL \impluego ((e.clave \leq e.der.clave) \wedge respetaMayMen(e.der))
              $ }

% \tadOperacion{EsABValido?}{estrDN /$e$}{bool}{}
% \tadAxioma{EsABValido?(e)}{$\textbf{if} \ HayLoop?(e) \ \emph{then} \\ \hspace*{10px} false \\ \textbf{else} \\ \hspace*{10px} NoHayRepetidos(DamePunteros(e)) \\  \textbf{fi}$}


% \tadOperacion{HayLoop?}{estrDN /$e$}{bool}{}
% \tadAxioma{HayLoop?(e)}{loopea(e, Ag(e, $\emptyset$))}

% \tadOperacion{loopea}{estrDN /$e$, conj(estrDN) /$ce$}{bool}{}
% \tadAxioma{loopea(c, ce)}{$\textbf{if} \ e = NULL \ \emph{then} \\ \hspace*{10px} false \\ \textbf{else} \\ \hspace*{10px}  \textbf{if} \ (e.izq \in c) \vee (e.der \in c)\ \emph{then} \\ \hspace*{20px} true \\ \hspace*{10px} \textbf{else} \\ \hspace*{20px} loopea(e.izq, Ag(e.izq, c) \wedge loopea(e.der, Ag(e.der, c) \\  \hspace*{10px} \textbf{fi} \\  \textbf{fi}$}


% \tadOperacion{DamePunteros}{estrDN /$e$}{multiconj(estrDN)}{}
% \tadAxioma{DamePunteros(e)}{$\textbf{if} \ e = NULL \ \emph{then} \\ \hspace*{10px} \emptyset \\ \textbf{else} \\ \hspace*{10px} Ag(e, (DamePunteros(e.izq) \cup (DamePunteros(e.der) )) \\  \textbf{fi}$}
 
%  \newpage
% \tadOperacion{NoHayRepetidos}{multiconj($\alpha$) /$m$}{bool}{}
% \tadAxioma{NoHayRepetidos(m)}{$\textbf{if} \ \emptyset ?(m) \ \emph{then} \\ \hspace*{10px} true \\ \textbf{else} \\ \hspace*{10px} \textbf{if} \ \# (dameUno(m), m) \neq 1 \ \emph{then} \\ \hspace*{20px} false \\ \hspace*{10px} \textbf{else} \\ \hspace*{20px} NoHayRepetidos(sinUno(m)) \\  \hspace*{10px} \textbf{fi} \\  \textbf{fi}$}


\subsubsection{Funcion de Abstraccion}

\Abs[estrDN]{dicc(nat, $\alpha$)}[e]{d}{($\forall c$: nat)(def?($c, d$) $=$ esClave?($c, e$) $\yluego$ \\
 (def?($c, d$) \impluego obtener($c, d$) $=$ significado($c, e$)))}

\textbf{Funciones Auxiliares:}

\tadOperacion{claves}{estrDN /$e$}{conj(nat)}{}
\tadAxioma{claves(e)}{$\textbf{if} \ e = NULL \ \emph{then} \\ \hspace*{10px} \emptyset \\ \textbf{else} \\ \hspace*{10px} Ag(e.clave, (claves(e.izq) \cup claves(e.der))) \\  \textbf{fi}$}

\tadOperacion{minimo}{conj(nat) /$c$}{nat}{$Cardinal(c) > 0$}
\tadAxioma{minimo(c)}{min(dameUno(c), minimo(sinUno(c)))}

%$\textbf{if} \ *** \ \emph{then} \\ \hspace*{10px} *** \\ \textbf{else} \\ \hspace*{10px} *** \\  \textbf{fi}$

\tadOperacion{esClave?}{nat /$c$, estrDN /$e$}{bool}{Rep($e$)}
\tadAxioma{esClave?(c,e)}{ $\textbf{if} \ e = NULL \ \emph{then} $\\$ \hspace*{10px} false $\\$ \textbf{else} $\\$ \hspace*{10px} \textbf{if} \ e.clave = c \ \emph{then} $\\$ \hspace*{20px} true $\\$ \hspace*{10px} \textbf{else} $\\$ \hspace*{20px}  esClave?(c,e.izq) \vee esClave?(c,e.der)$\\$ \hspace*{10px} \textbf{fi} $\\$ \textbf{fi}$ }




\tadOperacion{significado}{string /$c$, estrDN /$e$}{bool}{esClave?($c, e$) $\wedge$ Rep($e$)}
\tadAxioma{significado(c,e)}{ $\textbf{if} \ e.clave = c \ \emph{then} \\ \hspace*{10px} e.significado \\ \textbf{else} \\ \hspace*{10px} \textbf{if} \ e.clave > c \ \emph{then} $\\$ \hspace*{20px} significado(e.izq) $\\$ \hspace*{10px} \textbf{else} \\ \hspace*{20px}  esignificado(e.der)$\\$ \hspace*{10px} \textbf{fi} \\  \textbf{fi}$}

%%%%%%%%%%%%%%%%%%%%%%%%%%%%%%%%%

\begin{Algoritmos}

\begin{algoritmo}{\textbf{iVacio}}{}{estrDN}
			res $\leftarrow$ NULL \com*{$O(1)$}    	
\end{algoritmo}
\datosAlgoritmo{} % Descripción
  {} % Pre
  {} % Post
  {$O(1)$} % Complejidad
  {Se ejecuta una sola operacion con costo $O(1)$} % Justificación


\begin{algoritmo}{\textbf{iDefinido?}}{\In{n}{nat}, \In{d}{estrDN})}{bool}
			actual $\leftarrow$ d \com*{$O(1)$}
            \While(\com*[f]{$O(n)$}){actual $\neq$ NULL \yluego actual.clave $\neq$ n}{
            	\eIf(\com*[f]{$O(1)$}){actual.clave > n}{
                	actual $\leftarrow$ actual.izq \com*{$O(1)$}
                }{
                	actual $\leftarrow$ actual.der \com*{$O(1)$}
                }
            }
            res $\leftarrow$ (actual $\neq$ NULL) \com*{$O(1)$}	
\end{algoritmo}
\datosAlgoritmo{} % Descripción
  {} % Pre
  {} % Post
  {$O(n)$} % Complejidad
  {Dado que el resto de las operaciones son elementales con costo $O(1)$, la unica operacion que me suma complejidad es lo que tarde el ciclo de la linea 2 en recorrer las claves hasta llegar o no a la que esta buscando. El peor caso que tengo es si debo buscar una clave en un diccNat al cual se fueron agregando las claves ordenadas (de menor a mayor o de mayor a menor) por lo que para buscar una clave que es un maximo o un minimo debo recorrer todas las claves del arbol en el $while$ de la linea 2 por lo que mi complejidad sera $O(n)$. En el caso de que las claves que se encuentran en el ABB hayan sido insertadas de manera uniforme mi ABB se asemejara a un arbol balanceado. La propiedad que tienen estos arboles es que su altura es lg(n), por lo que si mi ABB es similar a este caso, para buscar si una clave esta definida, en promedio, para recorrer una rama a lo sumo voy a recorrer lg(n) claves (la altura promedio de las ramas de mi arbol). Por lo tanto si las claves fueron insertadas de manera uniforme la complejidad promedio de Definido? es $O(lg(n))$ } % Justificación

\begin{algoritmo}{\textbf{iDefinir}}{\In{n}{nat} , \In{a}{$\alpha$}, \Inout{d}{estrDN}}{}
			actual $\leftarrow$ d \com*{$O(1)$}
            \While(\com*[f]{$O(n)$}){actual $\neq$ NULL \yluego actual.clave $\neq$ n}{
            	\eIf(\com*[f]{$O(1)$}){actual.clave > n}{
                	actual $\leftarrow$ actual.izq \com*{$O(1)$}
                }{
                	actual $\leftarrow$ actual.der \com*{$O(1)$}
                }
                }
            \eIf(\com*[f]{$O(1)$}){actual = NULL}{    
            actual $\leftarrow$ <n, a, NULL, NULL> \com*{$O(copy(a))$}           
            }{
            actual.significado $\leftarrow$ a \com*{$O(copy(a))$}
            }
\end{algoritmo}
\datosAlgoritmo{} % Descripción
  {} % Pre
  {} % Post
  {$O(n + copy(a))$} % Complejidad
  {Dado que el resto de las operaciones son elementales con costo $O(1)$, la unica operacion que me suma complejidad es lo que tarde el ciclo de la linea 2 en recorrer las claves hasta llegar al lugar donde debo definir la clave. El peor caso que tengo es si debo insertar una clave en un diccNat al cual se fueron agregando las claves ordenadas (de menor a mayor o de mayor a menor) por lo que para insertar una clave que es un maximo o un minimo debo recorrer todas las claves del arbol en el $while$ de la linea 2, por lo que mi complejidad sera $O(n)$. En el caso de que las claves que ya se encuentran en el ABB hayan sido insertadas de manera uniforme mi ABB se asemejara a un arbol balanceado. La propiedad que tienen estos arboles es que su altura promedio es lg(n), por lo que si mi ABB es similar a este caso, para buscar el nodo donde debo insertar mi clave o modificar el significado de la clave existente, en promedio, para recorrer una rama voy a pasar por lg(n) claves (la altura promedio de las ramas de mi arbol). Por lo tanto si las claves fueron insertadas de manera uniforme la complejidad promedio de Definir es $O(lg(n))$ } % Justificación

\begin{algoritmo}{iObtener}{\In{n}{nat}, \In{d}{estrDN}}{$\alpha$}
			actual $\leftarrow$ d \com*{$O(1)$}
            \While(\com*[f]{$O(n)$}){actual $\neq$ NULL \yluego actual.clave $\neq$ n}{
            	\eIf(\com*[f]{$O(1)$}){actual.clave > n}{
                	actual $\leftarrow$ actual.izq \com*{$O(1)$}
                }{
                	actual $\leftarrow$ actual.der \com*{$O(1)$}
                }
                }
            res $\leftarrow$ actual.significado \com*{$O(1)$}
\end{algoritmo}
\datosAlgoritmo{} % Descripción
  {} % Pre
  {} % Post
  {$O(n)$} % Complejidad
  {Dado que el resto de las operaciones son elementales con costo $O(1)$, la unica operacion que me suma complejidad es lo que tarde el ciclo de la linea 2 en recorrer las claves hasta llegar al lugar donde se encuentra la clave de la cual debo devolver su significado, esa clave se encuentra definida en mi diccionario por la precondicion. El peor caso que tengo es si debo buscar una clave en un diccNat al cual se fueron agregando las claves ordenadas (de menor a mayor o de mayor a menor) por lo que para buscar una clave que es un maximo o un minimo debo recorrer todas las claves del arbol en el $while$ de la linea 2 por lo que mi complejidad sera $O(n)$. En el caso de que las claves que ya se encuentran en el ABB hayan sido insertadas de manera uniforme mi ABB se asemejara a un arbol balanceado. La propiedad que tienen estos arboles es que su altura promedio es lg(n), por lo que si mi ABB es similar a este caso, para buscar el nodo donde se encuentra la clave que busco, en promedio, para recorrer una rama voy a pasar por lg(n) claves (la altura promedio de las ramas de mi arbol). Por lo tanto si las claves fueron insertadas de manera uniforme la complejidad promedio de Obtener es $O(lg(n))$} % Justificación
\begin{algoritmo}{iBorrar}{\Inout{d}{estrDN}, \In{n}{nat}}{}
			actual $\leftarrow$ d \com*{$O(1)$}
            padre $\leftarrow$ NULL \com*{$O(1)$}
            \While(\com*[f]{$O(n)$}){actual.clave $\neq$ n}{
            	padre $\leftarrow$ actual \com*{$O(1)$}
            	\eIf(\com*[f]{$O(1)$}){actual.clave > n}{
                	actual $\leftarrow$ actual.izq \com*{$O(1)$}
                }{
                	actual $\leftarrow$ actual.der \com*{$O(1)$}
            }
            }

	\eIf(\com*[f]{$O(1)$}){actual.izq = NULL}{
    	\eIf(\com*[f]{$O(1)$}){actual.der = NULL}{
    		\eIf(\com*[f]{$O(1)$}){padre.izq = actual}{
            	padre.izq $\leftarrow$ NULL \com*{$O(1)$}
                actual $\leftarrow$ NULL \com*{$O(1)$}
            }{
             	padre.der $\leftarrow$ NULL \com*{$O(1)$}
                actual $\leftarrow$ NULL \com*{$O(1)$}         
            }
    	}{
    		\eIf(\com*[f]{$O(1)$}){padre.izq = actual}{
            	padre.izq $\leftarrow$ actual.der \com*{$O(1)$}
                actual $\leftarrow$ NULL \com*{$O(1)$}
            }{
             	padre.der $\leftarrow$ actual.der \com*{$O(1)$}
                actual $\leftarrow$ NULL \com*{$O(1)$}        
            }        	
        }
    }{
    	\eIf(\com*[f]{$O(1)$}){actual.der = NULL}{
    		\eIf(\com*[f]{$O(1)$}){padre.izq = actual}{
            	padre.izq $\leftarrow$ actual.izq \com*{$O(1)$}
                actual $\leftarrow$ NULL \com*{$O(1)$}
            }{
             	padre.der $\leftarrow$ actual.izq \com*{$O(1)$}
                actual $\leftarrow$ NULL \com*{$O(1)$}      
            }               
        }{
   			
				maximo $\leftarrow$ actual \com*{$O(1)$}
                padreMax $\leftarrow$ padre \com*{$O(1)$}
                \While(\com*[f]{$O(n)$}){maximo.der $\neq$ NULL}{
                	padreMax $\leftarrow$ maximo \com*{$O(1)$}
                	maximo = maximo.der \com*{$O(1)$}
                }
                valBorr $\leftarrow$ actual.clave \com*{$O(1)$}
                actual.clave $\leftarrow$ maximo.clave \com*{$O(1)$}
                actual.significado $\leftarrow$ maximo.significado \com*{$O(1)$}
                maximo.clave $\leftarrow$ valBorr \com*{$O(1)$}
                \eIf(\com*[f]{$O(1)$}){maximo.izq = NULL}{
                	padreMax $\leftarrow$ NULL \com*{$O(1)$}
                    maximo $\leftarrow$ NULL \com*{$O(1)$}
                }{
                	padreMax.der $\leftarrow$ maximo.izq \com*{$O(1)$}
                    maximo $\leftarrow$ NULL \com*{$O(1)$}
                }
                
                     	
        }
    }

\end{algoritmo}
\datosAlgoritmo{} % Descripción
  {} % Pre
  {} % Post
  {$O(n)$} % Complejidad
  {Dado que el resto de las operaciones son de costo $O(1)$, las unicas operaciones que me suman complejidad son los dos ciclos de las lineas 3 y 41. El ciclo de la linea 3 recorre las claves hasta llegar a la que quiero borrar, la clave va a estar definida en mi diccionario por la precondicion . El peor caso que tengo es si debo borrar una clave en un diccNat al cual se fueron agregando las claves ordenadas (de menor a mayor o de mayor a menor) por lo que para borrar una clave que es un maximo o un minimo debo recorrer todas las claves del arbol en el $while$ de la linea 3 por lo que mi complejidad sera $O(n)$. Mi otro peor caso sucede tambien si mis claves fueron insertadas de forma ordenada y si el nodo donde se encuentra la clave a borrar tiene hijo derecho e hijo izquierdo, en este caso buscar esa clave me costara menos que $O(n)$ ya que no recorro toda una rama ,ya que sino el nodo no tendria 2 hijos, pero al tener que buscar el maximo de ese subarbol para intercambiarlos y poder borrar de forma correcta debo recorrer el resto de una rama por lo que mi complejidad tambien sera $O(n)$ . \\ Por otro lado, en el caso de que las claves que se encuentran en el ABB hayan sido insertadas de manera uniforme mi ABB se asemejara a un arbol balanceado. La propiedad que tienen estos arboles es que su altura es lg(n), por lo que si mi ABB es similar a este caso, para buscar una clave , en promedio, voy a recorrer lg(n) claves (la altura promedio de las ramas de mi arbol). En el caso de tener que borrar un nodo con 0 o 1 hijo solo debo recorrer el ABB hasta encontrarlo y borrarlo lo que me terminara costando a lo sumo $O(lg(n))$ (la altura promedio de mi arbol). Si debo borrar un nodo con 2 hijos, encontrarlo me va a costar en promedio menos que $O(lg(n))$ porque no estoy recorriendo una rama hasta el final ya que sino no tendria 2 hijos, pero buscar el maximo para swapearlo con el nodo a borrar me llevara a terminar de recorrer una rama por lo que tambien me terminara costando $O(lg(n))$ en promedio.Por lo tanto si las claves fueron insertadas de manera uniforme la complejidad promedio de borrar es $O(lg(n))$} % Justificación

\begin{algoritmo}{\textbf{iminimaClave}}{\In{d}{estrDN}}{nat}
			min $\leftarrow$ NULL \com*{$O(1)$}
            actual $\leftarrow$ d \com*{$O(1)$}
            \While(\com*[f]{$O(n)$}){d $\neq$ NULL}{
            	min $\leftarrow$ actual \com*{$O(1)$}
                actual $\leftarrow$ actual.izq \com*{$O(1)$}
            }
            res $\leftarrow$ min \com*{$O(1)$}
\end{algoritmo}
\datosAlgoritmo{} % Descripción
  {} % Pre
  {} % Post
  {$O(n)$} % Complejidad
  {Mi peor caso sucede si las claves del diccionario fueron insertadas de mayor a menor ya que para encontrar el minimo voy a recorrer todas las claves tardando $O(n)$. Si mis claves fueron insertadas de manera uniforme mi arbol se asemejara a un arbol balanceado por lo que recorrer una rama me llevara $O(lg(n))$ por lo que la complejidad de minimaClave sera $O(lg(n))$} % Justificación
  
\begin{algoritmo}{\textbf{imaximaClave}}{\In{d}{estrDN}}{nat}
			max $\leftarrow$ NULL \com*{$O(1)$}
            actual $\leftarrow$ d \com*{$O(1)$}
            \While(\com*[f]{$O(n)$}){d $\neq$ NULL}{
            	max $\leftarrow$ actual \com*{$O(1)$}
                actual $\leftarrow$ actual.der
            }
            res $\leftarrow$ max \com*{$O(1)$}
\end{algoritmo}
\datosAlgoritmo{} % Descripción
  {} % Pre
  {} % Post
  {$O(n)$} % Complejidad
  {Mi peor caso sucede si las claves del diccionario fueron insertadas de menor a mayor ya que para encontrar el maximo voy a recorrer todas las claves tardando $O(n)$. Si mis claves fueron insertadas de manera uniforme mi arbol se asemejara a un arbol balanceado por lo que recorrer una rama me llevara $O(lg(n))$ por lo que la complejidad de maximaClave sera $O(lg(n))$} % Justificación





\end{Algoritmos}


















\section{Modulo DiccionarioString($\alpha $)}

\Encabezado{Notas Preliminares}
  En todos los casos, al indicar las complejidades de los algoritmos, las variables que se utilizan corresponden a:
  \vspace{-0.5em}\begin{itemize}
    \item $L$: Longitud del string en el Diccionario String pasado por par\'ametro.

  \end{itemize}
  

%%%%%%%%%%%%%%%%%%%%%%%%%%

\subsection{Interfaz}
  
  \textbf{parametros formales} \hangindent=2 \parindent \\
  \parbox{1.7cm}{\textbf{generos}} $\alpha$
  
  \textbf{se explica con}: \tadNombre{dicc(string, $\alpha$)}.
 
  \textbf{generos}: \TipoVariable{diccStr(string, $\alpha$)}.
  
   \servUsados{string}

\subsubsection{Operaciones basicas}
  
  \InterfazFuncion{Vacio{}}{}{diccStr(string, $\alpha)$}
  [true]
  {$res$ $\igobs$ vacio}%
  [$O(1)$]
  [Crea un Diccionario Vacio]
  []

  \InterfazFuncion{Definido?}{\In{s}{string}, \In{d}{diccStr(string, $\alpha$)})}{bool}
  [true]
	{$res$ $\igobs$ def?($s$, $d$)}
	[$O(Longitud(s))$]
	[Retorna si la Clave esta definida]
	[]
  

  \InterfazFuncion{Definir}{\In{s}{string} , \In{a}{$\alpha$}, \Inout{d}{diccStr(string, $\alpha)$}}{}
  [$ d \igobs d_0 $]
  {$ d \igobs$ definir($s, a, d_0$)}
  [$O$($Longitud(s)$ + copy(a))]
  [Define la Clave $s$ y el Significado $a$ en $d$]
  []
  
  \InterfazFuncion{Obtener}{\In{s}{string}, \In{d}{diccStr(string, $\alpha$)}}{$\alpha$}
  [def?($s$, $d$)]
  {alias($res$ $\igobs$ obtener($s$, $d$))}
  [$O(Longitud(s))$]
  [Retorna el Significado de la Clave $s$]
  [$res$ es modificable si y solo si $d$ es modificable]
  
    \InterfazFuncion{Borrar}{\Inout{d}{diccStr(string, $\alpha$)}, \In{c}{string}}{}
  [($d$ = $d_0$) $\wedge$ def?($s$, $d$)]
  {d $\igobs$ borrar($d_0$ , s)}
  [$O(Longitud(s))$]
  [Retorna el Significado de la Clave s]
  [Elimina la clave c y su significado de d]
 
  \InterfazFuncion{minimaClave}{\In{d}{diccStr(string, $\alpha$)}}{string}
  [Cardinal(claves(d)) > 0]
  {res $\igobs$ minimo(claves(d))}
  [$O(L)$]
  [Retorna la minima clave de mi diccionario]
  [$res$ es modificable si y solo si $d$ es modificable]
  
  ~
  
  \InterfazFuncion{maximaClave}{\In{d}{diccStr(string, $\alpha$)}}{string}
  [Cardinal(claves(d))]
  {res $\igobs$ maximo(claves(d))}
  [$O(L)$]
  [Retorna la maxima clave de mi diccionario]
  [$res$ es modificable si y solo si $d$ es modificable]
 
%%%%%%%%%%%%%%%%%%%%%%%%%%

\subsection{Representacion}

	\begin{Estructura}{diccStr$(string, \alpha)$}[dStr donde dStr es puntero(nodo)]
		\begin{Tupla}[nodo]
			\tupItem{significado}{puntero$(\alpha)$}
			\tupItem{prefijos}{arreglo[256] (puntero(nodo))}
		\end{Tupla}
	\end{Estructura}

\subsubsection{Invariante de Representacion}

\begin{enumerate}
\item La raiz representa al prefijo $" "$, su significado es siempre NULL
%\item Todas las posiciones de prefijos estan definidas (apuntan a NULL o a otro nodo)
\item No hay ninguna rama del trie que tenga Ciclos
\item Todas las hojas tienen su significado distinto a NULL, salvo que el arbol este vacio.


\end{enumerate}

\Rep[dStr][d]{\\
	(d.significado = NULL) $\wedge$ \\
    %todosDefinidos(d) $\yluego$ \\
	esArbValido(d) $\yluego$ \\
   % ($\forall$ (p $\in$ damePunteros(d))) p.vive = false $\impluego$ hijosNoViven(p, 256))
   $\textbf{if} \ esHoja?(d) \ \emph{then} \\ \hspace*{10px} true \\ \textbf{else} \\ \hspace*{10px} HojasConSignificado(d) \\  \textbf{fi}$
  }


\textbf{Funciones Auxiliares:}
% \tadOperacion{hijosNoViven}{puntero(nodo) /$p$, nat /$n$}{bool}{}
% \tadAxioma{hijosNoViven(p, n)}{$\textbf{if} \ p = NULL \ \emph{then} \\ \hspace*{10px} true \\ \textbf{else} \\ \hspace*{10px} \textbf{if} \ n = 0 \ \emph{then} \\ \hspace*{20px} true \\ \hspace*{10px} \textbf{else} \\ \hspace*{20px}  \\ (p.prefijos[n-1].vive = false $\yluego$ hijosNoViven(p.prefijos[n-1], 256)) $\yluego$ hijosNoViven(p, n-1 ) \textbf{fi} \\  \textbf{fi}$ }



% \tadOperacion{todosDefinidos}{dStr /$d$}{bool}{}
% \tadAxioma{todosDefinidos(d)}{$\textbf{if} \ d = NULL \ \emph{then} \\ \hspace*{10px} true \\ \textbf{else} \\ \hspace*{10px} todosDefinidosAux(d.prefijos, 256) \yluego todosHijosDefinidos(d, 256) \\  \textbf{fi}$}

% \tadOperacion{todosHijosDefinidos}{dStr /$d$, nat /$n$}{bool}{d $\neq$ NULL $\yluego$ $n = tam(d.prefijos)$}
% \tadAxioma{todosHijosDefinidos(d,n)}{$\textbf{if} \ n = 0 \ \emph{then} \\ \hspace*{10px} true \\ \textbf{else} \\ \hspace*{10px} todosDefinidos(d.prefijos[n-1] \yluego todosHijosDefinidos(d.prefijos, n-1) \\  \textbf{fi}$}

% \tadOperacion{todosDefinidosAux}{ad($\alpha$) /$a$, nat /$n$}{bool}{$n = tam(a)$}
% \tadAxioma{todosDefinidosAux(a,n)}{$\textbf{if} \ n = 0 \ \emph{then} \\ \hspace*{10px} true \\ \textbf{else} \\ \hspace*{10px} definido?(a, n-1) \yluego todosDefinidosAux(a, n-1) \\  \textbf{fi}$}
\tadAlinearFunciones{TodasMisHojasConSignificado}{ad($\alpha$) /$a$, nat /$n$, conj($\alpha$) /$c$}
\tadAlinearAxiomas{TodasMisHojasConSignificado(d,n)}
\tadOperacion{Loopea}{dStr /$d$, conj(dStr) /$cd$}{bool}{}
\tadAxioma{Loopea(d, cd)}{$\textbf{if} \ d = NULL \ \emph{then} \\ \hspace*{10px} false \\ \textbf{else} \\ \hspace*{10px} \textbf{if} \ perteneceAlguno (d.prefijos, 256, cd) \ \emph{then} \\ \hspace*{20px} true \\ \hspace*{10px} \textbf{else} \\ \hspace*{20px} loopeaAlguno(d.prefijos, 256, cd) \\ \hspace*{10px}  \textbf{fi} \\  \textbf{fi}$}

\tadOperacion{perteneceAlguno}{ad($\alpha$) /$a$, nat /$n$, conj($\alpha$) /$c$}{bool}{($n = tam(a)$) $\yluego$ todosDefinidos(a,n)}
\tadAxioma{perteneceAlguno(a, n, c)}{$\textbf{if} \ n = 0 \ \emph{then} \\ \hspace*{10px} false \\ \textbf{else} \\ \hspace*{10px} (a[n-1] \in c) \yluego perteneceAlguno(a, n-1, c) \\  \textbf{fi}$}

\tadOperacion{loopeaAlguno}{ad($\alpha$) /$a$, nat /$n$, conj($\alpha$) /$c$}{bool}{$n = tam(a)$) $\yluego$ todosDefinidos(a,n)}
\tadAxioma{loopeaAlguno(a, n, c)}{$\textbf{if} \ n = 0 \ \emph{then} \\ \hspace*{10px} false \\ \textbf{else} \\ \hspace*{10px} Loopea(a[n-1], Ag(a[n-1],c)) \yluego loopeaAlguno(a, n-1, c) \\  \textbf{fi}$}

\tadOperacion{HayLoop?}{dStr /$d$}{bool}{}
\tadAxioma{HayLoop?(d)}{Loopea(d, Ag(d, $\emptyset$))}

\tadOperacion{damePunteros}{dStr /$d$}{multiconj(dStr)}{}
\tadAxioma{damePunteros(d)}{$\textbf{if} \ d = NULL \ \emph{then} \\ \hspace*{10px} \emptyset \\ \textbf{else} \\ \hspace*{10px} Ag(d, damePunterosdeTodos(d.prefijos, 256)) \\  \textbf{fi}$}

\tadOperacion{damePunterosdeTodos}{ad($\alpha$) /$a$, nat /$n$}{multiconj(dStr)}{$n = tam(a)$) $\yluego$ todosDefinidos(a,n)}
\tadAxioma{damePunterosdeTodos(a,n)}{$\textbf{if} \ n = 0 \ \emph{then} \\ \hspace*{10px} \emptyset \\ \textbf{else} \\ \hspace*{10px} damePunteros(a[n-1]) \cup damePunterosdeTodos(d.prefijos, n-1) \\  \textbf{fi}$}

\tadOperacion{NoHayRepetidos}{multiconj($\alpha$) /$m$}{bool}{}
\tadAxioma{NoHayRepetidos(m)}{$\textbf{if} \ \emptyset ?(m) \ \emph{then} \\ \hspace*{10px} true \\ \textbf{else} \\ \hspace*{10px} \textbf{if} \ \# (dameUno(m), m) \neq 1 \ \emph{then} \\ \hspace*{20px} false \\ \hspace*{10px} \textbf{else} \\ \hspace*{20px} NoHayRepetidos(sinUno(m)) \\  \hspace*{10px} \textbf{fi} \\  \textbf{fi}$}

\tadOperacion{esArbValido}{dStr /$d$}{bool}{}
\tadAxioma{esArbValido(d)}{$\textbf{if} \ HayLoop?(d) \ \emph{then} \\ \hspace*{10px} false \\ \textbf{else} \\ \hspace*{10px} NoHayRepetidos(DamePunteros(d)) \\  \textbf{fi}$}


\tadOperacion{HojasConSignificado}{dStr /$d$}{bool}{}
\tadAxioma{HojasConSignificado(d)}{$\textbf{if} \ esHoja?(d) \ \emph{then} \\ \hspace*{10px} d.significado \neq NULL \\ \textbf{else} \\ \hspace*{10px} TodasMisHojasConSignificado(d, 256) \\  \textbf{fi}$}

\tadOperacion{TodasMisHojasConSignificado}{dStr /$d$, nat /$n$}{bool}{}
\tadAxioma{TodasMisHojasConSignificado(d,n)}{$\textbf{if} \ n == 0 \ \emph{then} \\ \hspace*{10px} true \\ \textbf{else} \\ \hspace*{10px}   \textbf{if} \ d.prefijos[n-1] == NULL \ \emph{then} \\ \hspace*{20px} true \\ \hspace*{10px} \textbf{else} \\ \hspace*{20px} HojasConSignificado(d.prefijos[n-1])  \yluego  \\ \hspace*{20px} TodasMisHojasConSignificado(d, n-1) \\  \hspace*{10px} \textbf{fi} \\ \textbf{fi}$}

\tadOperacion{esHoja?}{dStr /$d$}{bool}{d $\neq$ NULL}
\tadAxioma{esHoja?(d)}{HijosNull(d,256)}

\tadOperacion{HijosNull}{dStr /$d$, nat /$n$}{bool}{d $\neq$ NULL}
\tadAxioma{HijosNull(d,n)}{$\textbf{if} \ n == 0 \ \emph{then} \\ \hspace*{10px} true \\ \textbf{else} \\ \hspace*{10px} (d.prefijos[n-1] == NULL) \yluego HijosNull(d,n-1) \\  \textbf{fi}$}



\subsubsection{Funcion de Abstraccion}

\Abs[dStr]{dicc(string, $\alpha$)}[e]{d}{($\forall c$: string)(def?($c, d$) $=$ esClave?($c, e$) $\yluego$ \\
 (def?($c, d$) \impluego obtener($c, d$) $=$ significado($c, e$)))}

\textbf{Funciones Auxiliares:}

  ~
    
\tadOperacion{esClave?}{string /$c$, dStr /$e$}{bool}{}
\tadAxioma{esClave?(c,e)}{ $\textbf{if} \ vacia?(c) \ \emph{then} $\\$ \hspace*{10px} e.significado \neq NULL $\\$ \textbf{else} \\ \hspace*{10px} e.caracteres[ord(prim(c))] \neq NULL \yluego esClave?(fin(c), e.caracteres[ord(prim(c))]) $\\$  \textbf{fi}$}


\tadOperacion{significado}{string /$c$, dStr /$e$}{bool}{esClave?($c, e$)}
\tadAxioma{significado(c,e)}{ $\textbf{if} \ vacia?(c) \ \emph{then} \\ \hspace*{10px} e.significado \\ \textbf{else} \\ \hspace*{10px} significado(fin(c), e.caracteres[ord(prim(c))]) \\  \textbf{fi}$}

%%%%%%%%%%%%%%%%%%%%%%%%%%

\begin{Algoritmos}



\begin{algoritmo}{\textbf{iVacio}}{}{dStr}
    	     (res.significado) $\gets$ NULL \com*{$O(1)$}
			 (res.prefijos) $\gets$ CrearArreglo(256) \com*{$O(1)$}
			 \For (\com*[f]{$O(256)$}) {i $\leftarrow$ 0 to 255}{
				(res.prefijos[i]) $\gets$ NULL \com*{$O(1)$}
			 }
    	
\end{algoritmo}
\datosAlgoritmo{} % Descripción
  {} % Pre
  {} % Post
  {$O(1)$} % Complejidad
  {Por algebra de ordenes $O(1)$ + $O(1)$ + ($O(256)$ * $O(1)$) = $O(1)$} % Justificación
  
\begin{algoritmo}{\textbf{iDefinido?}}{\In{s}{string}, \In{d}{dStr($\alpha$)})}{bool}
			 i $\gets$ 0 \com*{$O(1)$}
			 noesta $\gets$ false \com*{$O(1)$}
			 actual $\gets$\ d \com*{$O(1)$}
			 \While(\com*[f]{$O(Longitud(s))$}){ i < Longitud(s) $\wedge$ $\neg$ noesta}{ 
				\If(\com*[f]{$O(1)$}) {actual.prefijos [ord(s[i])] = NULL}{ 
					noesta $\gets$ true \com*{$O(1)$}
				}
				actual $\gets$ (actual.prefijos[ord(c[i])]) \com*{$O(1)$}
				i $\gets$ i + 1 \com*{$O(1)$}
			 }
			 res $\gets$ ($\neg$ noesta $\wedge$ $\neg$(actual.significado = NULL)) \com*{$O(2)$}
			    	
\end{algoritmo}
\datosAlgoritmo{} % Descripción
  {} % Pre
  {} % Post
  {$O(L)$} % Complejidad
  {En este caso $O(Longitud(s))$ = $O(L)$. Por algebra de ordenes $O(1)$ + $O(1)$ + $O(1)$ + ($O(L)$ * ($O(1)$ + $O(1)$ + $O(1)$ + $O(1)$)) + $O(2)$ = $O(L)$} % Justificación

\begin{algoritmo}{\textbf{iDefinir}}{\In{c}{string}, \In{a}{$\alpha$}, \Inout{d}{dStr}}{}
			 i $\gets$ 0 \com*{$O(1)$}
			 actual $\gets$\ d \com*{$O(1)$}
			 \While (\com*[f]{$O(Longitud(c))$}){i < Longitud(c)}{ 
				\If(\com*[f]{$O(1)$}){ actual.prefijos [ord(c[i])] = NULL}{ 
					(actual.prefijos[ord(c[i])]) $\gets$ Vacio() \com*{$O(1)$}
				}
				actual $\gets$ (actual.prefijos[ord(c[i])]) \com*{$O(1)$}
				i $\gets$ i + 1 \com*{$O(1)$}
			 }
			 actual.significado $\gets$ a \com*{$O(copy(a))$}  	
\end{algoritmo}
\datosAlgoritmo{} % Descripción
  {} % Pre
  {} % Post
  {$O(L) + copy(a)$} % Complejidad
  {En este caso $O(Longitud(c))$ = $O(L)$. Por algebra de ordenes $O(1)$ + $O(1)$ + ($O(L)$ * ($O(1)$ + $O(1)$ + $O(1)$ + $O(1)$)) + $O(1)$ = $O(L)$} % Justificación
  
\begin{algoritmo}{\textbf{iObtener}}{\In {d}{dStr}, \In {c}{string}}{$\alpha$}
			 i $\leftarrow$ 0 \com*{$O(1)$}
			 actual $\leftarrow$\ d \com*{$O(1)$}
			 \While(\com*[f]{$O(Longitud(c))$}) {i < Longitud(c)}{ 
				actual $\leftarrow$ (actual.prefijos[ord(c[i])]) \com*{$O(1)$}
				i $\leftarrow$ i + 1 \com*{$O(1)$}
			 }
             
			 res $\leftarrow$ (actual.significado) \com*{$O(1)$}
			    	
\end{algoritmo}
\datosAlgoritmo{} % Descripción
  {} % Pre
  {} % Post
  {$O(L)$} % Complejidad
  {En este caso $O(Longitud(c))$ = $O(L)$. Por algebra de ordenes $O(1)$ + $O(1)$ + ($O(L)$ * ($O(1)$ + $O(1)$)) + $O(1)$ = $O(L)$} % Justificación
  
\begin{algoritmo}{\textbf{iBorrar}}{\Inout {d}{dStr}, \In {c}{string} }{}
			 i $\gets$ 0 \com*{$O(1)$}
             $pila \gets Vacia()$ \com*{$O(1)$}
			 actual $\gets$\ d \com*{$O(1)$}
             $Apilar(pila, actual)$ \com*{$O(copy(actual))$}
			 \While(\com*[f]{$O(Longitud(c))$}) {i < Longitud(c)}{
             	actual $\gets$ actual.prefijos[ord(c[i])] \com*{$O(1)$}	
                $Apilar(pila, actual)$ \com*{$O(copy(actual))$}
             	i $\gets$ i + 1 \com*{$O(1)$}
			 }
			 actual.significado $\gets$ NULL \com*{$O(1)$}
             \If(\com*[f]{$O(1)$}){todoNULL?(actual.prefijos)}{
             		ant $\gets$\ actual \com*{$O(1)$}
					\While(\com*[f]{$O(1)$}) {$\neg$ hayHermanos?(actual.prefijos) $\land$ actual.significado = NULL $\land$ $\neg$ EsVacia?(pila)}{
                		\For (\com*[f]{$O(256)$}) {i $\leftarrow$ 0 to 255}{
							\If(\com*[f]{$O(1)$}){actual.prefijos[i] $\neq$ NULL }{
                        		actual.prefijos[i] $\gets$ NULL \com*{$O(1)$}
			 				}
                    	}    
                        ant $\gets$\ actual \com*{$O(1)$}
						$actual \gets Desapilar(pila)$ \com*{$O(1)$}
                    }
                    \If(\com*[f]{$O(1)$}){actual $\neq$ ant }{
                    	\For (\com*[f]{$O(256)$}) {i $\leftarrow$ 0 to 255}{
                        	\If(\com*[f]{$O(1)$}){actual.prefijos[i] = ant}{
                            	actual.prefijos[i] $\gets$ NULL \com*{$O(1)$}
                          	}
                      	}
					}
                } 	
\end{algoritmo}
\datosAlgoritmo{} % Descripción
  {} % Pre
  {} % Post
  {$O(n)$} % Complejidad
  {En este caso $O(Longitud(c))$ = $O(L)$. Por algebra de ordenes $O(1)$ + ($O(L)$ * $O(1)$) + ($O(1)$ + ($O(L)$ * $O(1)$)) + $O(1)$ = $O(L)$, ya que como prefijos es acotado hayHermanos? y todoNULL? son $O(1)$ y copiar punteros es O($1$)} % Justificación

\begin{algoritmo}{\textbf{iminimaClave}}{\In {d}{dStr}}{string}
actual $\gets$ d \com*{$O(1)$}
res $\gets$ Vacia() \com*{$O(1)$}
tengoPrefijo $\gets$ false \com*{$O(1)$}
encontreMin $\gets$ false \com*{$O(1)$}
\While(\com*[f]{$O(L)$}){$\neg$ encontreMin}{
	\eIf(\com*[f]{$O(1)$}){actual.significado $\neq$ NULL}{
    	encontreMin $\gets$ true \com*{$O(1)$}
    }{
    	i $\gets$ 0 \com*{$O(1)$}
        \While(\com*[f]{$O(256)$}){i $<$ 256 $\land$ $\neg$ tengoPrefijo}{
			\If(\com*[f]{$O(1)$}){actual.prefijos[i] $\neq$ NULL}{
            actual $\gets$ actual.prefijos[i] \com*{$O(1)$}
            tengoPrefijo $\gets$ true \com*{$O(1)$}
            res $\gets$ AgregarAtras(res, $ord^{-1}$(i)) \com*{$O(long(res)+ 1)$}
            }
            i $\gets$ i + 1 \com*{$O(1)$}
		}
        tengoPrefijo $\gets$ false \com*{$O(1)$}
	}
}    
\end{algoritmo}
\datosAlgoritmo{} % Descripción
  {} % Pre
  {} % Post
  {$O(L)$} % Complejidad
  {El while que recorre los elementos de 0 a 256 es $O(256)$ = $O(1)$. El while externo a lo sumo recorre una rama, es decir $O(L)$} % Justificación

\begin{algoritmo}{\textbf{imaximaClave}}{\In {d}{dStr}}{string}
actual $\gets$ d \com*{$O(1)$}
res $\gets$ Vacia() \com*{$O(1)$}
tengoPrefijo $\gets$ false \com*{$O(1)$}
encontreMax $\gets$ false \com*{$O(1)$}
\While(\com*[f]{$O(L)$}){$\neg$ encontreMax}{
	\eIf(\com*[f]{$O(1)$}){todoNULL?(actual.prefijos)}{
    	encontreMax $\gets$ true \com*{$O(1)$}
    }{
    	i $\gets$ 255 \com*{$O(1)$}
        \While(\com*[f]{$O(256)$}){i $\geq$ 0 $\land$ $\neg$ tengoPrefijo}{
			\If(\com*[f]{$O(1)$}){actual.prefijos[i] $\neq$ NULL}{
            actual $\gets$ actual.prefijos[i] \com*{$O(1)$}
            tengoPrefijo $\gets$ true \com*{$O(1)$}
            res $\gets$ AgregarAtras(res, $ord^{-1}$(i)) \com*{$O(long(res)+ 1)$}
            }
            i $\gets$ i - 1 \com*{$O(1)$}
		}
        tengoPrefijo $\gets$ false \com*{$O(1)$}
	}
}      	
\end{algoritmo}
\datosAlgoritmo{} % Descripción
  {} % Pre
  {} % Post
  {$O(L)$} % Complejidad
  {El while que recorre los elementos de 255 a 0 es $O(256)$ = $O(1)$. El while externo a lo sumo recorre una rama, es decir $O(L)$} % Justificación

\begin{algoritmo}{\textbf{itodoNULL?}}{\In{a}{Arreglo[n](puntero(nodo))}}{bool}
	$res \gets true $ \com*{$O(1)$}
    $i \gets 0 $ \com*{$O(1)$}
	\While(\com*[f]{$O(1)$}){i $<$ Tam(a) $\land$ res}{
    	  \If(\com*[f]{$O(1)$}){$a[i]$ $\neq$ NULL}{
      			 $res \gets false $ \com*{$O(1)$}
  		}
        $i \gets i + 1$ \com*{$O(1)$}
     }   
\end{algoritmo}
\datosAlgoritmo{} % Descripción
  {} % Pre
  {} % Post
  {$O(n)$} % Complejidad
  {Itera n $=$ Tam(a) veces y hace operaciones $O(1)$, se cuenta con la operacion Tam que es $O(1)$} % Justificación
  
\begin{algoritmo}{\textbf{ihayHermanos?}}{\In{a}{Arreglo[n](puntero(nodo))}}{bool}
    $j \gets 0 $ \com*{$O(1)$}
    $i \gets 0 $ \com*{$O(1)$}
	\While(\com*[f]{$O(1)$}){i $<$ Tam(a) $\land$ (j $<$ 2)}{
    	  \If(\com*[f]{$O(1)$}){$a[i]$ $\neq$ NULL}{
      			 $j \gets j + 1$ \com*{$O(1)$}
  		}
        $i \gets i + 1$ \com*{$O(1)$}
     }
     $res \gets j $=$ 2 $ \com*{$O(1)$}
\end{algoritmo}
\datosAlgoritmo{} % Descripción
  {} % Pre
  {} % Post
  {$O(n)$} % Complejidad
  {Itera n $=$ Tam(a) veces y hace operaciones $O(1)$, se cuenta con la operacion Tam que es $O(1)$} % Justificación

\end{Algoritmos}


\end{document}
