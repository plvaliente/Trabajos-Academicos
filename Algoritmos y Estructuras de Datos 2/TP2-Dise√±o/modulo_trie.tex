\section{Modulo DiccionarioString($\alpha $)}

\Encabezado{Notas Preliminares}
  En todos los casos, al indicar las complejidades de los algoritmos, las variables que se utilizan corresponden a:
  \vspace{-0.5em}\begin{itemize}
    \item $L$: Longitud del string en el Diccionario String pasado por par\'ametro.

  \end{itemize}
  

%%%%%%%%%%%%%%%%%%%%%%%%%%

\subsection{Interfaz}
  
  \textbf{parametros formales} \hangindent=2 \parindent \\
  \parbox{1.7cm}{\textbf{generos}} $\alpha$
  
  \textbf{se explica con}: \tadNombre{dicc(string, $\alpha$)}.
 
  \textbf{generos}: \TipoVariable{diccStr(string, $\alpha$)}.
  
   \servUsados{string}

\subsubsection{Operaciones basicas}
  
  \InterfazFuncion{Vacio{}}{}{diccStr(string, $\alpha)$}
  [true]
  {$res$ $\igobs$ vacio}%
  [$O(1)$]
  [Crea un Diccionario Vacio]
  []

  \InterfazFuncion{Definido?}{\In{s}{string}, \In{d}{diccStr(string, $\alpha$)})}{bool}
  [true]
	{$res$ $\igobs$ def?($s$, $d$)}
	[$O(Longitud(s))$]
	[Retorna si la Clave esta definida]
	[]
  

  \InterfazFuncion{Definir}{\In{s}{string} , \In{a}{$\alpha$}, \Inout{d}{diccStr(string, $\alpha)$}}{}
  [$ d \igobs d_0 $]
  {$ d \igobs$ definir($s, a, d_0$)}
  [$O$($Longitud(s)$ + copy(a))]
  [Define la Clave $s$ y el Significado $a$ en $d$]
  []
  
  \InterfazFuncion{Obtener}{\In{s}{string}, \In{d}{diccStr(string, $\alpha$)}}{$\alpha$}
  [def?($s$, $d$)]
  {alias($res$ $\igobs$ obtener($s$, $d$))}
  [$O(Longitud(s))$]
  [Retorna el Significado de la Clave $s$]
  [$res$ es modificable si y solo si $d$ es modificable]
  
    \InterfazFuncion{Borrar}{\Inout{d}{diccStr(string, $\alpha$)}, \In{c}{string}}{}
  [($d$ = $d_0$) $\wedge$ def?($s$, $d$)]
  {d $\igobs$ borrar($d_0$ , s)}
  [$O(Longitud(s))$]
  [Retorna el Significado de la Clave s]
  [Elimina la clave c y su significado de d]
 
  \InterfazFuncion{minimaClave}{\In{d}{diccStr(string, $\alpha$)}}{string}
  [Cardinal(claves(d)) > 0]
  {res $\igobs$ minimo(claves(d))}
  [$O(L)$]
  [Retorna la minima clave de mi diccionario]
  [$res$ es modificable si y solo si $d$ es modificable]
  
  ~
  
  \InterfazFuncion{maximaClave}{\In{d}{diccStr(string, $\alpha$)}}{string}
  [Cardinal(claves(d))]
  {res $\igobs$ maximo(claves(d))}
  [$O(L)$]
  [Retorna la maxima clave de mi diccionario]
  [$res$ es modificable si y solo si $d$ es modificable]
 
%%%%%%%%%%%%%%%%%%%%%%%%%%

\subsection{Representacion}

	\begin{Estructura}{diccStr$(string, \alpha)$}[dStr donde dStr es puntero(nodo)]
		\begin{Tupla}[nodo]
			\tupItem{significado}{puntero$(\alpha)$}
			\tupItem{prefijos}{arreglo[256] (puntero(nodo))}
		\end{Tupla}
	\end{Estructura}

\subsubsection{Invariante de Representacion}

\begin{enumerate}
\item La raiz representa al prefijo $" "$, su significado es siempre NULL
%\item Todas las posiciones de prefijos estan definidas (apuntan a NULL o a otro nodo)
\item No hay ninguna rama del trie que tenga Ciclos
\item Todas las hojas tienen su significado distinto a NULL, salvo que el arbol este vacio.


\end{enumerate}

\Rep[dStr][d]{\\
	(d.significado = NULL) $\wedge$ \\
    %todosDefinidos(d) $\yluego$ \\
	esArbValido(d) $\yluego$ \\
   % ($\forall$ (p $\in$ damePunteros(d))) p.vive = false $\impluego$ hijosNoViven(p, 256))
   $\textbf{if} \ esHoja?(d) \ \emph{then} \\ \hspace*{10px} true \\ \textbf{else} \\ \hspace*{10px} HojasConSignificado(d) \\  \textbf{fi}$
  }


\textbf{Funciones Auxiliares:}
% \tadOperacion{hijosNoViven}{puntero(nodo) /$p$, nat /$n$}{bool}{}
% \tadAxioma{hijosNoViven(p, n)}{$\textbf{if} \ p = NULL \ \emph{then} \\ \hspace*{10px} true \\ \textbf{else} \\ \hspace*{10px} \textbf{if} \ n = 0 \ \emph{then} \\ \hspace*{20px} true \\ \hspace*{10px} \textbf{else} \\ \hspace*{20px}  \\ (p.prefijos[n-1].vive = false $\yluego$ hijosNoViven(p.prefijos[n-1], 256)) $\yluego$ hijosNoViven(p, n-1 ) \textbf{fi} \\  \textbf{fi}$ }



% \tadOperacion{todosDefinidos}{dStr /$d$}{bool}{}
% \tadAxioma{todosDefinidos(d)}{$\textbf{if} \ d = NULL \ \emph{then} \\ \hspace*{10px} true \\ \textbf{else} \\ \hspace*{10px} todosDefinidosAux(d.prefijos, 256) \yluego todosHijosDefinidos(d, 256) \\  \textbf{fi}$}

% \tadOperacion{todosHijosDefinidos}{dStr /$d$, nat /$n$}{bool}{d $\neq$ NULL $\yluego$ $n = tam(d.prefijos)$}
% \tadAxioma{todosHijosDefinidos(d,n)}{$\textbf{if} \ n = 0 \ \emph{then} \\ \hspace*{10px} true \\ \textbf{else} \\ \hspace*{10px} todosDefinidos(d.prefijos[n-1] \yluego todosHijosDefinidos(d.prefijos, n-1) \\  \textbf{fi}$}

% \tadOperacion{todosDefinidosAux}{ad($\alpha$) /$a$, nat /$n$}{bool}{$n = tam(a)$}
% \tadAxioma{todosDefinidosAux(a,n)}{$\textbf{if} \ n = 0 \ \emph{then} \\ \hspace*{10px} true \\ \textbf{else} \\ \hspace*{10px} definido?(a, n-1) \yluego todosDefinidosAux(a, n-1) \\  \textbf{fi}$}
\tadAlinearFunciones{TodasMisHojasConSignificado}{ad($\alpha$) /$a$, nat /$n$, conj($\alpha$) /$c$}
\tadAlinearAxiomas{TodasMisHojasConSignificado(d,n)}
\tadOperacion{Loopea}{dStr /$d$, conj(dStr) /$cd$}{bool}{}
\tadAxioma{Loopea(d, cd)}{$\textbf{if} \ d = NULL \ \emph{then} \\ \hspace*{10px} false \\ \textbf{else} \\ \hspace*{10px} \textbf{if} \ perteneceAlguno (d.prefijos, 256, cd) \ \emph{then} \\ \hspace*{20px} true \\ \hspace*{10px} \textbf{else} \\ \hspace*{20px} loopeaAlguno(d.prefijos, 256, cd) \\ \hspace*{10px}  \textbf{fi} \\  \textbf{fi}$}

\tadOperacion{perteneceAlguno}{ad($\alpha$) /$a$, nat /$n$, conj($\alpha$) /$c$}{bool}{($n = tam(a)$) $\yluego$ todosDefinidos(a,n)}
\tadAxioma{perteneceAlguno(a, n, c)}{$\textbf{if} \ n = 0 \ \emph{then} \\ \hspace*{10px} false \\ \textbf{else} \\ \hspace*{10px} (a[n-1] \in c) \yluego perteneceAlguno(a, n-1, c) \\  \textbf{fi}$}

\tadOperacion{loopeaAlguno}{ad($\alpha$) /$a$, nat /$n$, conj($\alpha$) /$c$}{bool}{$n = tam(a)$) $\yluego$ todosDefinidos(a,n)}
\tadAxioma{loopeaAlguno(a, n, c)}{$\textbf{if} \ n = 0 \ \emph{then} \\ \hspace*{10px} false \\ \textbf{else} \\ \hspace*{10px} Loopea(a[n-1], Ag(a[n-1],c)) \yluego loopeaAlguno(a, n-1, c) \\  \textbf{fi}$}

\tadOperacion{HayLoop?}{dStr /$d$}{bool}{}
\tadAxioma{HayLoop?(d)}{Loopea(d, Ag(d, $\emptyset$))}

\tadOperacion{damePunteros}{dStr /$d$}{multiconj(dStr)}{}
\tadAxioma{damePunteros(d)}{$\textbf{if} \ d = NULL \ \emph{then} \\ \hspace*{10px} \emptyset \\ \textbf{else} \\ \hspace*{10px} Ag(d, damePunterosdeTodos(d.prefijos, 256)) \\  \textbf{fi}$}

\tadOperacion{damePunterosdeTodos}{ad($\alpha$) /$a$, nat /$n$}{multiconj(dStr)}{$n = tam(a)$) $\yluego$ todosDefinidos(a,n)}
\tadAxioma{damePunterosdeTodos(a,n)}{$\textbf{if} \ n = 0 \ \emph{then} \\ \hspace*{10px} \emptyset \\ \textbf{else} \\ \hspace*{10px} damePunteros(a[n-1]) \cup damePunterosdeTodos(d.prefijos, n-1) \\  \textbf{fi}$}

\tadOperacion{NoHayRepetidos}{multiconj($\alpha$) /$m$}{bool}{}
\tadAxioma{NoHayRepetidos(m)}{$\textbf{if} \ \emptyset ?(m) \ \emph{then} \\ \hspace*{10px} true \\ \textbf{else} \\ \hspace*{10px} \textbf{if} \ \# (dameUno(m), m) \neq 1 \ \emph{then} \\ \hspace*{20px} false \\ \hspace*{10px} \textbf{else} \\ \hspace*{20px} NoHayRepetidos(sinUno(m)) \\  \hspace*{10px} \textbf{fi} \\  \textbf{fi}$}

\tadOperacion{esArbValido}{dStr /$d$}{bool}{}
\tadAxioma{esArbValido(d)}{$\textbf{if} \ HayLoop?(d) \ \emph{then} \\ \hspace*{10px} false \\ \textbf{else} \\ \hspace*{10px} NoHayRepetidos(DamePunteros(d)) \\  \textbf{fi}$}


\tadOperacion{HojasConSignificado}{dStr /$d$}{bool}{}
\tadAxioma{HojasConSignificado(d)}{$\textbf{if} \ esHoja?(d) \ \emph{then} \\ \hspace*{10px} d.significado \neq NULL \\ \textbf{else} \\ \hspace*{10px} TodasMisHojasConSignificado(d, 256) \\  \textbf{fi}$}

\tadOperacion{TodasMisHojasConSignificado}{dStr /$d$, nat /$n$}{bool}{}
\tadAxioma{TodasMisHojasConSignificado(d,n)}{$\textbf{if} \ n == 0 \ \emph{then} \\ \hspace*{10px} true \\ \textbf{else} \\ \hspace*{10px}   \textbf{if} \ d.prefijos[n-1] == NULL \ \emph{then} \\ \hspace*{20px} true \\ \hspace*{10px} \textbf{else} \\ \hspace*{20px} HojasConSignificado(d.prefijos[n-1])  \yluego  \\ \hspace*{20px} TodasMisHojasConSignificado(d, n-1) \\  \hspace*{10px} \textbf{fi} \\ \textbf{fi}$}

\tadOperacion{esHoja?}{dStr /$d$}{bool}{d $\neq$ NULL}
\tadAxioma{esHoja?(d)}{HijosNull(d,256)}

\tadOperacion{HijosNull}{dStr /$d$, nat /$n$}{bool}{d $\neq$ NULL}
\tadAxioma{HijosNull(d,n)}{$\textbf{if} \ n == 0 \ \emph{then} \\ \hspace*{10px} true \\ \textbf{else} \\ \hspace*{10px} (d.prefijos[n-1] == NULL) \yluego HijosNull(d,n-1) \\  \textbf{fi}$}



\subsubsection{Funcion de Abstraccion}

\Abs[dStr]{dicc(string, $\alpha$)}[e]{d}{($\forall c$: string)(def?($c, d$) $=$ esClave?($c, e$) $\yluego$ \\
 (def?($c, d$) \impluego obtener($c, d$) $=$ significado($c, e$)))}

\textbf{Funciones Auxiliares:}

  ~
    
\tadOperacion{esClave?}{string /$c$, dStr /$e$}{bool}{}
\tadAxioma{esClave?(c,e)}{ $\textbf{if} \ vacia?(c) \ \emph{then} $\\$ \hspace*{10px} e.significado \neq NULL $\\$ \textbf{else} \\ \hspace*{10px} e.caracteres[ord(prim(c))] \neq NULL \yluego esClave?(fin(c), e.caracteres[ord(prim(c))]) $\\$  \textbf{fi}$}


\tadOperacion{significado}{string /$c$, dStr /$e$}{bool}{esClave?($c, e$)}
\tadAxioma{significado(c,e)}{ $\textbf{if} \ vacia?(c) \ \emph{then} \\ \hspace*{10px} e.significado \\ \textbf{else} \\ \hspace*{10px} significado(fin(c), e.caracteres[ord(prim(c))]) \\  \textbf{fi}$}

%%%%%%%%%%%%%%%%%%%%%%%%%%

\begin{Algoritmos}



\begin{algoritmo}{\textbf{iVacio}}{}{dStr}
    	     (res.significado) $\gets$ NULL \com*{$O(1)$}
			 (res.prefijos) $\gets$ CrearArreglo(256) \com*{$O(1)$}
			 \For (\com*[f]{$O(256)$}) {i $\leftarrow$ 0 to 255}{
				(res.prefijos[i]) $\gets$ NULL \com*{$O(1)$}
			 }
    	
\end{algoritmo}
\datosAlgoritmo{} % Descripción
  {} % Pre
  {} % Post
  {$O(1)$} % Complejidad
  {Por algebra de ordenes $O(1)$ + $O(1)$ + ($O(256)$ * $O(1)$) = $O(1)$} % Justificación
  
\begin{algoritmo}{\textbf{iDefinido?}}{\In{s}{string}, \In{d}{dStr($\alpha$)})}{bool}
			 i $\gets$ 0 \com*{$O(1)$}
			 noesta $\gets$ false \com*{$O(1)$}
			 actual $\gets$\ d \com*{$O(1)$}
			 \While(\com*[f]{$O(Longitud(s))$}){ i < Longitud(s) $\wedge$ $\neg$ noesta}{ 
				\If(\com*[f]{$O(1)$}) {actual.prefijos [ord(s[i])] = NULL}{ 
					noesta $\gets$ true \com*{$O(1)$}
				}
				actual $\gets$ (actual.prefijos[ord(c[i])]) \com*{$O(1)$}
				i $\gets$ i + 1 \com*{$O(1)$}
			 }
			 res $\gets$ ($\neg$ noesta $\wedge$ $\neg$(actual.significado = NULL)) \com*{$O(2)$}
			    	
\end{algoritmo}
\datosAlgoritmo{} % Descripción
  {} % Pre
  {} % Post
  {$O(L)$} % Complejidad
  {En este caso $O(Longitud(s))$ = $O(L)$. Por algebra de ordenes $O(1)$ + $O(1)$ + $O(1)$ + ($O(L)$ * ($O(1)$ + $O(1)$ + $O(1)$ + $O(1)$)) + $O(2)$ = $O(L)$} % Justificación

\begin{algoritmo}{\textbf{iDefinir}}{\In{c}{string}, \In{a}{$\alpha$}, \Inout{d}{dStr}}{}
			 i $\gets$ 0 \com*{$O(1)$}
			 actual $\gets$\ d \com*{$O(1)$}
			 \While (\com*[f]{$O(Longitud(c))$}){i < Longitud(c)}{ 
				\If(\com*[f]{$O(1)$}){ actual.prefijos [ord(c[i])] = NULL}{ 
					(actual.prefijos[ord(c[i])]) $\gets$ Vacio() \com*{$O(1)$}
				}
				actual $\gets$ (actual.prefijos[ord(c[i])]) \com*{$O(1)$}
				i $\gets$ i + 1 \com*{$O(1)$}
			 }
			 actual.significado $\gets$ a \com*{$O(copy(a))$}  	
\end{algoritmo}
\datosAlgoritmo{} % Descripción
  {} % Pre
  {} % Post
  {$O(L) + copy(a)$} % Complejidad
  {En este caso $O(Longitud(c))$ = $O(L)$. Por algebra de ordenes $O(1)$ + $O(1)$ + ($O(L)$ * ($O(1)$ + $O(1)$ + $O(1)$ + $O(1)$)) + $O(1)$ = $O(L)$} % Justificación
  
\begin{algoritmo}{\textbf{iObtener}}{\In {d}{dStr}, \In {c}{string}}{$\alpha$}
			 i $\leftarrow$ 0 \com*{$O(1)$}
			 actual $\leftarrow$\ d \com*{$O(1)$}
			 \While(\com*[f]{$O(Longitud(c))$}) {i < Longitud(c)}{ 
				actual $\leftarrow$ (actual.prefijos[ord(c[i])]) \com*{$O(1)$}
				i $\leftarrow$ i + 1 \com*{$O(1)$}
			 }
             
			 res $\leftarrow$ (actual.significado) \com*{$O(1)$}
			    	
\end{algoritmo}
\datosAlgoritmo{} % Descripción
  {} % Pre
  {} % Post
  {$O(L)$} % Complejidad
  {En este caso $O(Longitud(c))$ = $O(L)$. Por algebra de ordenes $O(1)$ + $O(1)$ + ($O(L)$ * ($O(1)$ + $O(1)$)) + $O(1)$ = $O(L)$} % Justificación
  
\begin{algoritmo}{\textbf{iBorrar}}{\Inout {d}{dStr}, \In {c}{string} }{}
			 i $\gets$ 0 \com*{$O(1)$}
             $pila \gets Vacia()$ \com*{$O(1)$}
			 actual $\gets$\ d \com*{$O(1)$}
             $Apilar(pila, actual)$ \com*{$O(copy(actual))$}
			 \While(\com*[f]{$O(Longitud(c))$}) {i < Longitud(c)}{
             	actual $\gets$ actual.prefijos[ord(c[i])] \com*{$O(1)$}	
                $Apilar(pila, actual)$ \com*{$O(copy(actual))$}
             	i $\gets$ i + 1 \com*{$O(1)$}
			 }
			 actual.significado $\gets$ NULL \com*{$O(1)$}
             \If(\com*[f]{$O(1)$}){todoNULL?(actual.prefijos)}{
             		ant $\gets$\ actual \com*{$O(1)$}
					\While(\com*[f]{$O(1)$}) {$\neg$ hayHermanos?(actual.prefijos) $\land$ actual.significado = NULL $\land$ $\neg$ EsVacia?(pila)}{
                		\For (\com*[f]{$O(256)$}) {i $\leftarrow$ 0 to 255}{
							\If(\com*[f]{$O(1)$}){actual.prefijos[i] $\neq$ NULL }{
                        		actual.prefijos[i] $\gets$ NULL \com*{$O(1)$}
			 				}
                    	}    
                        ant $\gets$\ actual \com*{$O(1)$}
						$actual \gets Desapilar(pila)$ \com*{$O(1)$}
                    }
                    \If(\com*[f]{$O(1)$}){actual $\neq$ ant }{
                    	\For (\com*[f]{$O(256)$}) {i $\leftarrow$ 0 to 255}{
                        	\If(\com*[f]{$O(1)$}){actual.prefijos[i] = ant}{
                            	actual.prefijos[i] $\gets$ NULL \com*{$O(1)$}
                          	}
                      	}
					}
                } 	
\end{algoritmo}
\datosAlgoritmo{} % Descripción
  {} % Pre
  {} % Post
  {$O(n)$} % Complejidad
  {En este caso $O(Longitud(c))$ = $O(L)$. Por algebra de ordenes $O(1)$ + ($O(L)$ * $O(1)$) + ($O(1)$ + ($O(L)$ * $O(1)$)) + $O(1)$ = $O(L)$, ya que como prefijos es acotado hayHermanos? y todoNULL? son $O(1)$ y copiar punteros es O($1$)} % Justificación

\begin{algoritmo}{\textbf{iminimaClave}}{\In {d}{dStr}}{string}
actual $\gets$ d \com*{$O(1)$}
res $\gets$ Vacia() \com*{$O(1)$}
tengoPrefijo $\gets$ false \com*{$O(1)$}
encontreMin $\gets$ false \com*{$O(1)$}
\While(\com*[f]{$O(L)$}){$\neg$ encontreMin}{
	\eIf(\com*[f]{$O(1)$}){actual.significado $\neq$ NULL}{
    	encontreMin $\gets$ true \com*{$O(1)$}
    }{
    	i $\gets$ 0 \com*{$O(1)$}
        \While(\com*[f]{$O(256)$}){i $<$ 256 $\land$ $\neg$ tengoPrefijo}{
			\If(\com*[f]{$O(1)$}){actual.prefijos[i] $\neq$ NULL}{
            actual $\gets$ actual.prefijos[i] \com*{$O(1)$}
            tengoPrefijo $\gets$ true \com*{$O(1)$}
            res $\gets$ AgregarAtras(res, $ord^{-1}$(i)) \com*{$O(long(res)+ 1)$}
            }
            i $\gets$ i + 1 \com*{$O(1)$}
		}
        tengoPrefijo $\gets$ false \com*{$O(1)$}
	}
}    
\end{algoritmo}
\datosAlgoritmo{} % Descripción
  {} % Pre
  {} % Post
  {$O(L)$} % Complejidad
  {El while que recorre los elementos de 0 a 256 es $O(256)$ = $O(1)$. El while externo a lo sumo recorre una rama, es decir $O(L)$} % Justificación

\begin{algoritmo}{\textbf{imaximaClave}}{\In {d}{dStr}}{string}
actual $\gets$ d \com*{$O(1)$}
res $\gets$ Vacia() \com*{$O(1)$}
tengoPrefijo $\gets$ false \com*{$O(1)$}
encontreMax $\gets$ false \com*{$O(1)$}
\While(\com*[f]{$O(L)$}){$\neg$ encontreMax}{
	\eIf(\com*[f]{$O(1)$}){todoNULL?(actual.prefijos)}{
    	encontreMax $\gets$ true \com*{$O(1)$}
    }{
    	i $\gets$ 255 \com*{$O(1)$}
        \While(\com*[f]{$O(256)$}){i $\geq$ 0 $\land$ $\neg$ tengoPrefijo}{
			\If(\com*[f]{$O(1)$}){actual.prefijos[i] $\neq$ NULL}{
            actual $\gets$ actual.prefijos[i] \com*{$O(1)$}
            tengoPrefijo $\gets$ true \com*{$O(1)$}
            res $\gets$ AgregarAtras(res, $ord^{-1}$(i)) \com*{$O(long(res)+ 1)$}
            }
            i $\gets$ i - 1 \com*{$O(1)$}
		}
        tengoPrefijo $\gets$ false \com*{$O(1)$}
	}
}      	
\end{algoritmo}
\datosAlgoritmo{} % Descripción
  {} % Pre
  {} % Post
  {$O(L)$} % Complejidad
  {El while que recorre los elementos de 255 a 0 es $O(256)$ = $O(1)$. El while externo a lo sumo recorre una rama, es decir $O(L)$} % Justificación

\begin{algoritmo}{\textbf{itodoNULL?}}{\In{a}{Arreglo[n](puntero(nodo))}}{bool}
	$res \gets true $ \com*{$O(1)$}
    $i \gets 0 $ \com*{$O(1)$}
	\While(\com*[f]{$O(1)$}){i $<$ Tam(a) $\land$ res}{
    	  \If(\com*[f]{$O(1)$}){$a[i]$ $\neq$ NULL}{
      			 $res \gets false $ \com*{$O(1)$}
  		}
        $i \gets i + 1$ \com*{$O(1)$}
     }   
\end{algoritmo}
\datosAlgoritmo{} % Descripción
  {} % Pre
  {} % Post
  {$O(n)$} % Complejidad
  {Itera n $=$ Tam(a) veces y hace operaciones $O(1)$, se cuenta con la operacion Tam que es $O(1)$} % Justificación
  
\begin{algoritmo}{\textbf{ihayHermanos?}}{\In{a}{Arreglo[n](puntero(nodo))}}{bool}
    $j \gets 0 $ \com*{$O(1)$}
    $i \gets 0 $ \com*{$O(1)$}
	\While(\com*[f]{$O(1)$}){i $<$ Tam(a) $\land$ (j $<$ 2)}{
    	  \If(\com*[f]{$O(1)$}){$a[i]$ $\neq$ NULL}{
      			 $j \gets j + 1$ \com*{$O(1)$}
  		}
        $i \gets i + 1$ \com*{$O(1)$}
     }
     $res \gets j $=$ 2 $ \com*{$O(1)$}
\end{algoritmo}
\datosAlgoritmo{} % Descripción
  {} % Pre
  {} % Post
  {$O(n)$} % Complejidad
  {Itera n $=$ Tam(a) veces y hace operaciones $O(1)$, se cuenta con la operacion Tam que es $O(1)$} % Justificación

\end{Algoritmos}
